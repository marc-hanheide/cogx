\documentclass[letterpaper]{article}

\usepackage{aaai}
\usepackage{times}
\usepackage{helvet}
\usepackage{courier}
\usepackage{graphicx}
\usepackage{stfloats}
\usepackage{color}


\begin{document}

 \section{Evaluation}

To evaluate our progress toward building a cognitive system capabile of reasoning about these regions, we conducted the an experiment focusing on the following questions:
\begin{itemize}
\item{How well do anchor points capture context-dependent spatial regions?}
\item{When provided with an example definition of a CDSR, how well does our appraoch identify the CDSR in the new situation?}
\end{itemize}
 Our materials are six classrooms (two simulated and four real) and two simulated studio apartments. To determine how people consider CDRSs, we asked \textbf{SOMENUMBER} of users to draw polygons representing a list of region types for seach room, users were also shown a polygon drawn by another and asked to determine if it was an acceptable depiction of the region. All users provided acceptable regions, and therefore, we define the \textit{target region} as the union of the user-defined polygons.

We consider a \textit{problem instance} to be a room and a sought CDSR type (e.g., "front"). For each room containing a CDSR of the sought type, we generate an \textit{inferred region} our analogical approach. To assess the quality of the transfer, we use the following measures:

\begin{equation}
	precision=\frac{area(infered region \cap target region)}{area(infered region)}
\end{equation}
\begin{equation}
	recall=\frac{area(infered region \cap target region)}{area(target region)}
\end{equation}

As a baseline, we will compare our results against (1) the manually encoded anchors for the problem instance and (2) the entire room.

\end{document}