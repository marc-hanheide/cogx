%%%%%%%%%%%%%%%%%%%%%%%%%%%%%%%%%%%%%%%%%%%%%%%%%%%%%%%%%%%%%%%%%%%%%%%%%%%%%%%%
%2345678901234567890123456789012345678901234567890123456789012345678901234567890
%        1         2         3         4         5         6         7         8

%\documentclass[letterpaper, 10 pt, conference]{ieeeconf}  % Comment this line out
                                                          % if you need a4paper
\documentclass[a4paper, 10pt, conference]{ieeeconf}      % Use this line for a4
                                                          % paper

\IEEEoverridecommandlockouts                              % This command is only
                                                          % needed if you want to
                                                          % use the \thanks command
\overrideIEEEmargins
% See the \addtolength command later in the file to balance the column lengths
% on the last page of the document



% The following packages can be found on http:\\www.ctan.org
%\usepackage{graphics} % for pdf, bitmapped graphics files
%\usepackage{epsfig} % for postscript graphics files
%\usepackage{mathptmx} % assumes new font selection scheme installed
%\usepackage{times} % assumes new font selection scheme installed
%\usepackage{amsmath} % assumes amsmath package installed
%\usepackage{amssymb}  % assumes amsmath package installed

\title{\LARGE \bf
Supervised Learning Of Hidden and Non-Hidden $0$-order Affordances
}

%\author{ \parbox{3 in}{\centering Huibert Kwakernaak*
%         \thanks{*Use the $\backslash$thanks command to put information here}\\
%         Faculty of Electrical Engineering, Mathematics and Computer Science\\
%         University of Twente\\
%         7500 AE Enschede, The Netherlands\\
%         {\tt\small h.kwakernaak@autsubmit.com}}
%         \hspace*{ 0.5 in}
%         \parbox{3 in}{ \centering Pradeep Misra**
%         \thanks{**The footnote marks may be inserted manually}\\
%        Department of Electrical Engineering \\
%         Wright State University\\
%         Dayton, OH 45435, USA\\
%         {\tt\small pmisra@cs.wright.edu}}
%}

%\author{Huibert Kwakernaak and Pradeep Misra% <-this % stops a space
%\thanks{This work was not supported by any organization}% <-this % stops a space
%\thanks{H. Kwakernaak is with Faculty of Electrical Engineering, Mathematics and Computer Science,
%        University of Twente, 7500 AE Enschede, The Netherlands
%        {\tt\small h.kwakernaak@autsubmit.com}}%
%\thanks{P. Misra is with the Department of Electrical Engineering, Wright State University,
%        Dayton, OH 45435, USA
%        {\tt\small pmisra@cs.wright.edu}}%
%}

\author{Aitor Aldoma, Markus Vincze\\
ACIN - Vienna University of Technology\\
{\tt\small {aldoma,vincze}@tuwien.ac.at}
% For a paper whose authors are all at the same institution,
% omit the following lines up until the closing ``}''.
% Additional authors and addresses can be added with ``\and'',
% just like the second author.
% To save space, use either the email address or home page, not both
\and
Federico Tombari\\
Computer Vision Lab\\
DEIS - ARCES, University of Bologna\\
{\tt\small {federico.tombari}@unibo.it}
}


\begin{document}



\maketitle
\thispagestyle{empty}
\pagestyle{empty}


%%%%%%%%%%%%%%%%%%%%%%%%%%%%%%%%%%%%%%%%%%%%%%%%%%%%%%%%%%%%%%%%%%%%%%%%%%%%%%%%
\begin{abstract}

%This paper focuses on the learning and detection of $0$-order affordances.

\end{abstract}


%%%%%%%%%%%%%%%%%%%%%%%%%%%%%%%%%%%%%%%%%%%%%%%%%%%%%%%%%%%%%%%%%%%%%%%%%%%%%%%%
\section{INTRODUCTION}

From a robotic perspective being able to understand a
scene and moreover, understanding which are the interaction
possibilities that are provided in a specific environment, are
a key capability for a task-guided robotic agent. What an
environment affords depends strongly on two factors: (i) the
objects and their configuration in the world and second,
the interaction capabilities embodied on a specific agent.
The combination of both factors is coined under the term
affordance in the literature:

\begin{quote} "Affordances relate the utility of things, events, and
places to the needs of animals and their actions in fulfilling
them [...]. Affordances themselves are perceived and, in
fact, are the essence of what we perceive." \end{quote}

This quote provides an intuitive definition of the term affordance
and moreover, poses an opportunity to motivate our analysis on
affordances from a robotic perspective.	

The affordances of an object - what we term under $0$-order affordances -
are supported by geometrical -local or global- properties of the object. For instance, objects like chairs or sofas
can be used for sitting as they provide a parallel surface to the ground and a perpendicular one used to lean back and mugs, bowls
and in general containers, are used for liquid-containment because they provide a closed concavity. $0$-order affordances
do not depend solely on the geometry of the objects but also on their configuration in the world. Liquid containers can only
fulfill their function if they are in an upright pose or objects like sofas and chair can be used for sitting when found
in a specific pose. 

The following NEEDS TO BE STRUCTURED AND PHRASED PROPERLY!!

\hspace{5 mm}

Moreover, if we think about perceiving affordances directly from a certain viewpoint, 
things get even more interesting as the ... TODO: for example a chair seen from behind does not afford sittable ...

\hspace{2 mm}

From a robotics perspective, do we really want to be able to ONLY perceive affordances when directly perceivable.... We can manipulate
the environment ... Contradict Gibson!!

\hspace{2 mm}

Object recognition based on CAD models...

\hspace{2 mm}

Supervised-Learning of affordances based on geometry for CAD models...

\hspace{2 mm}

Talk about why stable poses can used for this things because we live in an structured environment and man-made objects have been
designed to fulfill their functions usually when found in a certain stable pose... Objects that fulfill the same function share
at least one stable plane!!

\hspace{2 mm}

If we solve object recognition => we get a direct mapping for affordances

\hspace{2 mm}

%Gibson says "[...] affordances themselves are perceived [...] "
%\hspace{5 mm}
%\begin{quote} "There’s little we can find in common to all chairs – except
%for their intended use." \end{quote}

%This quotes extracted from ancient literature about affordances emphasize tha

%%%%%%%%%%%%%%%%%%%%%%%%%%%%%%%%%%%%%%%%%%%%%%%%%%%%%%%%%%%%%%%%%%%%%%%%%%%%%%%%
\section{RELATED WORK}





%%%%%%%%%%%%%%%%%%%%%%%%%%%%%%%%%%%%%%%%%%%%%%%%%%%%%%%%%%%%%%%%%%%%%%%%%%%%%%%%
\section{LEARNING $0$-ORDER AFFORDANCES}

Which affordances

Why use stable planes?

\subsection{Affordance labelling on CAD models}

Stable planes

Picture of the tool

\subsection{Classifiers}

Evaluation of different classifiers for one or two affordances.

\section{OBJECT RECOGNITION}

Evaluation of different descriptors

\section{DETECTING HIDDEN AND NON-HIDDEN AFFORDANCES}

Explain how once the pose is retrieved from the object recognition module,
how the stable pose is found if any.

\section{EVALUATION}

Evaluation of the whole process.

%%%%%%%%%%%%%%%%%%%%%%%%%%%%%%%%%%%%%%%%%%%%%%%%%%%%%%%%%%%%%%%%%%%%%%%%%%%%%%%%
\section{CONCLUSIONS AND FUTURE WORKS}

\subsection{Conclusions}

\subsection{Future Works}

%%%%%%%%%%%%%%%%%%%%%%%%%%%%%%%%%%%%%%%%%%%%%%%%%%%%%%%%%%%%%%%%%%%%%%%%%%%%%%%%
\section{ACKNOWLEDGMENTS}

The authors gratefully acknowledge the contribution of National Research Organization and reviewers' comments.


%%%%%%%%%%%%%%%%%%%%%%%%%%%%%%%%%%%%%%%%%%%%%%%%%%%%%%%%%%%%%%%%%%%%%%%%%%%%%%%%

References are important to the reader; therefore, each citation must be complete and correct. If at all possible, references should be commonly available publications.

\begin{thebibliography}{99}

\bibitem{c1}
J.G.F. Francis, The QR Transformation I, {\it Comput. J.}, vol. 4, 1961, pp 265-271.

\bibitem{c2}
H. Kwakernaak and R. Sivan, {\it Modern Signals and Systems}, Prentice Hall, Englewood Cliffs, NJ; 1991.

\bibitem{c3}
D. Boley and R. Maier, "A Parallel QR Algorithm for the Non-Symmetric Eigenvalue Algorithm", {\it in Third SIAM Conference on Applied Linear Algebra}, Madison, WI, 1988, pp. A20.

\end{thebibliography}

\end{document}
