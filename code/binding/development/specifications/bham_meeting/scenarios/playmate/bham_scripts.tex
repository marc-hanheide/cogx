\documentclass{article}
\usepackage{verbatim}
\usepackage{amssymb}
\usepackage{amsmath}
\usepackage{mdwlist}
\usepackage{graphicx}

\title{Bham scenario specification}
\date{\today}
\author{CoSy}


\begin{document}
\maketitle
\begin{abstract}
Specification of the Bham scenarios...
\end{abstract}

%%%%%%%%%%%%%%%%%%%%%%%%%%%%%%%%%%%%%%%%%%%%%%%%%%%%%%%%%%%%%%%%%%%%%%%%%%%%%%%%%%%%%%%%

\section{Script -- level 1}
Some of the science demostrated through this scenario:
\begin{enumerate}
\item Planning of sensing actions and visual processing.
\item Planning of clarifying questions.
\item Ability to recognise objects from a variety of views.
\item Ability to recognise complex actions (composed of sequences of
  actions).  
\item Ability to use learned categories for shape, colour and
  projective spatial relations.
\item Ability to understand which events in a sequence an utterance
  refers to.
\item Ability to reason about episodes and sequences of events.
\end{enumerate}

Before the demo we could record movies of the robot learning about
shape, colour and projective relations using the objects to be used in
the demonstration. Also, before the demo the games will be learned
off-line, using the complex action recognition system developed at
Birmingham. Learning on-line will be too challenging.

The objects to be used in the demonstration are (other than the human):
\begin{itemize}
\item \emph{part1}: a red triangle, a blue triangle, a blue square,
  and a red square, a blue car and a red car (cars being used as
  beacons/flags to which the shapes will be moved to during the
  color/shape game).
\item \emph{part2}: a car, and an occluding object, say a wall (car
  placed behind the wall) or a box (car placed within the box). 
\end{itemize}

In the shape game triangles go to the left, squares to the right.  In
the color game shapes go to the car of their color, wherever it is.

For Part1, the human sits next to the robot. There are two cars as
landmarks either side of the table in front of the robot. A red one is
on the left. A blue one is on the right. The script is as follows:

Recognition of complex action sequences from vision + recognizing the
game being played:
\begin{enumerate}
\item \label{lvl1:step1} 
  H: ``I am playing a game.''\\
  R: ``OK.''

\item \label{lvl1:step2} The human puts a red triangle in front of the
  robot. He retracts his arm.

\item \label{lvl1:step3} The human reaches out, grasps the triangle,
  and pushes it to the left car. He then pulls back the hand.

\item \label{lvl1:step4} The human withdraws the shape away from the
  left car (so that it is not in view).

\item \label{lvl1:step5}
  H: ``Which game am I playing?''\\
  R: ``I do not know, the shape game or the colour game.''

\item \label{lvl1:step6} The human puts a red square in front of the
  robot.  Same as Step~\ref{lvl1:step2}

\item \label{lv1:step7} The human moves the red square to the right
  car.  Same as Step~\ref{lvl1:step3}.

\item \label{lvl1:step8} The human withdraws the red square from the
  right car.  Same as Step~\ref{lvl1:step4}.

\item \label{lvl1:step9}
  H: ``Which game am I playing?''\\
  Same as Step~\ref{lvl1:step5}.

\item \label{lvl1:step10}
  R: ``You are playing the shape game.''\\
  We could instead have Step~\ref{lvl1:step6} happen here.

\item \label{lvl1:step11}
  The human withdraws the red square from the right car.
  same as Step~\ref{lvl1:step8}.

\item \label{lvl1:step12}
  The human swaps the cars.

\item \label{lvl1:step13}
  Human puts a red triangle down and withdraws hand.

\item \label{lvl1:step14}
  H reaches and pushes red triangle to the right car.

\item \label{lvl1:step15}
  H: ``Which game?''\\
  R: ``The color game.''

\item \label{lvl1:step16}
  H withdraws the red triangle.
  Same as in Step~\ref{lvl1:step4}

\item \label{lvl1:step17} (We now start on the ability to play the
  game, including the planning of sensory processing to support
  action).\\
  The human puts down the red triangle in front of the robot.
  Analysis similar to Step~\ref{lvl1:step2}

\item \label{lvl1:step18}  
  H: ``Play the colour game.''
  
\item \label{lvl1:step19} 
  The robot picks up the shape and puts it next to the red car on
  the right.

\item \label{lvl1:step20}
  H removes the shape (red triangle) from the car.\\
  See Step~\ref{lvl1:step4}.

\item \label{lvl1:step21}
  Human puts the blue square in front of the robot.\\
  H: ``Play the shape game.''\\
  The robot picks up the shape and puts it next to the right hand
  (red) car.
  See Step~\ref{lvl1:step17}-\ref{lvl1:step20}.
\end{enumerate}

% \begin{figure}[ht]
% \centering
% \includegraphics[width=\linewidth]{example1_msc.png}
% \caption{
%   Some simple msc-graph can be generated to clarify interaction
%   between things in the the scenario. Use the
%   \texttt{../gen\_msc.sh}-script to generate pngs from all
%   msc-files. mscgen is documented here:
%   \texttt{http://www.mcternan.me.uk/mscgen/}. It's VERY simple}
% \label{fig:lvl1}
% \end{figure}

%%%%%%%%%%%%%%%%%%%%%%%%%%%%%%%%%%%%%%%%%%%%%%%%%%%%%%%%%%%%%%%%%%%%%%%%%%%%%%%%%%%%%%%%

\section{Script -- level 2}
The scripts are discussed here in 'slightly' more detail...

\begin{enumerate}
\item \label{lvl2:step1}
  H: ``I am playing a game.''\\
  R: ``OK.''
  \begin{enumerate}
  \item \label{lvl1:step1:detail0} The robot produces color and
    position representations of the cars + Car recognition (viewpoint
    planning may be involved here -- more details later).
  \item \label{lvl2:step1:detail1} Robot produces a representation
    (where?) that it is now ``in'' a game episode.
  \item \label{lvl2:step1:detail2} Robot produces as acknowledgement
    utterance.
  \end{enumerate}

\item \label{lvl2:step2}
  The human puts a red triangle in front of the robot.
  \begin{enumerate}
  \item \label{lvl2:step2:detail} The robot has to notice the object
    (and track the hand before this?).
  \item \label{lvl2:step2:detail2} The robot has to extract color,
    shape and position of objects.
  \end{enumerate}

\item \label{lvl2:step3}
  The human reaches out, grasps the triangle, and pushes it to the
  left car. He then pulls back the hand.
  \begin{enumerate}
  \item \label{lvl2:step3:detail1}
    The robot tracks the hand and the object throughout the movement.
  \item \label{lvl2:step3:detail2} The robot has to recognize the
    ``reach'', ``push to car'', ``retract'' sequence as a 'game
    move'.
  \item \label{lvl2:step3:detail3} The robot should recognize that it
    is playing the color game or the shape game.
  \item \label{lvl2:step3:detail4} The robot extracts shape, color and
    position in the new scene.
  \end{enumerate}

\item \label{lvl2:step4}
  The human withdraws the shape from the left car.
  \begin{enumerate}
  \item \label{lvl2:step4:detail1} The robot recognizes complex
    actions such as ``reach'', ``pull to human''.
  \item \label{lvl2:step4:detail2} The robot extracts color and
    position for the cars.
  \end{enumerate}

\item \label{lvl2:step5}
  H: ``Which game am I playing?''\\
  R: ``I do not know, the shape game or the color game.''
  \begin{enumerate}
  \item \label{lvl2:step5:detail1}Understand the utterance, and that
    it is being asked a question, and that it refers to a game.
  \item \label{lvl2:step5:detail2} Ascertain the answer and deliver
    it.
  \end{enumerate}

\item \label{lvl2:step6}
  The human puts a red square in front of the robot.
  Same as Step~\ref{lvl2:step2}

\item \label{lv2:step7}
  The human moves the red square to the right car.\\
  Same as Step~\ref{lvl2:step3}.

\item \label{lvl2:step8}
  The human withdraws the red square from the right car.\\
  Same as Step~\ref{lvl2:step4} \emph{except} that it concludes that it is
  playing the shape game.

\item \label{lvl2:step9}
  H: ``Which game am I playing?''\\
  Same as Step~\ref{lvl2:step5}.

\item \label{lvl2:step10}
  R: "You are playing the shape game."\\
  We could instead have Step~\ref{lvl2:step6} happen here.

\item \label{lvl2:step11}
  The human withdraws the red square from the right car. \\
  Instead we could have Step~\ref{lvl2:step8} happen here.

\item \label{lvl2:step12}
  The human swaps the cars.
  \begin{enumerate}
  \item \label{lvl2:step12:detail1} The robot will track the hand and
    cars. ``Reach'', ``pull'' actions.
  \item \label{lvl2:step12:details2} The robot re-extracts the new
    car positions, ``recognize car''.
  \end{enumerate}

\item \label{lvl2:step13}
  Human puts a red triangle down and withdraws hand.
  \begin{enumerate}
  \item \label{lvl2:step13:detail1} Robot extracts color and shape.
  \end{enumerate}
  
\item \label{lvl2:step14}
  H reaches and pushed red triangle to the right hand car.
  \begin{enumerate}
  \item \label{lvl2:step14:detail1} Robot recognizes the component
    actions ``reach'', ``push to car'', and the complex actions.
  \item \label{lvl2:step14:detail2} The robto recognizes that it is
    playing the color game.
  \end{enumerate}

\item \label{lvl2:step15}
  H: ``Which game?''\\
  R: ``The color game.''
  Same as Step~\ref{lvl2:step5} and Step~\ref{lvl2:step6}.

\item \label{lvl2:step16}
  H withdraws the red triangle.
  Same as in Step~\ref{lvl2:step4}

\item \label{lvl2:step17} (We now start on the ability to play the
  game, including the planning of sensory processing to support
  action).\\
  The human puts down the red triangle in front of the robot.
  Same as in Step~\ref{lvl2:step2}

\item \label{lvl2:step18}  
  H: ``Play the colour game.''
  \begin{enumerate}
  \item \label{lvl2:step18:detail1} Recognizes utteranc and
    understands that it is an imperative to play the color game.
  \item \label{lvl2:step18:detail2} Produces a ``move the red triangle
    to the red car (on the right)'' plan.
  \end{enumerate}
  
\item \label{lvl2:step19} The robot picks up the shape and puts it
  next to the red car on the right.
  \begin{enumerate}
  \item \label{lvl2:step19:detail1} Recognize the grasp
    location/surface (might involve moving the hand camera).
  \item \label{lvl2:step19:detail2} Plan a trajectory to grasp it.
  \item \label{lvl2:step19:detail3} Execute the visual servoing.
  \item \label{lvl2:step19:detail4} Check that it has succeeded in
    grasping the red triangle.
  \item \label{lvl2:step19:detail5} Move the red triangle to the red
    car and put it down.
  \item \label{lvl2:step19:detail6} Retract the arm.
  \item \label{lvl2:step19:detail7} Re-extract the shape, color and
    position of the red triangle.
  \item \label{lvl2:step19:detail8} Check that the goal has been
    reached.
  \end{enumerate}

\item \label{lvl2:step20}
  H removes red triangle.\\
  Analysis similar to Step~\ref{lvl2:step4}

\item \label{lvl2:step21}
  Human puts the blue square in front of the robot.\\
  H: ``Play the shape game.''\\
  The robot picks up the shape and puts it next to the right hand
  (red) car.\\
  See Step~\ref{lvl2:step17}-Step~\ref{lvl2:step20}.

\end{enumerate}

% \begin{figure}[!htcb]
% \centering
% \includegraphics[width=\linewidth]{example2_msc.png}
% \caption{
% A more detailed msc-graph could be used on this or next level...
% }
% \label{fig:lvl2}
% \end{figure}


%%%%%%%%%%%%%%%%%%%%%%%%%%%%%%%%%%%%%%%%%%%%%%%%%%%%%%%%%%%%%%%%%%%%%%%%%%%%%%%%%%%%%%%%

\section{Subarchitecture Level}

%%----------------------------------------------------------
\subsection{Step~\ref{lvl2:step1}}
Cars are in place when the system is switched on.

\subsubsection{Vision SA}
\begin{itemize}
\item Change detection (Bham-done).
\item ROIs + Scene Objects (Bham-done).
\item Feature information from ROIs (UoL) matlab wrapped + (Bham-done) Mohan.
\item ROIs - color classifier - Scene Object color (UoL) Alen + (Bham-done) Mohan.
\item ROIs - Car recognizer - Scene Object 'type' (TUD) Nico.
\end{itemize}

\subsubsection{Comsys SA}
\begin{itemize}
\item Speech Signal - speech recognition - text string (DFKI-done) Pierre.
\item Text - OpenCCG + Moloko mk4 - logical forms (DFKI-done) GJ, Trevor.
\item Text - Dialogue analysis - type of dialogue, say assertion (DFKI-done).
\end{itemize}

\subsubsection{Spatial SA}
\begin{itemize}
\item Scene Objects (VSA) - positions on the ground plane.
\item Positions - spatial relations.
\end{itemize}

\subsubsection{Binding SA}
\begin{itemize}
\item Proxies for car 1 and car2.
\item Unions created for car1 and car2.
\end{itemize}

\subsubsection{Motivation SA}
\begin{itemize}
\item Logical forms - posts motivation that I am playing a game.
\item Posts acknowledgement.
\end{itemize}

%%----------------------------------------------------------
\subsection{Step~\ref{lvl2:step2}}
Human puts red triangle in front of the robot.

\subsubsection{Vision SA}
\begin{itemize}
\item Tracking information (Bham-done) Somboon.
\item Change detection (Bham-done).
\item ROIs + Scene Objects (Bham-done).
\item Feature information: color, shape (Bham-done), possibly use
  (UoL) when ready.
\end{itemize}

\subsubsection{Spatial SA}
\begin{itemize}
\item Recognize ``putting'' action based on hand (and object)
  tracking.
\item Realize that something is in front of the robot.
\item Cars 1 and 2 recognized.
\end{itemize}

\subsubsection{Binding SA}
\begin{itemize}
\item Proxies for cars and objects.
\item Unions.
\end{itemize}


%%----------------------------------------------------------
\subsection{Step~\ref{lvl2:step3}}



%%----------------------------------------------------------
\subsection{Step~\ref{lvl2:step4}}




%%%%%%%%%%%%%%%%%%%%%%%%%%%%%%%%%%%%%%%%%%%%%%%%%%%%%%%%%%%%%%%%%%%%%%%%%%%%%%%%%%%%%%%%

\section{Dependency Level}

\textit{
  What else is required to generate the above information and
  behaviour. Remember the already existing specs prepared prior to the
  meeting: \texttt{http://www.dfki.de/\~\ henrikj/doc\_test/specifications/main/html/}}


%%%%%%%%%%%%%%%%%%%%%%%%%%%%%%%%%%%%%%%%%%%%%%%%%%%%%%%%%%%%%%%%%%%%%%%%%%%%%%%%%%%%%%%%

\section{Component Level}

\textit{
  First pass as describing the behaviours in terms of the components
  that are involved.}


%%%%%%%%%%%%%%%%%%%%%%%%%%%%%%%%%%%%%%%%%%%%%%%%%%%%%%%%%%%%%%%%%%%%%%%%%%%%%%%%%%%%%%%%

\section{Responsibility Level}
Some details specified at the subarchitecture-level specifications (level 3)...


%%%%%%%%%%%%%%%%%%%%%%%%%%%%%%%%%%%%%%%%%%%%%%%%%%%%%%%%%%%%%%%%%%%%%%%%%%%%%%%%%%%%%%%%

\end{document}

