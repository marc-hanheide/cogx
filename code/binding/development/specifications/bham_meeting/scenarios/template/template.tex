
\documentclass{article}
\usepackage{verbatim}
\usepackage{amssymb}
\usepackage{amsmath}
\usepackage{mdwlist}
\usepackage{graphicx}

\title{Template scenario specification for Bham meeting}
\date{\today}
\author{CoSy}


\begin{document}
\maketitle
\begin{abstract}
This template defines a structure for the documentation and workflow
of the Wednesday meeting. Look at the \texttt{../yr3example/} for a
more detailed example of what is expected. 'svn cp' this entire
template dir to your own directory before you start (one per
\emph{sub}scenario is desired, i.e.\ more than one for explorer and
playmate respectively). Rename template.tex to sth. sensible and then
get going and delete the instructions as you progress. Good luck and
keep things simple! At least initially :-)
\end{abstract}

\section{Script -- level 1}
\textit{
  The script should describe the ``surface'' level of the interaction
  between the robot, the user and the environment of the robot. This
  serves as a basis for the more detailed account of the processes in
  the architecture in the following sections.}

\textit{
  make sure you keep track of the \texttt{label}s... it will make it easier for
  you to refer back to the script later on. }

\begin{enumerate}
\item \label{lvl1:sth}
  sth happens
\item \label{lvl1:sth2}
  sth else happens
\end{enumerate}

\begin{figure}[ht]
\centering
\includegraphics[width=\linewidth]{example1_msc.png}
\caption{
  Some simple msc-graph can be generated to clarify interaction
  between things in the the scenario. Use the
  \texttt{../gen\_msc.sh}-script to generate pngs from all
  msc-files. mscgen is documented here:
  \texttt{http://www.mcternan.me.uk/mscgen/}. It's VERY simple}
\label{fig:lvl1}
\end{figure}


\section{Script -- level 2}
\textit{
  The level 1 script describes what can obviously be seen when the
  robot is interacting, for example in a video uptake of the
  scenario. The next level is to describe, without any technical
  detail, what activities take place within the robot during this
  interaction. }

\textit{
  make sure the enumeration from level 1 is preserved... it makes the
  whole thing easier to read. Tip: finish level 1 first, and copy
  paste the enumeration and make subenumerations as below.}

\begin{enumerate}
\item \label{lvl2:sth}
  sth happens
\begin{enumerate}
\item \label{lvl2:sth:detail}
    sth happens inside the robot
\item \label{lvl2:sth:detail2}
    sth else happens inside the robot
\end{enumerate}
\item \label{lvl2:sth2}
  sth else happens
\begin{enumerate}
\item \label{lvl2:sth2:detail}
    sth happens inside the robot
\item \label{lvl2:sth2:detail2}
    sth else happens inside the robot
\end{enumerate}
\end{enumerate}

\begin{figure}[ht]
\centering
\includegraphics[width=\linewidth]{example2_msc.png}
\caption{
A more detailed msc-graph could be used on this or next level...
}
\label{fig:lvl2}
\end{figure}

\section{Subarchitecture Level}
\textit{
  Describe for each step in the detailed lvl2 script what the involved
  subarchitectures are and how they are involved.}

\subsection{\ref{lvl2:sth}}
\subsubsection{Subarchitecture X}
\begin{itemize}
\item do this
\item do that
\end{itemize}

\subsubsection{Subarchitecture Y}
\begin{itemize}
\item do this
\item do that
\end{itemize}

\subsection{\ref{lvl2:sth:detail}}
...
\subsection{\ref{lvl2:sth:detail2}}
...
\subsection{\ref{lvl2:sth2}}
...
\subsection{\ref{lvl2:sth2:detail}}
...
\subsection{\ref{lvl2:sth2:detail2}}
...

\section{Dependency Level}

\textit{
  What else is required to generate the above information and
  behaviour. Remember the already existing specs prepared prior to the
  meeting: \texttt{http://www.dfki.de/\~\ henrikj/doc\_test/specifications/main/html/}}


\section{Component Level}

\textit{
  First pass as describing the behaviours in terms of the components
  that are involved.}

\section{Responsibility Level}

\textit{
  Who's doing what!}

\end{document}


\end{document}
