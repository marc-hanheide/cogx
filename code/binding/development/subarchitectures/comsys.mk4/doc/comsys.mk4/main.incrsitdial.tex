%% ==============================================================
%% DOCUMENTCLASS 
%% ==============================================================

\documentclass[onecolumn,11pt]{elsart}

%% ==============================================================
%% PACKAGES 
%% ==============================================================

%% ==============================================================
%% NEW COMMANDS
%% ==============================================================

\newcommand{\cmpn}[1]{\textsf{#1}}
\newcommand{\comsys}{\textsc{comsys.mk4}}
\newcommand{\data}[1]{\texttt{#1}}

%% ==============================================================
%% DOCUMENT - FRONTMATTER
%% ==============================================================

\begin{document}

\begin{frontmatter}
\title{Incremental processing of situated human-robot dialogue \thanksref{cosy}}
\author{Geert-Jan M. Kruijff}
\address{DFKI GmbH, Saarbr\"ucken Germany\\ $\langle$\emph{\texttt{gj@dfki.de}}$\rangle$}
\thanks[cosy]{Thanks to ... CoSy, Kerstin Hadelich, Maria Staudte}


%%% Local Variables: 
%%% mode: latex
%%% TeX-master: "main.sitund.kruijff"
%%% End: 

\begin{abstract}

\end{abstract}
   

\end{frontmatter}

%% ==============================================================
%% DOCUMENT - BODY MATTER
%% ==============================================================


\section{Introduction}



\section{Implementation}


%% ==============================================================
%% DOCUMENT - APPENDIX MATTER
%% ==============================================================

\appendix

\section{Code documentation}

\subsection{IDL structures / ontological data types}

The \comsys\ working memory contains data structures of the following ontological types.  

\data{LexicalRequest} is produced by the \cmpn{GrammaticalInference} component, and contains fields for specifying the word form for which lexical data is being requested.  

\paragraph{LexicalData}


\end{document}