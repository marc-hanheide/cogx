\subsection{Qualitative Spatial Representation Extraction}

For each object that Dora detects we compute 8 spatial relations between that object and each of the objects adjacent to it; adjacency is determined using a voronoi diagram, as is standard in geometric reasoning \cite{Forbus/etal2003}. The relations we compute between each given \emph{landmark} object and its adjacent neighbour are analogous to the cardinal and intermediate points on the compass when the compass is centered on the object. The canonical directions of these relations are defined using the following vectors: $<0,1>$, $<1,1>$, $<1,0>$, $<1,-1>$, $<0,-1>$, $<-1,-1>$, $<-1,0>$, $<-1,1>$.

We generate these relations as follows. Taking each object in the room in turn to be the current landmark, we translate the origin of the room to the landmark's centroid. This results in the coordinates of the all the other objects in the room being translated into a frame of reference whose origin is the centroid of the landmark.  For each neighbour object adjacent to the landmark we compute the 8 relations by calculating the inner angle between the vector from the origin (the landmark's centroid) and the neighbour's location and the direction vector of the relation, as defined above. This angle is computed by taking the dot product of these two vectors. $\dots$



%%\from{Nick}{I have changed the below paragraph to match the actual numbers and images. I've marked what I changed}
%%Desk 1 and Desk 2 are not labeled in the Figure.

%Consider the situation in Figure~\ref{fig:dora-spatial} where the robot is facing two desks in a row\footnote{Please note that although the desks in front of the robot in Figure~\ref{fig:dora-spatial} appear distributed left to right from the robot's current position, they are actually distributed bottom to top, according to our definitions, in the global coordinate frame.}. Because we represent the desks as distinct points, \fw{(rcc8-DC Desk1 Desk2)} states that they are disjoint. To account for their relative positions, neither \fw{(leftOf Desk1 Desk2)} nor \fw{(leftOf Desk2 Desk1)} are true because the x coordinate of \fw{Desk1} is within a threshold of the x coordinate of \fw{Desk2}. The statement \fw{(below Desk1 Desk2)} is true because \fw{Desk1} has a lower y coordinate. Each desk is also related to the room by the non-tangential proper part relation, \fw{rcc8-NTPP}. These qualitative spatial relationships provide the structure necessary for analogical processing.
