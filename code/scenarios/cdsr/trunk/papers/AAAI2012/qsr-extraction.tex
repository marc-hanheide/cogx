\subsection{Qualitative Spatial Representation Extraction}

JK working on this!

%Given data from Dora's spatial model, we compute the following qualitative spatial representations. First, we extract the object and room entities from the sensor data. Next, we determine the qualitative spatial relationships between these entities. For the room, we compute its region as a convex hull around the set of points. For the objects, we consider only their centroids. For each pair of adjacent objects, we compute topological and positional relationships. Adjacency is determined by creating a voronoi diagram as is standard geometric reasoning \cite{Forbus/etal2003}. For the topological relationships, we use the region connection calculus, RCC8, \cite{Cohn:2001}. For positional relations, we use two predicates \fw{leftOf} and \fw{below}. An entity is \fw{leftOf} another if the x coordinate of its centroid is less than that of the other entity's. An entity is \fw{below} another if the y coordinate of its centroid is less than that of the other entity's. Objects whose x or y coordinates are within a threshold of one another will not be related by \fw{leftOf} or \fw{below} positional relations.

%%\from{Nick}{I have changed the below paragraph to match the actual numbers and images. I've marked what I changed}
%%Desk 1 and Desk 2 are not labeled in the Figure.

%Consider the situation in Figure~\ref{fig:dora-spatial} where the robot is facing two desks in a row\footnote{Please note that although the desks in front of the robot in Figure~\ref{fig:dora-spatial} appear distributed left to right from the robot's current position, they are actually distributed bottom to top, according to our definitions, in the global coordinate frame.}. Because we represent the desks as distinct points, \fw{(rcc8-DC Desk1 Desk2)} states that they are disjoint. To account for their relative positions, neither \fw{(leftOf Desk1 Desk2)} nor \fw{(leftOf Desk2 Desk1)} are true because the x coordinate of \fw{Desk1} is within a threshold of the x coordinate of \fw{Desk2}. The statement \fw{(below Desk1 Desk2)} is true because \fw{Desk1} has a lower y coordinate. Each desk is also related to the room by the non-tangential proper part relation, \fw{rcc8-NTPP}. These qualitative spatial relationships provide the structure necessary for analogical processing.
