\section{Evaluation}\label{sec:evaluation}

To evaluate our progress toward building a cognitive system capable of reasoning about CDSRs, we conducted the an experiment focusing on the following questions:
\begin{itemize}
\item{Are anchor points able to encode context-dependent spatial regions?}
\item{When provided with a base representation containing a labelled CDSR, how well does our approach identify the CDSR in a given target?}
\end{itemize}

\subsection{Materials}

We evaluated our approach on six classrooms (two simulated and four real) and two simulated studio apartments. The simulated rooms were based on real-life counterparts. For each room we manually encoded appropriate CDSRs that could be represented by our approach. For the classrooms these were the front and back, and the front and back rows of desks. For the studios these were the kitchen, office and living areas. These manually encoded regions were used as the base CDSRs for analogical transfers, and can be considered the training data for our evaluation. 

To determine how people define CDRSs, we asked three na\"ive users to draw polygons for each region type for each room. This task was performed using a webpage on which each user was presented with an image of the real room plus an image of the map data generated by the robot onto which the drawing could be done. The webpage (link removed for blind review) is shown in the inset in Figure~\ref{fig:ug40}. The user-defined polygons define the \textit{target regions} against which we evaluate our transfers.

We consider a \textit{problem instance} to be a room and a sought CDSR type. For each room containing a manually encoded CDSR of the sought type, we generate a \textit{transferred region} using analogical transfer. To assess the quality of the transfer, we calculate precision ($p$, the proportion of the transferred region that overlaps with the target region) and recall ($r$, the proportion of the target region that overlaps with the transferred region) as follows:

\begin{equation}
	p=\frac{area(transferred\ region \cap target\ region)}{area(transferred\ region)}
\end{equation}
\begin{equation}
	r=\frac{area(transferred\ region \cap target\ region)}{area(transferred\ region)}
\end{equation}

Using this approach we generate results showing the matches between each of the following pairs of regions: the transferred region and the appropriate target region; the CDSR we manually encoded for the target room and target region; and the region for the whole room and the target region. 
Results comparing transferred and target regions measure how well our system is able apply a single example to new situations. The comparisons between the manual annotations to the target regions measure how well the anchor points we chose capture the users' regions (who were not constrained to anchor points). Results from the whole room regions provide a baseline performance for comparison. 


\subsection{Results}
\begin{table*}
	\center
\begin{tabular}{|c|c|c|}
\hline
Transferred & Manually Encoded & Entire Room \\
\hline
$\bar{p}$=.47 $\sigma$=.37, $\bar{r}$=.46 $\sigma$=.38 & $\bar{p}$=.71 $\sigma$=.30, $\bar{r}$=.67 $\sigma$=.25 & $\bar{p}$=.17 $\sigma$=.11, $\bar{r}$=.98 $\sigma$=.05  \\
\hline
\end{tabular}
\caption{Overall Performance Compared Against Target Regions Defined by Three Users}
  \label{tab:overall}	

\end{table*}

To assess overall performance, Table~\ref{tab:overall} summarizes the results across all problem instances against user-defined target regions from three different users. The transferred regions achieved a precision of .47 ($\sigma$=.37) and a recall of .46 ($\sigma$=.38). Comparing the manually encoded regions against each target region results in a mean precision of .71 ($\sigma$=.30) and recall of .67 ($\sigma$=.25). The region defined by the room corresponds to the target region with a precision of .17 ($\sigma$=.11) and recall of .98 ($\sigma$=.05).

\begin{table*}
	\center
\begin{tabular}{|c|c|c|c|}
\hline
Region & Transferred & Manually Encoded & Entire Room \\
\hline
Front & $\bar{p}$=.32 $\sigma$=.33, $\bar{r}$=.49 $\sigma$=.41 & $\bar{p}$=.60 $\sigma$=.29, $\bar{r}$=.83 $\sigma$=.19  & $\bar{p}$=.16 $\sigma$=.10, $\bar{r}$=1 $\sigma$=0  \\
\hline
Back & $\bar{p}$=.44 $\sigma$=.37, $\bar{r}$=.56 $\sigma$=.41 & $\bar{p}$=.66 $\sigma$=.25, $\bar{r}$=.84 $\sigma$=.17  & $\bar{p}$=.11 $\sigma$=.06, $\bar{r}$=.99 $\sigma$=.03  \\
\hline
Front Rows & $\bar{p}$=.76 $\sigma$=.27, $\bar{r}$=.28 $\sigma$=.21 & $\bar{p}$=.83 $\sigma$=.31, $\bar{r}$=.50 $\sigma$=.11  & $\bar{p}$=.22 $\sigma$=.08, $\bar{r}$=1 $\sigma$=0  \\
\hline
Back Rows & $\bar{p}$=.72 $\sigma$=.30, $\bar{r}$=.42 $\sigma$=.26 & $\bar{p}$=.80 $\sigma$=.29, $\bar{r}$=.43 $\sigma$=.26  & $\bar{p}$=.19 $\sigma$=.06, $\bar{r}$=1 $\sigma$=0  \\
\hline
Kitchen & $\bar{p}$=.60 $\sigma$=.05, $\bar{r}$=.59 $\sigma$=.34 & $\bar{p}$=.78 $\sigma$=.20, $\bar{r}$=.71 $\sigma$=.13  & $\bar{p}$=.16 $\sigma$=.02, $\bar{r}$=.92 $\sigma$=.13  \\
\hline
Office & $\bar{p}$=.00 $\sigma$=.00, $\bar{r}$=.00 $\sigma$=.00 & $\bar{p}$=.78 $\sigma$=.29, $\bar{r}$=.55 $\sigma$=.20  & $\bar{p}$=.08 $\sigma$=.03, $\bar{r}$=.94 $\sigma$=.06  \\
\hline
Living Room & $\bar{p}$=.40 $\sigma$=.39, $\bar{r}$=.01 $\sigma$=.01 & $\bar{p}$=.63 $\sigma$=.34, $\bar{r}$=.54 $\sigma$=.13  & $\bar{p}$=.35 $\sigma$=.22, $\bar{r}$=.96 $\sigma$=.06  \\
\hline
\end{tabular}
\caption{Performance by Region Type}
  \label{tab:region}	
\end{table*}

To identify how our approach fairs under different conditions, Table~\ref{tab:region} separates the results by CDSR type. The mean precision for the transferred regions ranged from .76 for the front rows of classrooms to 0 for the office in studio apartments. Comparing manually encoded against target regions resulted in a minimum mean precision of .60. This occurred for the front of the classroom. The whole room precision, which is directly proportionally to the size of the target region, varied from .08 for the office to .35 for the living area.

\subsection{Discussion}

These results support the hypothesis that anchor points can provide a symbolic representation on top of sensor data for context-dependent spatial regions, and, when combined with qualitative spatial relations, they facilitate learning from a single example through analogical transfer. Collaboration with human users requires a high precision and recall, because cognitive systems must be able to understand as well as refer to these regions in human environments. Consequently, the high manually encoded precisions and recalls indicate that the defined anchor points are a reasonable starting point for a symbolic representation. Our future work seeks to further evaluate the utility of this representation by embedding the cognitive system within tasks with human users.

The transferred regions were considerably more precise (.47) when compared to the room as whole (.17), and their recalls (.46) indicate that they captured almost half of the area indicated by the human user. As we create CDSRs using anchor points defined on perceived entities, our approach performs best when the boundary of the target CDSR is closely tied to such entities. This is the case in the front rows of the classroom, with $p$ of .76 and .82 for the inferred and the manually encoded regions respectively. The system performs worst when the extent of the CDSR is defined as an unbounded area near or around particular objects. The office of a studio apartment is loosely defined as the region around the desk. This motivates one direction of future work:  expanding the vocabulary of anchor points to better capture these notions of space.
