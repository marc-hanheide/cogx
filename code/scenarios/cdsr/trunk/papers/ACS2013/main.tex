\documentclass[11pt,letterpaper]{article}
\usepackage{cogsys}
\usepackage{cogsysapa}
% \usepackage{apacite}
% \usepackage{graphicx}
\usepackage[T1]{fontenc}
\usepackage{times}
\usepackage[pdftex]{graphicx} % use this when importing PDF files
\usepackage{caption}
\usepackage{subcaption}
\usepackage{multirow}
% \usepackage{subfig}

% Put back in

% First page headings for accepted submissions.
% \cogsysheading{1}{2012}{1-18}{9/2012}{12/2012}
 % First page headings for poster submissions.
%\cogsysposterheading{First}{2012}{1-18}

% \ShortHeadings{Formatting Instructions}
              % {P.\ Langley, G.\ Hunt, and D.\ G.\ Shapiro}

\begin{document} 

\title{The Role of Context in Spatial Region Identification}
 
\author{Kate Lockwood}{klockwood@csumb.edu}
\address{ITCD Department, California State University - Monterey Bay}
\author{John D. Kelleher}{john.d.kelleher@dit.ie}
\address{Applied Intelligence Research Centre, Dublin Institute of Technology }
\author{Matthew Klenk}{klenk@parc.com}
\address{Palo Alto Research Center, Palo Alto CA }
\author{Nick Hawes}{n.a.hawes@cs.bham.ac.uk}
\address{School of Computer Science, University of Birmingham, UK }
\vskip 0.2in
 
\begin{abstract}
Representing and reasoning about spatial terms is an essential ability for cognitive systems interacting with humans in a shared environment.  Consider an agent asked to remain ``in the safe zone'' during a military engagement.  The size, shape, and location of this area depend on the capabilities of the agent, other agents in the scene, and the constraints imposed by the environment.  We call these types of regions \textit{context-dependent spatial regions} (CDSRs) due to the context-dependent nature of their extent.  We claim that understanding these regions requires integrating semantic and geometric knowledge about a dynamic environment as well as understanding how task-specific information influences spatial interpretation.  We report the results of a human study conducted to explore how people interpret three types of CDSRs and discuss the implications for developing cognitive systems with this ability.
\end{abstract}

\section{Introduction} 
Natural collaboration between humans and cognitive systems requires both participants to communicate using a shared understanding of spatial language.  Cognitive systems can already understand spatial terms defined in terms of coordinates on a map~\cite{Zender2008a}, or that can be specified using known landmarks~\cite{brenneretal07ijcai}.  However, many spatial terms are defined not only by their geometry, but also by their context.  That is, to understand this language requires considering the other objects and agents in the environment as well as their configuration, functional roles, and goals.  We call regions that can only be understood in this way \textit{context-dependent spatial regions} (CDSRs).

Consider the following regions shown in Figure \ref{fig:examples}: the front of a classroom, safety in a military engagement, and a geographic bottleneck.  To determine the extent of these regions, one must identify the physical boundaries (e.g., walls, bodies of water), the objects within the environment, their orientation and their functional use (e.g., desks oriented toward a whiteboard), and other agents' intentions and capabilities (e.g., enemy tanks can move and have a range of attack).%  Furthermore, these regions exist at different levels of spatial resolution.  Consider a neighborhood in a city, whose boundaries expand and contract over time with changes to the surrounding residents and businesses.

%Combine these figures into a single figure across
\begin{figure}
\centering
%% Change the classroom to be a picture stimuli
\begin{tabular}{c c c} 
  \begin{subfigure}[b]{0.32\textwidth}
  \includegraphics[width=\textwidth]{figures/classroom.png}
%  \caption{``The front of the classroom'' could be defined as the area between the desks and the whiteboard where people give presentations.}
  \caption{``The front of the classroom''.}
  \end{subfigure} &
\begin{subfigure}[b]{0.32\textwidth}
  \includegraphics[width=\textwidth]{figures/safety-3.jpg}
%  \caption{Safety for the scout is contextually defined by the location of the friendly turrets and the capabilities of the opposing force.}
  \caption{``The safe region for the scout''.}
\end{subfigure} & 
\begin{subfigure}[b]{0.32\textwidth}
  \includegraphics[width=\textwidth]{figures/bottleneck.JPG}
%  \caption{A bottleneck occurs when paths are confined to a narrow region by geographic features.  In this case, the water prevents ground movement.}
 \caption{``The bottleneck''.}
  \label{fig:safety}
  \end{subfigure}\\
\end{tabular}
  \caption{Examples of spatial regions used as stimuli in our experiments.}
  \label{fig:examples}
\end{figure}

Understanding CDSRs is crucial to joint intelligent action with humans; among other reasons they are frequently tieds to goals and commands: ``go to the front of the classroom'', ``defend the bottleneck'', and ``get to safety''.  We claim that (1) people can effectively communicate using these regions and (2) the size, location and shape of these regions change with the context.  To explore these claims, we present the results of a human study in which people marked regions and rated points representing CDSRs in three different environments in 13 different contexts.  In discussing the results, we consider the applicability of approaches from cognitive systems and robotics.  We posit that to account for this ability, a cognitive system must respond to changing context, apply learned regions to new contexts, and reuse learned knowledge in multiple tasks.  Current approaches have a number of shortcomings that must be addressed not only to improve our understanding of human spatial reasoning, but also to enable robust intelligent assistants in human environments.

%JK Removed the following text and replaced with text above primarily to save space, feel free to switch back.
%  Not only are these regions frequently tied to goals and commands: ``go to the front of the classroom'', ``defend the bottleneck'', and ``get to safety'', but they are also central to understanding human activities: e.g., ``Is there a class in session?''.
%The long-term goal of our work is to create cognitive systems that can naturally interact with people in human environments.  While the study presented below illustrates the importance of context, object types and configurations, on human spatial judgements, constructing a cognitive systems approach requires more than reproducing these results.  First, the representational approach should be transferable to new contexts.  While humans do not demonstrate mastery after a single example of a concept, they are able to begin using it immediately refining it over time.  Second, the representation should be applicable in different reasoning tasks.  The functional nature of these regions makes them important for many tasks from action planning to activity recognition.


\section{Human Studies}

We developed the study described below in order to examine the following questions about CDSRs:
\begin{enumerate}
	\item Given a particular context, do people agree on the location and extent of a named CDSR?
	\item Does changing the context in a given scenario predictably affect the CDSR?
\end{enumerate}

Both of these research questions address the fundamental claim that CDSRs are contextually defined. Question 1 addresses this claim by examining: (a) whether or not a CDSR is always applicable in a given environment (e.g., does a classroom always have a front?), and (b) the variation in CDSR placement and size within different contexts. Question 2 addresses the contextual nature of CDSRs in terms of whether or not the variation in the location and extent of a CDSR systematically follows the variations  in the given context. 

%In a context exhibiting clear contextual clues (e.g. determining a safe zone in a battle context where all information is known and battle formations are well-defined) one would expect high agreement in human identification of the CDSR. However, in the same environment (classroom, geographic location) but with less well-defined context, one might expect little to no agreement between human communicators.  

%Research question 1 addresses the contextual nature of CDSRs because if CDSRs are contextually dependent then one would expect a variation %across the agreement between subjects for different context: in a context that afforded the use of a particular CDSR one would expect to see a %high agreement between subject reponses, conversely in a context where a CDSR was not appropriate one might expect little or no agreement %between subjects. Research question 2 address the contextual nature of CDSRs. Research question 2 addresses the contextual nature of CDSRs %in terms of whether or not the variation in the location and extent of a CDSR systematically follow the variations  in the given context. 

We developed two experimental tasks that each address both research questions:
\begin{itemize}
	\item{Task 1:} Subjects were presented with a visual representation of a spatial scene plus a linguistic description containing a CDSR and were tasked with drawing a polygon on the visual stimulus that defined the CDSR (or stating there was no such region).
	\item{Task 2:} Subjects were presented with a visual representation of a spatial scene with a marked location and an accompanying spatial description and were asked to rate how well the location fit with their understanding of the CDSR on a five point \textit{Likert scale}.
\end{itemize}

We developed a set of visual stimuli depicting different contexts, and paired each with an accompanying linguistic description of a CDSR (see Figure \ref{fig:exp_stimuli}). These stimuli were used in both Task 1 and 2. Throughout the rest of the paper we will use the labels from Figure \ref{fig:exp_stimuli} as the names for the different stimuli. The stimuli can be broken into three groups based on the situation: (1) the classroom stimuli were always accompanied by the spatial description ``the front of the classroom''; (2) the military engagement stimuli were always accompanied by the description ``safety''; and (3) the geographical map stimuli were always accompanied by the description ``the bottleneck''.

The classroom stimuli consisted of pairs of floor plans and photos of classrooms with different chair configurations. For example, Figure \ref{fig:classroom-a} is a floor plan of a classroom where the chairs have been arranged in a circle and Figure \ref{fig:classroom-b} is the accompanying photo of this classroom. We developed four different classroom stimuli each with a different layout of desks: a circular layout, a U shaped layout (Figure \ref{fig:classroom-c} and Figure \ref{fig:classroom-d}), a typical row-based layout with the first row of chairs close to the expected front of the room (Figure \ref{fig:classroom-e}), and the same arrangement with the front row of chairs further away from the expected front of the room (Figure \ref{fig:classroom-f})\footnote{In the interest of saving space we do not show all the classroom photo stimuli.}. To control for left-right bias effects, two versions of the U shaped layout, close rows layout and far rows layout were created by horizontally flipping the images. Consequently, there were seven classroom stimuli (circle, U $\times$ 2, close $\times$ 2, far $\times$ 2).  The two geographical map stimuli, Figures \ref{fig:bottleneck-a} and \ref{fig:bottleneck-b}, differed in the extent of the water.  Finally, the four military engagement stimuli, Figures \ref{fig:tanks-a}, \ref{fig:tanks-b}, \ref{fig:turrets-a}, \ref{fig:turrets-b} varied the functionality of the objects in the scene (tanks that could move versus fixed turret emplacements), the location of the objects in the scenes (tanks close to the bottom of the image versus near the top), and the layout of the objects (turrets in a row versus turrets in a row with a lone turret in front). 

\begin{figure}
%\includegraphics[width=0.4\textwidth]{figures/turrets-plus-one.jpeg}
\centerline{
\begin{tabular}{ccc}
\begin{subfigure}[b]{0.33\textwidth}\includegraphics[width=\textwidth]{figures/classroom-circle.pdf}\caption{}\label{fig:classroom-a}\end{subfigure}&
\begin{subfigure}[b]{0.33\textwidth}\includegraphics[width=\textwidth]{figures/classroom-circle-photo.jpg}\caption{}\label{fig:classroom-b}\end{subfigure}&
\begin{subfigure}[b]{0.33\textwidth}\includegraphics[width=\textwidth]{figures/classroom-u-shaped.pdf}\caption{}\label{fig:classroom-c}\end{subfigure}\\
\begin{subfigure}[b]{0.33\textwidth}\includegraphics[width=\textwidth]{figures/classroom-u-photo.jpg}\caption{}\label{fig:classroom-d}\end{subfigure}&
\begin{subfigure}[b]{0.33\textwidth}\includegraphics[width=\textwidth]{figures/classroom-close.pdf}\caption{}\label{fig:classroom-f}\end{subfigure}&
\begin{subfigure}[b]{0.33\textwidth}\includegraphics[width=\textwidth]{figures/classroom-far-away.pdf}\caption{}\label{fig:classroom-e}\end{subfigure}\\
\begin{subfigure}[b]{0.33\textwidth}\includegraphics[width=\textwidth]{figures/bottleneck-1.jpg}\caption{}\label{fig:bottleneck-a}\end{subfigure}&
\begin{subfigure}[b]{0.33\textwidth}\includegraphics[width=\textwidth]{figures/bottleneck-2.jpg}\caption{}\label{fig:bottleneck-b}\end{subfigure}&
\begin{subfigure}[b]{0.33\textwidth}\includegraphics[width=\textwidth]{figures/tanks-close.jpg}\caption{}\label{fig:tanks-a}\end{subfigure}\\
\begin{subfigure}[b]{0.33\textwidth}\includegraphics[width=\textwidth]{figures/tanks-far.jpg}\caption{}\label{fig:tanks-b}\end{subfigure}&
\begin{subfigure}[b]{0.33\textwidth}\includegraphics[width=\textwidth]{figures/turrets.jpeg}\caption{}\label{fig:turrets-a}\end{subfigure}&
\begin{subfigure}[b]{0.33\textwidth}\includegraphics[width=\textwidth]{figures/turrets-plus-one.jpeg}\caption{}\label{fig:turrets-b}\end{subfigure}
\end{tabular}
}
\caption{The stimuli used in the experiment.}
\label{fig:exp_stimuli}
\end{figure}

%\subsection{Participants and Procedure}
%Using these stimuli pairs, we developed an online experiment with two different types of trials to explore our initial questions. 43 undergraduate students from two different third-level institutes took part in the study. To control for sequence effects the presentation order of the trials were randomised for each participant as was the order of the stimulus presented within each trial.   

\textbf{Participants and Procedure:} The experiment was delivered online. 43 undergraduate students from two third-level institutes took part in the study. To control for sequence effects the order of the trials were randomised for each participant as was the order of the stimuli within each trial.   

\subsection{Task 1 - Marking Region}
In task one, the subject was asked to draw a polygon that defined the extent of the CDSR given or to select a no region option.  The drawing was done on the diagram in all stimuli with the classroom photographs provided for reference in those situations.  Subjects drew polygons by creating a sequence of vertices with lines being automatically generated between them as the subject drew.  There was no upper bound placed on the number of line segments drawn, but subjects were required to provide at least three line segments or to choose no region before continuing.  We used two methods to analyze the results inter-annotator agreement and comparing features of CDSRs between stimuli. 


Second, we use features of the regions (e.g., area, maximum x-coordinate) to assess changes between trials.  That is, as the desks are moved back from stimuli (e) to (f), we expect the maximum x-coordinate of the region to increase showing that region extent varies systematically with context changes.

\subsubsection{Inter-Annotator Agreement}

%Kate:  (1) if there is fair to good overall agreement, this is evidence that people change the regions they produce based systematically with %context,  I changed this SUBSTANTIALLY to something that I think sounds better, but makes a different argument.  Let me know if it is ok.  Really %agreement would indicated shared understanding while variation between contexts would lend support to the importance of context.  Not sure I %captured this
We use two methods to analyze the results of the region drawing task.  First, we use inter-annotator agreement for two reasons: (1) fair to good overall agreement would provide evidence that people have a shared interpretation of CDSRs in the given contexts, (2) we expect less agreement in ambiguous contexts (e.g., the circle (a) and U-shaped (c) desk arrangements as well as the turret + 1 (l) military engagement configuration).

To measure the agreement between two regions, we use \textit{Cohen's kappa} coefficient, $\kappa$, as follows.  We generated a set of 47,000 test points that were uniformly distributed across each stimulus and used each subject's polygon to classify the test points as being inside or outside the polygon. After classifying the test points for each subject, we then computed an average pairwise kappa score for each subject with all the other subjects on a per stimulus basis. In this calculation, if either of the subjects in a pair entered a no region response for a particular stimulus a $\kappa$ value of 0 was used.  $\kappa$ ranges from 0 indicating no agreement to 1 indicating complete agreement.  Typically, scores above 0.4 are considered fair agreement and 0.75 considered excellent.

% Klenk: This is where I left off.  I think we can drop the histogram and talk.

%However, this calculation was done across all the stimuli and as we have already mentioned some of the stimuli were deliberately designed to be non-prototypical, in particular we expected the classroom stimuli where the chairs were arranged in a circle (Figure \ref{fig:classroom-a}) and in a U shaped pattern (Figure \ref{fig:classroom-c} and its horizontally flipped equivalent) to be relatively novel contexts for the subjects. Indeed, if we look at the histogram of the average pairwise $\kappa$ scores by stimulus, Figure \ref{fig:stim-kappa}, it is evident that there is a large variation, ranging from $0$ to $0.8$, in $\kappa$ scores for each stimulus. 
\renewcommand{\thefootnote}{\fnsymbol{footnote}}
%\newcommand\footnoteref[1]{\protected@xdef\@thefnmark{\ref{#1}}\@footnotemark}

%\begin{table}[t]
%\parbox{.50\linewidth}{
%\centering 
%\begin{tabular}{l c c}
%% Stim ID & Average Pairwise $\kappa$ & No Region Responses\\
% & Average & No \\
%Stim ID & Pairwise $\kappa$ & Reg.\\
%\hline
%\hline
%(h) & 0.5034 & 0\\ % 1_2
%(g) & 0.4585 & 1\\ % 1_1
%(k) & 0.4548 & 0\\ % 3_3
%(f) & 0.3930 & 0\\ % 2_4
%(i) & 0.3906 & 0\\ % 3_1
%(e-flipped) & 0.3813 & 2\\ % 2_3
%(j) & 0.3787 & 1\\ % 3_2
%(f-flipped) & 0.3720 & 2\\ % 2_5
%(e) & 0.3629 & 1\\ %2_2
%(l) & 0.3164& 0\\ % 3_4
%(c) & 0.2438 & 2\\ % 2_6
%(c-flipped) & 0.2414 & 4\\ %2_7
%(a) & 0.0996 & 10 \\ % 2_1
%\hline
%\end{tabular}
%\caption{The average pairwise $\kappa$ scores and the number of no region response by visual stimulus.}
%\label{tab:stimuli-kappa-scores}
%}
%\hfill
%\parbox{.50\linewidth}{
%\centering
%\begin{tabular}{c c}
%Expected Relationship & Human Results  \\
%\hline
%\hline
%area(g) > area(h) & 291.2 > 276.2\footnotemark[2] \\
%maxX(f) > maxX(e) & 176.1 > 167.1 \\
%% minimumX(e-flipped) > minimumX(f-flipped) & 336.1 > 321.3 (p<.12) \\
%minX(e-flipped)  &  \\
% > minX(f-flipped) & 336.1 > 321.3\footnotemark[2] \\
%maxY(i) > maxY(j) & 304.64 > 265.04\footnotemark[2] \\
%maxY(j) > maxY(k) & 265.04 > 251.71 \\
%maxY(k) = maxY(l) & 265.04 != 290.03 \\
%\hline
%\end{tabular}
%\caption{Comparison of the expected relationships to the mean human responses.}
%\label{tab:region-location-tests}
%}
%\end{table}
%\footnotetext[2]{Significance value from t-test (p<0.1)}


\begin{table}[htb]
\begin{center}
\begin{tabular}{l c c c}
% Stim ID & Average Pairwise $\kappa$ & No Region Responses\\
Stim ID & Ambiguity Expected? & Average Pairwise $\kappa$ & No Reg.\\
\hline
\hline
(h) & N & 0.5034 & 0\\ % 1_2
(g) & N & 0.4585 & 1\\ % 1_1
(k) & N & 0.4548 & 0\\ % 3_3
(f) & N & 0.3930 & 0\\ % 2_4
(i) & N & 0.3906 & 0\\ % 3_1
(e-flipped) & N & 0.3813 & 2\\ % 2_3
(j) & N & 0.3787 & 1\\ % 3_2
(f-flipped) & N & 0.3720 & 2\\ % 2_5
(e) & N & 0.3629 & 1\\ %2_2
(l) & Y & 0.3164& 0\\ % 3_4
(c) & Y & 0.2438 & 2\\ % 2_6
(c-flipped) & Y & 0.2414 & 4\\ %2_7
(a) & Y & 0.0996 & 10 \\ % 2_1
\hline
\end{tabular}
\end{center}
\caption{The average pairwise $\kappa$ scores and the number of no region response by visual stimulus.}
\label{tab:stimuli-kappa-scores}
\end{table}





%\begin{table}
%\centerline{
%\begin{scriptsize}
%\begin{tabular}{l c c c c c c c c c c c c c}
%% (h) - 1_2; (g) - 1_1; (k) - 3_3; (f) - 2_4; (i) - 3_1; (e-flipped) - 2_3; (j) - 3_2; (f-flipped) - 2_5; (e) - 2_2;  (l) - 3_4;  (c) -  2_6;  (c-flipped) - 2_7;  (a) - 2_1
%Stim ID & (h) & (g) & (k) & (f) & (i) & (e-flipped) & (j) & (f-flipped) & (e) & (l) & (c) & (c-flipped) & (a)\\ 
%Average Pairwise $\kappa$ & 0.5034 & 0.4585 & 0.4548 & 0.3930 & 0.3906 & 0.3813 & 0.3787 & 0.3720 & 0.3629 & 0.3164 & 0.2438 & 0.2414 & 0.0996 \\ 
%No Region Responses & 0 & 1 & 0& 0& 0& 2&1&2& 1& 0&2&4& 10\\
%\hline
%\end{tabular}
%\end{scriptsize}
%}
%\label{tab:stimuli-kappa-scores}
%\caption{The average pairwise $\kappa$ scores, ordered in descending order from left to right, and the number of no region response by visual stimulus. The stimuli id are based on the labels in Figure \ref{exp-stimuli} with the horizontally flipped versions of stimuli indicated.}
%\end{table}

Table \ref{tab:stimuli-kappa-scores} lists the average pairwise $\kappa$ score for each of the visual stimulus used in the experiment and the number of no region responses for each stimulus.  It is clear from this table that the less common stimuli have the lowest $\kappa$ scores.  Each of the non-typical regions, the turret +1 arrangement (l), the u-shaped layouts (c and c-flipped), and the circle layout have kappa scores of less than 0.35. Looking at the classroom results as a whole supports the hypothesis that the definition of CDSRs in a given context is affected by the location and orientation of objects in the environment. An extreme example of this relationship between the configuration of objects in a context and the meaning of a CDSR is that our results indicate that the shared understanding of ``the front of the room'' breaks down in the classroom setting when the chairs in the room are moved into a circle configuration as indicated by the fact that context (a) (Figure \ref{fig:classroom-a}) had an extremly low average pairwise agreement of  $0.0996$).

%Given this relatively large variation across stimuli, which we attribute to the typicality of the depicted context, we computed the average pairwise $\kappa$ scores for each subject when the non-prototypical stimuli trials were excluded. 
%Figure \ref{fig:subj-kappa-proto} presents the histogram of the resulting average pairwise subject  $\kappa$ scores by subject. Not surpringly, with the exclusion on the non-prototypical stimuli the average pairwise $\kappa$ scores increase, with over half the subjects having an average pairwise $\kappa$ score between $0.4$ and $0.6$. $\kappa$ scores in this range can be interpreted as fair to good.

%Kate: this is a place (above) that I think we could combine/cut.  I think we can talk about the non-typical classrooms more concisely.  Let's
% say something like "some of the layouts were non-typical -> less experience with the context -> we expect lower agreement -> this is what 
% we see" then we can say that if we remove these examples, agreement on the more typical scence is good.  So suggestions are
% (1) condense discussion (2) make it clear exactly what we expect for the non typical scenes and WHY this is tied back to our claim
% about the role of context in CDSRs.  I am happy to make these changes if John agrees that I am not horribly misinterpreting his fine results 

%The analysis of the task 1 trials indicates that, over a range of domains and spatial regions, the responses from the subjects did agree, to a good degree. Furthermore, their results support the hypothesis that the defintion of CDSRs in a given context is effected by the location and orientation of objects in the environment. For example, focusing on the classroom scenes (stimuli, $a$, $c$, $e$, and $f$) the only difference between these trials was the layout of the chairs in the room and yet there is a substantial variation across these stimuli of the agreement between subjects on the specified spatial region, namely ``the front of the room''. Situation $f$ resulted in an average pairwise agreement of $0.393$ and in comparision context $a$ had an average pairwise agreement of $0.0996$. 


%\begin{figure}
%%\includegraphics[width=0.4\textwidth]{figures/turrets-plus-one.jpeg}
%\centerline{
%\begin{tabular}{ccc}
%\begin{subfigure}[b]{0.3\textwidth}\includegraphics[width=\textwidth]{figures/subj-kappa-hist.pdf}\caption{}\label{fig:subj-kappa}\end{subfigure}&
%\begin{subfigure}[b]{0.3\textwidth}\includegraphics[width=\textwidth]{figures/stimuli-kappa-hist.pdf}\caption{}\label{fig:stim-kappa}\end{subfigure}&
%\begin{subfigure}[b]{0.3\textwidth}\includegraphics[width=\textwidth]{figures/subj-kappa-hist-protostim.pdf}\caption{}\label{fig:subj-kappa-proto}\end{subfigure}
%\end{tabular}
%}
%\caption{(a) Histogram of average pairwise subject kappa score across all stimuli by subject, (b) histogram of average pairwise subject kappa scores for each stimulus, (c) histogram of average pairwise subject kappa score excluding non-prototypical stimuli by subject.}
%\label{fig:type-1-hists}
%\end{figure}

\subsubsection{Predicted Region Changes and Simulation}
\label{sec:simulation}
To assess the claim that changing contexts predictably affects the CDSRs, we identified six expected relationships of region features between stimuli.  For example, we expect the wider bottleneck (g) to result in a larger area than (h), and, in the classroom, the maximum-x coordinate of the region should be greater in the case where the desks are further away from the whiteboard (f) than (e).  CDSRs are dynamic, in the move to safety scenario, we expect that people will simulate the actions of the agents.  Therefore, we expect the maximum-Y to be larger in the tanks scenario (j) than the turrets (k) because the friendly tanks will pursue the enemy tanks.  Similarly, we expect maximum-Y coordinate to the be same in the two different turrets scenarios (k) and (l), because people would simulate the results of the battle and realize that a single turret will not provide safety against three tanks.  

\begin{table}[htb]
\begin{center}
\begin{tabular}{r c l r c l}
\multicolumn{3}{c}{Expected Relationship} & \multicolumn{3}{c}{Human Results}  \\
\hline
\hline
area(g) &>& area(h) & 291.2 & >& 276.2* \\
maxX(f) &>& maxX(e) & 176.1& >& 167.1 \\
% minimumX(e-flipped) &>& minimumX(f-flipped) & 336.1& >& 321.3 (p<.12) \\
minX(e-flipped) &>& minX(f-flipped) & 336.1& >& 321.3* \\
maxY(i) &>& maxY(j) & 304.64 &>& 265.04* \\
maxY(j) &>& maxY(k) & 265.04 &>& 251.71 \\
maxY(k) &=& maxY(l) & 265.04 &!=& 290.03 \\
\hline
\end{tabular}
\end{center}
\caption{Comparison of the expected relationships to the mean human responses. *Significance value from t-test (p<0.1).}
\label{tab:region-location-tests}
\end{table}


Table \ref{tab:region-location-tests} compares the expected relationships between the CDSR features between stimuli.  In five of the six comparisons, the expected relationships were found, supporting our claim that changes of context systematically change the perceived extent of a CDSR.  The last two rows test our hypothesis that people are simulating the actions of others in defining the extent of the CDSR.  The comparison between the maximum-y coordinate of the safety area with tanks (j) versus turrets (k) provides some evidence, although not statistically significant, that people are simulating actions of other agents in the scene.  In the other simulation case, we did not observe the expected relationship between the row of turrets (k) and turrets +1(l).  The average maximum-Y was greater for the turret +1 stimuli indicating that people felt that the single turret provided protection from the enemy tanks.  As discussed in the previous section, we expected some ambiguity in this trial as people likely have different models for the capabilities of the different units in this scenario.


%Kate: I took out "Our expectation was that people would reason that a lone turret would be no match for three tanks and indicate the safe region. " because it didn't make sense to me.

\subsection{Task 2 - Rating Points} 
In task two, the subject was asked to rank three points on a five point \textit{Likert scale} how well the location matched the given CDSR, or select not applicable.  The location of these points was informed by the results of a pilot study, such that one of the domain's points, called the \textit{sweetspot}, was located near the expected prototypical center of the CDSR. In the classroom domain, the three points were located along the horizontal line that bisected the stimulus with P1 positioned on the left of the stimulus, P2 in the center and P3 on the right of the stimulus.  This provides comparable data points between the trials where the classroom stimuli were horizontally flipped.  In the geography situations, P1 was positioned in the top left corner of the stimulus, P2 was the expected \textit{sweetspot} and was located within the expected bottleneck region for both stimulus and P3 was located in the bottom right corner of the stimulus.  In the military engagement domain the three test points were located on the vertical line that bisected the stimulus: P1 was located near the enemy tanks, P3 was located with all friendly units between it and the enemy tanks and P2 was located near the center of the image.  The locations are shown in Figure \ref{fig:locations}.
%Kate: to save some space, maybe we could talk generally about location placement and only give specifics for one set of stimuli?  Just a though

\begin{figure}
\centerline{
\begin{tabular}{ccc}
\begin{subfigure}[b]{0.3\textwidth}\includegraphics[width=\textwidth]{figures/classroom-circle-points.png}\caption{}\label{fig:classroom-points}\end{subfigure}&
\begin{subfigure}[b]{0.3\textwidth}\includegraphics[width=\textwidth]{figures/bottleneck-2-points.png}\caption{}\label{fig:bottleneck-points}\end{subfigure}&
\begin{subfigure}[b]{0.3\textwidth}\includegraphics[width=\textwidth]{figures/tanks-close-points.png}\caption{}\label{fig:tanks-points}\end{subfigure}\\
\end{tabular}
}
\caption{The pre-selected locations for each domain used in the Type 2 Trials.}
\label{fig:locations}
\end{figure}


\subsubsection{Sweetspots}

%Task 1: The participant draws a polygon denoting the region
%Analysis of results:
%(1) Do participants agree/understand-the-task? (validation of approach) + Should anyone be excluded?
%Within the data for each trial,
%-- For each participant, generate the intersection and union polygon of all the other participants
%---- Generate N random points in the area of the full stimulus
%---- For each of these N points check whether the participant's polygon and the [intersection/union] polygon both include it (agree++) or if one includes it and the other doesn't (disagree++)
%---- Then we calculate a \textit{Cohen kappa} score as a measure of agreement between the participant and eveyone else. Note we need to calculate the chance of random agreement but we can do this as a function over the relative size of the participant's polygon in the total area of the stimulus and the relative size of the intersection/union polygon
%(see http://en.wikipedia.org/wiki/Cohen%27s_kappa )
%This process will result in for each trial a list of kappa scores for each participant indicating the extent to which they agree with everyone else. We are hoping for a high kappa score
%(2) Does context matter?
%I understand that all the rooms stimuli have the same extent.
%So one way of evaluating whether context matters is to use the kapp score between the intersection/union polygon generated using all the response (less the excluded partitipants) for one trial and the equivalent polygon for another trial.
%We are hoping for a low kappa score

%\textit{Cohen's kappa} measures the agreement between two annotators who each classify a number of items into two or more mutually exclusive classes as is defined as follows:
%\begin{equation*}
%\kappa = \frac{P(a)-P(ca)}{1-P(ca)}
%\end{equation*}
%where $P(a)$ is the observed relative agreement between the annotators, and $P(ca)$ is the probability of a chance agreement between these to annotators. 

%In the type 1 trials subjects drew a polygon on the visual stimulus that defined the location and extent of the relevant spatial region. In analysising the results of these trials we wished to investigate how well the subject's responses agreed with each other and, also, whether the location and orientation of objects in the scene effected the definition of spatial regions in the scene. 




%The type 2 trials the same visual stimulus and accompanying spatial descriptions as the type 1 trials with the addition that a location was marked on %the visual stimulus (using a red dot) and the subject was asked to rank on a five point \textit{Likert scale} their judgement of how well the location matched %the described spatial region. For each stimulus domain (classroom, map, military engagement) three points (P1, P2, and P3) were pre-selected.

Our hypothesis was that the expected \textit{sweetspots} would score highest among the three points in all situations, and this would demonstrate broad agreement between participants concerning the locations of the CDSRs.  Table \ref{tab:point-likert-scores} lists the average Likert scores for the different points in each context. In all contexts, the expected \textit{sweetspot} received the maximum average Likert rating that was statistically significant using one-tailed two-sample t-tests $(p<.05)$.  To control for left--right bias, we compared the ratings of \textit{sweetspots} in the classroom stimuli (c), (e), and (f) with their flipped variants.  There was no statistical difference between the stimuli and their flipped variants.  These Likert ratings on individual stimuli support the claim that people are able to agree on the locations of CDSRs with changing context.  

%There were two of these t-tests carried out for each stimulus, each one tested whether the \textit{sweetspot} ranking was statistically higher than one of the other points. Table \ref{tab:point-likert-scores} also lists the results of these t-tests for each stimulus. In all cases, the results indicate that the null hypothesis, $H_0$, should be rejected and that the alternative hypothesis, namely that the \textit{sweetspot} was rated significantly higher than the other point, could be accepted with $\alpha=0.05$.

%As noted above, we controlled for left-right bias by flipping some of the stimuli on the horizontal axis. These flipped stimuli included the classroom stimuli (c), (e) and (f). This flipping of stimuli gave us the opportunity to compare the results of the \textit{sweetspots} rankings between this pairs of stimuli trials. Given the horizontal flipping of the stimuli it was expected that the \textit{sweetspot} for the stimuli should also be flipped between P1 and P3. To test whether or not the expected shift in the \textit{sweetspots} occurred we ran two-tailed two-sample t-tests between the \textit{sweetspots} of the flipped and non-flipped stimuli pairs. In all cases the results indicated that there was no statistical difference between the rating of the \textit{sweetspot} for the non-flipped stimuli and the rating of the \textit{sweetspot} for the flipped stimuli using a confidence level of $95\%$. In other words, the \textit{sweetspot} flipped from P1 to P3 systematically with the horizontal flipping of the stimuli. %lending support for question 2 ...

% The following paragraph should end by saying this about question 1.


\begin{table}
\centerline{
\begin{tabular}{l c c c c c c c }
\hline
\hline
%Domain & \multicolumn{7}{l}{Map/Bottleneck} \\
\textbf{Map} & g & h & ~ & ~ & ~ & ~ & ~ \\ 
P1 & $1.1707$ & $1.1707$ & ~ & ~ & ~ & ~ & ~ \\
P2 & $\mathbf{4.6667}$ & $\mathbf{4.7561}$ & ~ & ~ & ~ & ~ & ~ \\
P3 & $1.85$ & $1.585$ & ~ & ~ & ~ & ~ & ~ \\
\hline
%Domain & \multicolumn{7}{l}{Classrom/Front of Room} \\
\textbf{Classroom}& a & c & c-flipped & e & e-flipped & f & f-flipped \\
P1 & $\mathbf{3.5122}$ & $\mathbf{4.7073}$ & $1.244$  & $\mathbf{4.7561}$ & $1.2439$ & $\mathbf{4.8292}$ & $1.1463$\\
P2 & $2.6341$ & $2.9024$ & $2.9268$ & $2.0000$ & $1.8500$ & $2.0976$ & $2.3171$\\
P3 & $1.9024$ & $1.3571$ & $\mathbf{4.6429}$ & $1.2439$ & $\mathbf{4.7073}$ & $1.2750$ & $\mathbf{4.6585}$\\
\hline
%Domain & \multicolumn{7}{l}{Military Engagement/Safe Region} \\
\textbf{Military} & i & j & k & l & ~ & ~ & ~ \\
P1 & $1.1951$ & $1.0952$ & $1.0714$ & $1.2327$ & ~ & ~ & ~ \\
P2 & $4.3902$ & $2.2927$ & $2.7804$ & $3.1667$& ~ & ~ & ~ \\
P3 & $\mathbf{4.9024}$ & $\mathbf{4.5122}$ & $\mathbf{4.6585}$ & $\mathbf{4.7073}$ & ~ & ~ & ~ \\
\hline
\end{tabular}
}
\caption{Average Likert ratings per point by situation and stimulus.}
\label{tab:point-likert-scores}
\end{table}


\subsubsection{Between Stimuli Comparisons}

To assess the claim that changing contexts predictably affects the CDSRs, we can compare the ratings of points between stimuli.  In the classroom domain, we assume that part of the definition of the front of the room is that where one would stand when teaching a class, therefore, we expected the rating for the center point, P2, to be higher in the u-shaped scenario (c) than the rows scenarios (e) and (f).  This is confirmed in the results from Table \ref{tab:point-likert-scores}.  In the military domain,  In stimulus (i) the friendly tanks are positioned between the P2 point and the enemy tanks whereas in stimulus (j) the P2 point is in \textit{no-man's land}\footnote{We use the term no-man's land to describe the unoccupied and contested area between the opposing forces.}.  Given this change in context one would expect a higher ranking for the P2 point in context (i) than in context (j). This is confirmed in Table \ref{tab:point-likert-scores} and is statistically significant $(p<.05)$.  To assess the extent of participants use of simulation in determining the context, we compare the scores the \textit{no-man's land} point between the tanks (j) to the turrets (k) stimuli and the turrent (k) and turrets +1 (l).  The mobility of the friendly tanks should increase the size of the safe area over time resulting in higher ratings in stimuli (j) than (k).  The single turret should offer only temporary protection against the energy tank force, therefore it should not increase the safety rating.  Neither of these results were found in Table \ref{tab:point-likert-scores}.  As discussed in Section \ref{sec:simulation}, we expected ambiguity with the turrets +1 stimuli.  The difference in the rating of P2 between the tanks and turrets scenario provides two alternative explanations regarding how participants interpreted the context:  (1) the turrets were viewed as superior defensive units and, thus, better able to provide safety, or (2) the turrets being fixed could not move away, and thus, establish a more permanent CDSR for safety.  Further experiments are needed to explore these hypotheses. 



%\begin{table}
%\centerline{
%\begin{tabular}{l c c c }
%\hline
%\hline
%Domain & \multicolumn{3}{c}{Map/Bottleneck} \\
%% ~ & \multicolumn{3}{c}{Average Likert Rating} \\
%Stim ID & P1 & P2 & P3 \\
%(g) & $1.1707$ & $\mathbf{4.6667}$ & $1.85$\\ % 1_1
%(h) & $1.1707$ & $\mathbf{4.7561}$ & $1.585$ \\ % 1_2
%\\
%\hline
%Domain & \multicolumn{3}{c}{Classrom/Front of Room} \\
%% ~ & \multicolumn{3}{c}{Average Likert Rating} \\
%Stim ID & P1 & P2 & P3 \\
%(a) & $\mathbf{3.5122}$ & $2.6341$ & $1.9024$ \\ % 2_1
%(c) & $\mathbf{4.7073}$ & $2.9024$ & $1.3571$ \\ % 2_6
%(c-flipped) & $1.244$ & $2.9268$ & $\mathbf{4.6429}$ \\ %2_7
%(e) & $\mathbf{4.7561}$ & $2.0000$ & $1.2439$ \\ %2_2
%(e-flipped) & $1.2439$ & $1.8500$ & $\mathbf{4.7073}$ \\ % 2_3
%(f) & $\mathbf{4.8292}$ & $2.0976$ & $1.2750$ \\ % 2_4
%(f-flipped) & $1.1463$ & $2.3171$ & $\mathbf{4.6585}$ \\ % 2_5
%\\
%\hline
%Domain & \multicolumn{3}{c}{Military Engagement/Safe Region} \\
%% ~ & \multicolumn{3}{c}{Average Likert Rating} \\
%Stim ID & P1 & P2 & P3\\
%(i) & $1.1951$ & $4.3902$ & $\mathbf{4.9024}$ \\ % 3_1
%(j) & $1.0952$ & $2.2927$ & $\mathbf{4.5122}$ \\ % 3_2
%(k) & $1.0714$ & $2.7804$ & $\mathbf{4.6585}$ \\ % 3_3
%(l) & $1.2327$ & $3.1667$ & $\mathbf{4.7073}$\\ % 3_4
%\\
%\hline
%\end{tabular}
%}
%\label{tab:point-likert-scores}
%\caption{Average Likert Ratings per Point by Domain and stimulus.}
%\end{table}

%\begin{table}
%\centerline{
%\begin{tabular}{l c c c l}
%\hline
%Domain & \multicolumn{4}{c}{Map/Bottleneck} \\
% ~ & \multicolumn{3}{c}{Average Likert Rating} & Sweetspot Rated Significantly Higher? \\
%Stim ID & P1 & P2 & P3 & One-Tailed Two Sample t-test with $\alpha=0.05$, $H_0:\mu(P_i)=\mu(P_j)$\\
%(g) & $1.1707$ & $\mathbf{4.6667}$ & $1.85$ & $H_A:\mu(P2)>\mu(P1),~p<2.2^{-16};~H_A:\mu(P2)>\mu(P3),~p<2.2^{-16}$\\ % 1_1
%(h) & $1.1707$ & $\mathbf{4.7561}$ & $1.585$ & $H_A:\mu(P2)>\mu(P1),~p<2.2^{-16};~H_A:\mu(P2)>\mu(P3),~p<2.2^{-16}$\\ % 1_2
%\\
%\hline
%Domain & \multicolumn{4}{c}{Classrom/Front of Room} \\
% ~ & \multicolumn{3}{c}{Average Likert Rating} & Sweetspot Rated Significantly Higher? \\
%Stim ID & P1 & P2 & P3 & One-Tailed Two Sample t-test with $\alpha=0.05$, $H_0:\mu(P_i)=\mu(P_j)$\\
%(a) & $\mathbf{3.5122}$ & $2.6341$ & $1.9024$ & $H_A:\mu(P1)>\mu(P2),~p=6.497^{-4};~H_A:\mu(P1)>\mu(P3),~p=4.073^{-8}$\\ % 2_1
%(c) & $\mathbf{4.7073}$ & $2.9024$ & $1.3571$ & $H_A:\mu(P1)>\mu(P2),~p=6.396^{-16};~H_A:\mu(P1)>\mu(P3),~p<2.2^{-16}$\\ % 2_6
%(c-flipped) & $1.244$ & $2.9268$ & $\mathbf{4.6429}$ & $H_A:\mu(P3)>\mu(P1),~p<2.2^{-16};~H_A=\mu(P3)>\mu(P2),~p:1.864^{-12}$\\ %2_7
%(e) & $\mathbf{4.7561}$ & $2.0000$ & $1.2439$ & $H_A=\mu(P1)>\mu(P2),~p<2.2^{-16};~H_A=\mu(P1)>\mu(P3):~p<2.2^{-16}$\\ %2_2
%(e-flipped) & $1.2439$ & $1.8500$ & $\mathbf{4.7073}$ & $H_A=\mu(P3)>\mu(P1),~p<2.2^{-16};~H_A=\mu(P3)>\mu(P2):~p<2.2^{-16}$\\ % 2_3
%(f) & $\mathbf{4.8292}$ & $2.0976$ & $1.2750$ & $H_A=\mu(P1)>\mu(P2),~p<2.2^{-16};~H_A=\mu(P1)>\mu(P3),~p<2.2^{-16}$\\ % 2_4
%(f-flipped) & $1.1463$ & $2.3171$ & $\mathbf{4.6585}$ & $H_A=\mu(P3)>\mu(P1),~p<2.2^{-16};~H_A=\mu(P3)>\mu(P2),~p<2.2^{-16}$\\ % 2_5
%\\
%\hline
%Domain & \multicolumn{4}{c}{Military Engagement/Safe Region} \\
% ~ & \multicolumn{3}{c}{Average Likert Rating} &  Sweetspot Rated Significantly Higher?\\
%Stim ID & P1 & P2 & P3 & One-Tailed Two Sample t-test with $\alpha=0.05$, $H_0:\mu(P_i)=\mu(P_j)$\\
%(i) & $1.1951$ & $4.3902$ & $\mathbf{4.9024}$ & $H_A=\mu(P3)>\mu(P1),~p<2.2^{-16};~H_A=\mu(P3)>\mu(P2),~p=5.382^{-4}$\\ % 3_1
%(j) & $1.0952$ & $2.2927$ & $\mathbf{4.5122}$ & $H_A=\mu(P3)>\mu(P1),~p<2.2^{-16};~H_A=\mu(P3)>\mu(P2),~p<7.981^{-14}$\\ % 3_2
%(k) & $1.0714$ & $2.7804$ & $\mathbf{4.6585}$ & $H_A=\mu(P3)>\mu(P1),~p<2.2^{-16};~H_A=\mu(P3)>\mu(P2),~p=2.708^{-13}$\\ % 3_3
%(l) & $1.2327$ & $3.1667$ & $\mathbf{4.7073}$ & $H_A=\mu(P3)>\mu(P1),~p<2.2^{-16};~H_A=\mu(P3)>\mu(P2),~p=6.621^{-12}$\\ % 3_4
%\\
%\hline
%\hline
%\end{tabular}
%}
%\label{tab:point-likert-scores}
%\caption{Average Likert Ratings per Point by Domain with t-test results: in all cases the average rating of the sweetspot is significantly higher than the other points tested for that stimulus}
%\end{table}


%\begin{table}
%\centerline{
%\begin{tabular}{c c c c }
%\hline
%Domain & \multicolumn{3}{c}{Classrom/Front of Room} \\
%$H_0$ & $H_A$ & $\alpha$ & p-value\\
%$\mu(c.P1)=\mu(c-flipped.P3)$ & $\mu(c.P1)!=\mu(c-flipped.P3)$ & $0.05$ & $0.6315$\\
%$\mu(e.P1)=\mu(e-flipped.P3)$ & $\mu(e.P1)!=\mu(e-flipped.P3)$ & $0.05$ & $0.7358$\\
%$\mu(f.P1)=\mu(f-flipped.P3)$ & $\mu(f.P1)!=\mu(f-flipped.P3)$ & $0.05$ & $0.2592$\\
%\hline
%\hline
%\end{tabular}
%}
%\label{tab:flipped-point-likert-scores}
%\caption{Two Tailed Two Sample t-tests for flipped stimuli pair trials.}
%\end{table}





%-----------------
%
%Task 2: The participant selected a point denoting the \textit{sweetspot} of the region
%Analysis of results:
%(1) Do participants agree/understand-the-task? (validation of approach) + Should anyone be excluded?
%Compare the variance across the subjects with the variance across a number of set of randomly generate points, where set size equals the number of participants, using the F distribution (http://www.ltcconline.net/greenl/courses/201/regression/comparingVariances.htm) where we are checking if the variance of the random points are greater than that of the participants.
%We would hope that the variance of the subjects responses is lower than that of randomly generated points.
%After the initial comparison with the random point sets we may exclude participants based on extreme variance.
% 
%(2) Does context matter?
%Do the clusters of participants responses move?
%- for each experimental stimulus compute the mean x and y location for the responses for that trial
%- check whether the variance between the means of different stimuli responses is greater than the variance within each stimulus responses.

%-----------------
%
%
%Task 3: The participant grades a point with respect to membership of the region on a \textit{Likert scale}
%I understand this experiment to be useful because (a) it asks the participants to evaluate (rather than generate cf task 1 and 2) (b) hopefully we can use it to reinforce our findings in task 1 and task 2
%I think we need to be careful that the points we present to people for evaluation are reasonably well distributed so that we have a good chance of getting a variation across the responses
%Assuming that there is a variation across the Likert scores for a particular stimulus then check if their is a correlation between the scores and:
%(a) inclusion in the intersection/union polygons from task 1 for the same stimulus
%(b) distance from the mean sweet spot for that stimulus from task 2
%
%-----------------
 
\subsection{discussion}
Taken together, these results support the following conclusions. First, people are able to agree on location and area of CDSRs.  This agreement is reduced in situations where the domain knowledge of the context is ambiguous, such as, the circular desk arrangement or the defensive capabilities of a single turret.  Second, changing the context predictably changes the subjects assessments of location, shape and size of the CDSR.  These results were supported by tasks involving people labeling regions as well as rating individual points.  On the other hand, our experiments produced mixed results concerning our hypothesis that people use simulation to evaluate dynamic changes to CDSRs.  Further studies are required to control for the domain knowledge of the participants exploring this question.

\section{Applicability of Previous Work to CDSRs}

Whilst spatial reasoning has been an active topic in AI and robotics for decades, the properties we have demonstrated for CDSRs (i.e. that they vary systematically with the function and configuration of objects, plus the agents interacting with them) are not addressed by previous work, and thus it is not possible to take existing work and build a cognitive system capable of reasoning about, or communicating with, CDSRs. The remainder of this section reviews related previous work with reference to the requirements of modelling CDSRs.
 
Regions are central to qualiative spatial reasoning (QSR)~\cite{Cohn:2001}. Regions can either have crisp boundaries, usually defined by geometric shapes (e.g. the bounding boxes of visually tracked objects~\cite{SridharCohn:10}), or vague boundaries created qualitatively by multiple crisp regions (e.g. the `egg-yolk model'~\cite{Cohn96b}) or by thresholding an underlying spatially-mapped function (e.g. those created by potential fields~\cite{brenneretal07ijcai} or by Gaussian kernels~\cite{burbridge-dearden12}). The fundamental decision when creating a computational model of a CDSR is the method used to represent its boundary. Given their context-dependent nature, boundary definitions based purely on single geometric shapes or a fixed number of points seem unlikely to model the regions with sufficient flexibility.

However, arbitrary boundaries can be defined using a sequence of geometric primitives (usually straight lines), as in chain codes for image region representations~\cite{Freeman:1961}. This is an approach we followed in previous work \cite{Hawes:2012}, where we defined CDSRs using \textit{anchor points} \cite{DBLP:journals/jetai/KlenkFTK11} as the vertices of region polygons, thereby linking the regions to perceived objects.  While satisfying the requirements of a cognitive systems solution, considering the application of this model to the  results of our experiment illustrates two limitations.  First, regions are not always physically bounded by objects.  In the classroom examples, people frequently did not extend the front of the room all the way to the desks.  Second, people consider \emph{dynamics} part of the context, e.g. in stimuli (l), people did not indicate that it was safe behind the lone turret in front of the others.

The connection between spatial regions and function has been explored in other robotics work, e.g. ~\cite{Karg:2012,Fasola:2013}, but these regions are based on fields anchored at single points in the environment or on a particular object. Such a representation cannot account for the changes in context shown in this experiment, changes which depend on multiple objects, agents, and function.

%JK Previous version above I replaced it with this shorter version below
% Nick: now reinstating for longer version and editing

% Cognitive systems seek to reproduce the full range of human behavior. Spatial reasoning is an important element of human behavior.  While spatial reasoning is an established area in AI and robotics, CDSRs present a new problem which requires solutions beyond the current state-of-the-art. Regions are central to qualitative spatial reasoning (QSR) work~\cite{Cohn:2001} and also in the area of cognitive robotics (see, \emph{inter alia,} ~\cite{brenneretal07ijcai,kelleher/costello:09}). A fundamental decision when creating a computational model of a CDSR is the representation of its boundary. Given their context-dependent nature, boundary definitions based purely on single geometric shapes or a fixed number of points, seem unlikely to model the regions with sufficient accuracy. However, arbitrary boundaries can be defined using a sequence of geometric primitives (usually straight lines). This is an approach we followed in previous work \cite{Hawes:2012}, where we defined CDSRs using \textit{anchor points} as the vertices of the region polygon thereby linking the conceptual regions to perceived objects. One limitation of the \textit{anchor point} approach is that regions are not always physically bounded by objects.  In the classroom examples, people frequently did not extend the front of the room all the way to the desks. The connection between spatial regions and function has been explored in other robotics work, e.g. ~\cite{Karg:2012,Fasola:2013}, but these regions are based on fields anchored at single points in the environment or on a particular object. Therefore, they cannot account for the changes in context shown in this experiment which depend on multiple objects, agents, and simulation.  McLure and Forbus \cite{McLureForbus:2012} present a learning system that selects which encoding schemes work best for identifying geographic features (e.g., bottlenecks).  Although this approach incorporates functional aspects of the geography, it is unclear how it would apply to the other situations in our experiment.  


%--------------------------


%Once a particular representation has been modelled in this manner, we were able to find the same type of region in previously unseen rooms using analogical transfer. An example of the representation and transfer approach is Figure \ref{fig:anchor-point-transfer}.  In the know example, a human defines the front of the classroom as a polygon consisting of four \textit{anchor points}.  Analogical transfer works by aligning structured representations and transferring partially aligned statements from the known example to the new environment.  In this case, the \textit{anchor points} are transferred from the group of desks in the known example to the corresponding group of desks in the new environment.

%\begin{figure}
%  \includegraphics[width=\columnwidth]{figures/classroom-transfer.png}
%  \caption{The known example defines the front of the room with four \textit{anchor points}.  By transferring the \textit{anchor points} to the new environment, the system is able to ground the room's front in its perception.}
%  \label{fig:anchor-point-transfer}
%\end{figure}



\section{Conclusion}

% Cognitive systems do not face problems in isolation.  Context, including its goals, objects in the environment, and other agents, impact how information is processed and how actions are undertaken.  

In this paper, we presented a human study which explored the requirements for reasoning about context-dependent spatial regions in a cognitive system.  We found that (1) people can communicate with CDSRs, although their ability to do so decreases when situations become ambiguous and (2) varying the location and types of the objects and agents in an environment produced changes in the location, size, and shape of the CDSR.  We posit that a cognitive system should be able to learn new regions, account for changing contexts, and apply this knowledge in multiple reasoning tasks. Solutions to this problem will significantly improve the interaction between people and intelligent systems in human environments with applications to domestic robots, security, and activity recognition.

\begin{acknowledgements} 
\noindent
The research leading to these results has received funding from the European Union Seventh Framework Program (FP7/2007-2013) under the project STRANDS (grant agreement No 600623), and from the EPSRC under grant EP/K014293/1.
\end{acknowledgements} 




\vspace{-0.25in}

{\parindent -10pt\leftskip 10pt\noindent
\footnotesize
\bibliographystyle{cogsysapa}
\bibliography{cdsr}

}

% Leave a blank line before the closing brace to ensure the final 
% reference has the proper indentation. 

\end{document} 
