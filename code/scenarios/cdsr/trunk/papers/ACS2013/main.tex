\documentclass[11pt,letterpaper]{article}
\usepackage{cogsys}
\usepackage{cogsysapa}
% \usepackage{apacite}
% \usepackage{graphicx}
\usepackage[T1]{fontenc}
\usepackage{times}
\usepackage[pdftex]{graphicx} % use this when importing PDF files


% Put back in

 % First page headings for accepted submissions.
% \cogsysheading{1}{2012}{1-18}{9/2012}{12/2012}
 % First page headings for poster submissions.
%\cogsysposterheading{First}{2012}{1-18}

% \ShortHeadings{Formatting Instructions}
              % {P.\ Langley, G.\ Hunt, and D.\ G.\ Shapiro}

\begin{document} 

\title{CDSR Human Data/Modelling Paper}
 
% \author{Pat Langley}{langley@asu.edu}
% \author{Glen Hunt}{glen.hunt@asu.edu}
% \address{Computing Science and Engineering, Arizona State University, 
%          Tempe, AZ 85287 USA}
% \author{Daniel G.\ Shapiro}{dgs@isle.org}
% \address{Institute for the Study of Learning and Expertise, 
%          2164 Staunton Court, Palo Alto, CA 94306 USA}
\vskip 0.2in
 
\begin{abstract}
Abstract goes here
\end{abstract}

\section{Introduction - KLENK} 
 
\section{Context-Dependent Spatial Regions - KLENK}

\begin{itemize}
	\item Explanation of the basic idea.
	\item The desiderata of the model - context, extent, generalises to unseen examples

\end{itemize}



\section{Human Studies - John}
Research Questions: 
\begin{enumerate}
	\item Does context matter?
	\item Do people agree? 
\end{enumerate}
 

\section{Modeling - Nick}
If we want to capture this data in a generative model, what is our hypothesis space. 
Related work? 

\subsection{Symbolic Boundary Method - KLENK}

Overview
Example
Results?
Discussion

\subsection{Kernel Methods - NICK}

Overview - either Chris’s approach or something similar
Example
Results?
Discussion 
Where do the points come from

\section{Discussion}
 
\begin{acknowledgements} 
\noindent
STRANDS etc.
\end{acknowledgements} 




\vspace{-0.25in}

{\parindent -10pt\leftskip 10pt\noindent
\bibliographystyle{cogsysapa}
\bibliography{cdsr}

}

% Leave a blank line before the closing brace to ensure the final 
% reference has the proper indentation. 

\end{document} 
