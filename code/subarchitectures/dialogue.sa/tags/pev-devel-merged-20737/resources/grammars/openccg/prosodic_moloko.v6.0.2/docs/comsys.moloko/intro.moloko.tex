\section{Introduction}

This is where a nice sexy introduction will soon appear. Until then, my hopes is that these few words plus the table of contents will provide enough of a guide.

Basically, \textbf{chapter 1} discusses the overall 'themes' of the semantics, though I am still VERY unhappy with the intro and sections 2.1 and 2.2. So don't put too much stock in em.

Chapters 3 to 5 discuss major and novel design decisions. \textbf{Chapter 3} describes why adjectives lexically receive ONLY \cxx{adj}{M} form (i.e why \cx{n}{T} \fwdsl{} \cx{n}{T}  is generated via rule). \textbf{Chapter 4} discusses how we have managed to constrain event modifiers lexically. This is the counterpart to not having every imaginable adjunct attached to every verbal argument structure (something which Chapter 1 currently does a crappy job of explaining). \textbf{Chapter 5} discusses how we have essentially added clause-level mood handling constructions to the grammar, and why this is a good thing. It also overviews question semantics.

\textbf{Chapter 6} discusses the fricking copular verb in detail. Including lots of examples of questions.

\textbf{Chapter 7} discusses how the grammar was designed to be incremental, including a few nice step-by-step examples.

\textbf{Chapter 8} goes in detail through all of the families, syntactic features, rules, dictionary macros, etc.

Finally \textbf{Chapter 9} gives a brief tour of the grammar files (this will be useful for the next Moloko master,  whoever the poor schmuck may be).

