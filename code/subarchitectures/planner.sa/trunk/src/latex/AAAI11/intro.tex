


%% LONG PLANNING HORIZON. MORITZ, CAN YOU PLEASE GIVE ME THE NUMBER OF
%% STEPS IN THE BIGGER PLANS. 

A number of recent integrated robotic systems incorporate a
high-level {\em continual planning} and execution monitoring
subsystem~\cite{wyattetal2010tamd,talamadupula:2010,Kraft2008}.
%%
For the purpose of planning, sensing is modelled {\em
deterministically}, and beliefs about the underlying state are
modelled {\em qualitatively}.
%%
Both~\citeauthor{talamadupula:2010} and~\citeauthor{wyattetal2010tamd}
identify
\emph{continual planning} with {\em probabilistic} --i.e., {\em quantitative}--
models of noisy sensing and state as an important challenge for future
research.
%%
Motivating that sentiment, planning according to accurate
stochastic models should yield more efficient and robust
deliberations.
%%
%% In the first place, that means moving from deterministic to
%% probabilistic models of noisy sensing, and from qualitative to
%% quantitative models of state uncertainty. 
%%
In essence, the challenge is to develop a {\em base planner} that
exhibits similar speed and scalability as planners employed in
existing robotic systems ---e.g.,~\citeauthor{wyattetal2010tamd} use a
satisficing {\em classical} procedure--- which is also able to
synthesise relatively efficient deliberations according to detailed
probabilistic models of the environment.


This paper describes a {\em switching} domain independent planning
approach we have developed to address this challenge. 
%%
Our Planner is {\em continual} in the usual sense that plans are
adapted and rebuilt online in reaction to changes to the model of the
underlying problem and/or domain -- e.g., when goals are
modified, or when the topological map is altered by a door being
closed.
%%
It is integrated on a mobile robot platform that continuously
deliberates in a stochastic dynamic environment in order to achieve
goals set by the user, and acquire knowledge about its surroundings.
%%
Our planner takes problem and domain descriptions expressed in a novel
extension of PPDDL~\cite{younes:etal:2005}, called {\em
Decision-Theoretic (DT)PDDL}, for modelling stochastic decision
problems that feature partial observability.  In this paper we
restrict our attention to problem models that correspond to
deterministic-action multi-goal-oriented POMDPs\footnote{We note that
POMDPs with stochastic actions can be compiled into equivalent
deterministic-action POMDPs, where all the original action uncertainty
is expressed in the starting-state
distribution~\cite{ng:Jordan:2000}.} in which all actions have
non-zero cost -- i.e., an optimal policy can be formatted as a finite
horizon contingent plan. Moreover, we target problems of a size and
complexity that is challenging to state-of-the-art sequential
satisficing planners, and which are too large to be solved directly by
{\em decision-theoretic} (DT) systems.

Our planner {\em switches}, in the sense that the base planning
procedure changes depending on our robot's subjective degrees of
belief, and progress in plan execution. When the underlying planner is
a fast (satisficing) {\em classical} planner, we say planning is in a
{\em sequential} session, and otherwise it is in a DT session.
%%
A {\em sequential} session plans, and then pursues a high-level
strategy -- e.g., go to the kitchen bench, and then observe the
cornflakes on it.
%%
A DT session proceeds in a practically sized {\em abstract} process,
determined according to the current sequential strategy and underlying
belief-state.
%%


We evaluate our approach in simulation on problems posed by {\em
object search} and {\em room categorisation} tasks that our indoor
robot undertakes. Those feature a deterministic task planning aspect
with an active sensing problem. The larger of these problems features
$6$ rooms, $25$ topological places, and $21$ active sensing
actions. The corresponding decision process has a number of states
exceeding $10^{36}$, and high-quality plans require long ---hundreds
of actions--- planning horizons.
%%
Although our approach is not optimal, particularly as it relies on the
results of satisficing sequential planning directly, we find that it
does nevertheless perform better than a purely sequential replanning
baseline. Moreover, it is fast enough to be used for real-time
decision making on a mobile robot.
