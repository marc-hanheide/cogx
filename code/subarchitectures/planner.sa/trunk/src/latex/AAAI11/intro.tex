


%% LONG PLANNING HORIZON. MORITZ, CAN YOU PLEASE GIVE ME THE NUMBER OF
%% STEPS IN THE BIGGER PLANS. 

A number of recent integrated robotic systems incorporate a
high-level {\em continual planning} and execution monitoring
subsystem~\cite{wyattetal2010tamd,talamadupula:2010,Kraft2008}.
%%
For the purpose of planning, sensing is modelled {\em
deterministically}, and beliefs about the underlying state are
modelled {\em qualitatively}.
%%
Both~\citeauthor{talamadupula:2010} and~\citeauthor{wyattetal2010tamd}
identify
\emph{continual planning} with {\em probabilistic} --i.e., {\em quantitative}--
models of noisy sensing and state as an important challenge for future
research.
%%
Motivating that sentiment, planning according to such accurate
stochastic models should yield more efficient and robust
deliberations.
%%
%% In the first place, that means moving from deterministic to
%% probabilistic models of noisy sensing, and from qualitative to
%% quantitative models of state uncertainty. 
%%
in essence, the challenge is to develop a {\em base planner} that
exhibits similar speed and scalability as planners employed in
existing robotic systems ---e.g.,~\citeauthor{wyattetal2010tamd} use a
satisficing {\em classical} procedure--- which is also able to
synthesise relatively efficient deliberations according to detailed
probabilistic models of the environment.


This paper describes a {\em switching} domain independent planning
approach we have developed to address this challenge. This is
implemented on a mobile robot platform that continuously deliberates
in a stochastic dynamic environment in order to achieve goals set by
the user, and acquire knowledge about its surroundings.
%%
Our planner takes problem and domain descriptions expressed in a novel
extension of PPDDL~\cite{younes:etal:2005}, called {\em
Decision-Theoretic (DT)PDDL}, for modelling stochastic decision
problems that feature partial observability.  In this paper, we
restrict our attention to problem models that correspond to
deterministic-action goal-oriented POMDPs\footnote{We note that POMDPs
with stochastic actions can be compiled into equivalent
deterministic-action POMDPs, where all the original action uncertainty
is expressed in the starting-state
distribution~\cite{ng:Jordan:2000}.} -- i.e., where an optimal policy
can be formatted as a finite horizon contingent plan.  The planner
{\em switches}, in the sense that the underlying planning procedure
changes depending on our robot's subjective degrees of belief, and
progress in plan execution. When the underlying planner is a
(satisficing) {\em classical} planner, we say planning is in a {\em
sequential} session, and otherwise it is in a {\em decision-theoretic}
(DT) session. Finally, planning is {\em continual} in the usual sense
that, whatever the session, plans are adapted and rebuilt online in
reaction to changes to the model of the underlying problem and/or
domain -- e.g., when objectives are modified, or when the topological
map is altered by a door being closed.


We evaluate our approach in simulation on problems posed by {\em
object search} and {\em room categorisation} tasks that our indoor
robot undertakes. Those feature a deterministic task planning aspect
with an active sensing problem. The larger of these problems features
$6$ rooms, $25$ topological places, and $21$ active sensing
actions. The corresponding decision processes have a number of states
exceeding $10^{36}$, and high-quality plans require long ---hundreds
of actions--- planning horizons.
%%
Although our approach is not optimal, particularly as it relies on the
results of satisficing sequential planning directly, we find that it
does nevertheless perform better than a purely sequential replanning
baseline. Moreover, it is fast enough to be used for real-time
decision making on a mobile robot.

%% the significant source of uncertainty in the
%% domain is the unreliability of observations and this approach is
%% particularly suited to problems that combine a deterministic task
%% planning problem with observations because the decision-theoretic planner
%% is only invoked to deal with state uncertainty.

%% The switching planner provides important advantages in our
%% mobile robot domain.

%% Our system is domain independent, taking domain and problem
%% descriptions in a first-order declarative language we have developed,
%% called the decision-theoretic planning domain definition language
%% (DTPDDL). In this paper, we restrict our attention to deterministic
%% action goal-oriented POMDPs\footnote{We note that POMDPs with
%% stochastic actions can be compiled into equivalent
%% deterministic-action POMDPs, where all the original action uncertainty
%% is expressed in the starting-state
%% distribution~\cite{ng:Jordan:2000}.} -- I.e., where a finite horizon
%% optimal contingent plan exists.  
%%
%% From these descriptions of POMDPs, we
%% automatically construct a deterministic model for the sequential
%% planner. That model includes actions which correspond to making
%% assumptions about the values of imperfectly known state
%% variables. Assumptions scheduled by the sequential system are used to
%% propose a pragmatic abstract belief-state space to the
%% decision-theoretic system, and to modify the reward function, so that
%% system might pursue sensing related to those assumptions.


%% The switching planner provides important advantages in our
%% mobile robot domain. First, the significant source of uncertainty in the
%% domain is the unreliability of observations and this approach is
%% particularly suited to problems that combine a deterministic task
%% planning problem with observations because the decision-theoretic planner
%% is only invoked to deal with state uncertainty.
%% %determine a characteristic of the underlying state. 
%% Secondly, replanning makes the system robust to
%% changing objectives and discoveries about the world,
%% both of which feature in our domain.



%% The remainder of the paper proceeds as follows. We begin by describing
%% related work. Next we describe our DTPDDL language with example
%% declarations from a mobile robot exploration task. Then we describe
%% the switching planner, how sequential planning proceeds in our problem
%% models, and how this then provides the input for the
%% decision-theoretic planner. Finally we provide an evaluation of the
%% system on the example robot domain, and then provide some concluding
%% remarks and future directions.





%% BEGIN SCRAPPY


%% The initial abstract process, let's suppose, has two states. The state
%% where everything assumed is true, and the null state. This has a KL
%% divergence from the :init term, because we can suppose it uniformly
%% chooses one concrete state, given the abstract :init states.

%% As you add propositions to the abstract :init term, then the KL
%% divergence decreases. 


%% END SCRAPPY

%%

%% In our work we take a concrete step towards addressing all the
%% challenges we outlined. We have developed a {\em switching}
%% domain-independent planning system that operates according to the
%% continual planning paradigm.  It uses first-order By autonomously mixing these two
%% types of sessions our robot is able to be robust and responsive to
%% changes in its environment
%% \emph{and} make appropriate decisions in the face of uncertainty.




%% This domain features partial observability, particularly
%% because the state is not perfectly known, and state information is
%% gained by the robot through the use of noisy sensing
%% actions. 
%% %Using a propositionally factored state representation, for
%% For
%% interesting sized tasks the corresponding POMDP model is too large for a
%% POMDP solver to be applied directly. Inspired by the replanning
%% approaches used successfully for MDP
%% planning~\cite{yoon:etal:2007}, we propose a continual
%% planning approach that uses a classical planning system to compute a
%% reasonably valuable trace in the model, and then uses a
%% decision-theoretic planner on small subproblems where reasoning about
%% observations might be useful. We refer to the complete system as a
%% {\em switching continual planner}. The approach is not optimal,
%% particularly as it relies on the results of satisficing sequential planning
%% directly. It does nevertheless perform better than a purely sequential
%% replanner, and is fast enough to be used for real-time decision-making
%% on a mobile robot.



%% The underlying planning procedures are {\em classical}, i.e.,
%% sequential, planning system.



%% Planning for agents acting in partially observable environments with
%% stochastic actions is extremely computationally
%% expensive~\cite{mdp-complexity}. Even relatively small partially
%% observable Markov decision problems (POMDPs), represented by tens of
%% propositional variables, have state spaces far larger than existing
%% optimal or close to optimal planners can handle. For the easier
%% problem of planning in completely observable stochastic domains, an
%% alternative has been to use classical sequential planners and
%% replan when actions have unplanned-for outcomes~\cite{yoon:etal:2007}.
%% Execution monitoring tracks properties of the state that determine
%% whether further planning is required. For partially observable
%% domains, only probabilistic information about such properties will, in
%% general, be available.



%% However, these approaches rely
%% crucially on being able to determine properties of the current state
%% in order to decide to replan. For partially observable domains, only
%% probabilistic information about these properties will in general be
%% available.



%% Switching gives us some important advantages in our robotic
%% domain. First, the significant source of uncertainty in the domain is
%% the unreliability of observations, and this approach is particularly
%% suited to problems that combine a deterministic task planning problem
%% with observations as the decision-theoretic planner is often invoked
%% to determine a characteristic of the underlying state. Secondly,
%% replanning makes our system robust to changing objectives and the
%% discovery of new facts about the world.


%% As an illustration, consider a robot searching for a box that it knows
%% is often found in offices. The sequential planner might produce a plan
%% that assumes that {\sc Room a} is an office, that the box is in {\sc
%% Room a}, and that the box is in location 1 in {\sc Room a}. Given
%% these assumptions, the plan might be to go to {\sc Room a}, go to
%% location 1 and search for the box. When the plan gets executed and the
%% search begins the decision-theoretic planner is called since searching
%% involves making observations. The decision-theoretic planner might
%% retract the last assumption but leave the others, and build a plan to
%% search the whole of {\sc Room a} for the box. If this plan fails to
%% find the box, it will disprove the assumption that the box is in {\sc
%% Room a}, which will then lead to replanning by the sequential planner
%% based on the newly discovered knowledge.


%To accompany
%that language we have implemented an information-state
%\laostar\ procedure for solving problems expressed in DTPDDL. 

%%

%Have a good understanding of the contents and function of rooms, as
%well as the linguistic referrents to rooms, widgets, and their visual
%qualities.

%Having confidence in its beliefs about the linguistic terms for
%spaces, and the function of those spaces.



%sometimes difficult to predict, with exogenous events, such as the
%changing of the objective, 





%the speed and scalability of software for sequential planning in
%deterministic




%% We developed a first-order declarative language, called DTPDDL, for
%% describing domains of planning problems that correspond to POMDPs.  To
%% accompany that language we have implemented an information-state
%% \laostar\ procedure for solving problems expressed in DTPDDL. 

%% We have extended MAPSIM to parse and simulate DTPDDL problems, and
%% modified \fastdownward\ so that it can find useful sequential executions
%% given DTPDDL models of the problem at hand.




%%  called DTPDDL, along
%% the lines of PDDL for the partially observable case, an \laostar\
%% solution procedure, a determinisation of the DTPDDL problem in MAPL,
%% and modify the \fastdownward\ system to find high quality sequential plans
%% .

%% We model the environment as a partially observable Markov decision
%% process. Although planning in that model is undecidable in
%% general~\cite{}, an optimal finite-horizon plan corresponds to a
%% contingent plan, that is a function mapping action-observation
%% histories to actions.

%% our continual planner is reactive, replanning whenever the underlying
%% domain and problem models change. For example, replanning occurs if
%% the motivational component alters the objectives, and if an assumed
%% outcome of a sensing action is not realised.

%% brittle sensing model, 

%% the evaluation of a fluent at a state is either known. Moreover, there
%% is a sensing process that run-time variables  after-which 

%% For $\prop \in \state$ we say proposition $\prop$ characterises state
%% $\state$. There is always a unique starting state $\state_0$. The goal
%% $\goal$ is a set of propositions, and we say that state $\state$ is a
%% goal state iff $\goal \subseteq \state$.

%%  To keep this exposition
%% simple, for any two distinct actions $\stochAction_i \neq
%% \stochAction_j$, if outcome $\detAct$ is a possibility for
%% $\stochAction_i$ then it cannot also be a possibility for
%% $\stochAction_j$ -- i.e., if $\detAct \in
%% \detActions(\stochAction_i)$ then $\detAct \not\in
%% \detActions(\stochAction_j)$.


%% The solution to a probabilistic planning problem is a contingency
%% plan. This consists of an assignment of actions to states at each
%% discrete timestep up to the planning horizon $n$. The optimal
%% contingency plan is one which prescribes actions to states that
%% maximise the probability that the goal is achieved within $n$ steps
%% from the starting state $\state_0$. For the purposes of this paper
%% we say a plan fails, i.e. achieves the goal with probability $0$, in
%% situations where it does not prescribe an action. Computing the
%% optimal plan for a problem is computationally intractable, and an
%% important direction for research in the field is to develop
%% heuristic mechanisms for generating small linear plans
%% quickly~\cite{littman:etal:98}

%%  which can operate
%% with the original representation of the domain.


%% The sequential planner we use is based
%% on \fastdownward~\cite{fast-downward}. We add the capability to replan
%% ~\cite{brenner:nebel:jaamas09}, and also allow agent
%% knowledge to be represented explicitly, so we can write actions that
%% gain the agent knowledge. LEASE CITE MICHAEL's WORK AS THE BASIS FOR
%% THIS. For the decision-theoretic planner we have implemented our own
%% forward search in the belief-state space.
