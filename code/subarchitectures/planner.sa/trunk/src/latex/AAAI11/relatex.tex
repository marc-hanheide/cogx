


%%  and
%% using various techniques to manage the large state
%% space

%% \citeauthor{hippo-jnl}~(\citeyear{hippo-jnl}) take this
%% approach in a 



Addressing task and observation planning specifically, there have been
a number of recent developments where the underlying problem is
modelled as a POMDP.
%%
For vision algorithm
selection,~\citeauthor{hippo-jnl}~(\citeyear{hippo-jnl}) exploit a
natural hierarchical decomposition of the underlying POMDP into
subproblems where decision-theoretic planning is reasonably
fast. \citeauthor{doshi08:pref_elic}~(\citeyear{doshi08:pref_elic})
represent a preference elicitation problem as a POMDP and take
advantage of symmetry in the belief space ---essentially the idea that
it does not matter what the value of the variable you are trying to
observe is--- to exponentially shrink the state space. Although we
have been actively exploring the \citeauthor{doshi08:pref_elic}
approach, those exploitable structures are not present in problems we
have considered so far due to the task planning requirement. 
%%
%%
Finally,
our approach is in a similar vein to the more classical {\em
dual-mode} control~\cite{cassandra96actingunder}, however in our case
entropy reduction is targeted by planning in an abstract process which
is informed by one execution trace ---computed by a {\em classical}
planner--- and the underlying belief-state.


More generally, there has been much recent work on scaling POMDP
solution procedures to medium-sized
instances~\cite{shani:etal:08,KurHsu08}. In that
setting \citeauthor{kurniawati:etal:2010}~(\citeyear{kurniawati:etal:2010})
recently addressed an inefficiency of offline point-based techniques
in problems with medium length planning horizons, however their
approach take tens of minutes to good plans, and only scales to tens
of thousands of states. In the case of general domain-independent
factored systems, the state-of-the-art scales to relatively small
problems with $2^{22}$ states~\cite{shani:etal:2008}. At their limit,
these procedures take over an hour to converge, and $\sim10$ seconds
on average to perform a single Bellman backup.  For classes of POMDP
that feature exploitable structures (e.g., no actions with negative
effects), problems with as many as $10^{30}$ states can be targeted by
offline procedures~\cite{brunskill:russell:2010}. Moving someway
towards addressing real-time decision making, recent online POMDP
solution procedures have been developed which can exploit highly
approximate value functions -- typically computed using a point-based
procedure -- and heuristics in forward
search~\cite{ross:etal:2008}. Their applicability in our setting is
limited, firstly because they only scale to smaller problems, with
thousands of states, and also due to the large amount of
\emph{problem-specific} offline processing that might be required to get useful
search guidance. A {\em very} recent and promising online approach for
large POMDPs employs Monte-Carlo sampling to break the curse of
dimensionality in situations where goal reachability is {\em easily}
determined~\cite{silver:veness:2010}.


%% Although we suppose it an interesting
%% item for future work to pursue that direction, it should be noted that
%% ease of goal reachability is not guaranteed in the problems we face,
%% and is certainly not a property to be assumed in domain independent
%% planning.


In the direction of leveraging {\em classical} systems/approaches for
planning under uncertainty, the most highlighted system to date has
been \system{FFR$_a$}~\cite{yoon:etal:2007}; The winning entry from
the probabilistic track of the 2004 International Planning
Competition.  In the continual paradigm, \system{FFR$_a$}
uses \system{FF} to compute sequential plans and execution traces.
%%
More computationally expensive approaches in this vein combine
sampling strategies on valuations over {\em runtime variables} with
deterministic planning procedures~\cite{yoon:etal:2008}. %% The outcome is typically a more
%% robust sequential plan, or contingent
%% plan~\cite{majercik:2006}. 

Also leveraging deterministic planners in problems that feature
uncertainty, \system{Conformant-FF}~\cite{hoffmann:brafman:2006} and
$T_0$~\cite{palacios:geffner:2009} demonstrate how conformant planning
---i.e., sequential planning in unobservable worlds--- can be modelled
as a deterministic problem, and therefore solved using sequential
systems. In this conformant setting, advances have been towards
compact representations of beliefs amenable to existing best-first
search planning procedures, and lazy evaluations of beliefs. Most
recently this research thread has been extended to contingent planning
in fully observable non-deterministic
environments~\cite{albore:etal:2009}.
%%
%% The continual planning system that motivated our
%% project~\cite{brenner:nebel:jaamas09} also has this characteristic,
%% and has been applied in completely observable domains, particularly
%% those featuring multiple communicating agents. 

%% The use of knowledge
%% operators in domains allows plans that act to gain knowledge, but the
%% approach assumes that such actions are deterministic and reliable, an
%% assumption that we relax.



%%No! They simply haven't been evaluated in PO settings. They may, or
%%may not struggle. They have sampling of traces, and that would
%%include observations, and therefore evolutions of beliefs. SSAT was
%%proposed by Littman for POMDPs. So the majercik stuff is perfectly
%%suited to POMDPs.

% Normally if we are going to compare with related work, we do
% actually *compare*. Why didn't you try those approaches? I think
% it's safe to say that FFR will struggle. Why would it even include
% in its plan an observational action that doesn't change the world?

%%
%% However, as we said in the introduction,
%% all these approaches struggle in partially observable domains as they
%% rely on being able to determine the state at all times.
