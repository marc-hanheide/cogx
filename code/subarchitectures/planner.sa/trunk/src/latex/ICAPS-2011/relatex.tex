There have been a number of developments recently towards planning
under uncertainty using systems that were intended for sequential
planning in deterministic problems.  Notably for example,
\system{FFR$_a$}~\cite{yoon:etal:2007}, the winning entry from the
probabilistic track of the 2004 International Planning Competition.
In the continual paradigm, \system{FFR$_a$} uses the fast satisficing
procedure \system{FF}~\cite{hoffmann:nebel:2001} to compute sequential
plans and corresponding execution traces.
%%
More computationally expensive approaches in this vein combine
sampling strategies on valuations over {\em runtime variables} with
deterministic planning procedures. The outcome is typically a more
robust sequential plan~\cite{yoon:etal:2008}, or contingent
plan~\cite{majercik:2006}. However, as we said in the introduction,
all these approaches struggle in partially observable domains as they
rely on being able to determine the state at all times.

Also leveraging deterministic planners in problems that feature
uncertainty, \system{Conformant-FF}~\cite{hoffmann:brafman:2006} and
$T_0$~\cite{palacios:geffner:2009} demonstrate how conformant planning
---i.e., sequential planning in unobservable worlds--- can be modelled
as a deterministic problem, and therefore solved using sequential
systems. In this conformant setting, advances have been towards
compact representations of beliefs amenable to existing best-first
search planning procedures, and lazy evaluations of beliefs. Most
recently this research thread has been extended to contingent planning
in fully observable non-deterministic
environments~\cite{albore:etal:2009}.
%%
The continual planning algorithm we use \cite{brenner:nebel:jaamas09}
also has this characteristic, and has been applied in completely
observable domains, particularly those featuring multple communicating
agents. The use of knowledge operators in domains allows plans that
act to gain knowledge, but the approach assumes that such actions are
deterministic and reliable, an assumption that we relax in this work.

The particular domain we are interested in contains a mixture of task
planning and observation planning. There have been a number of recent
papers representing observation planning problems as POMDPs and using
various techniques to manage the large state
space. In \cite{hippo-jnl} the authors take this approach in a vision
algorithm selection problem. In their case there is a natural
factorisation of the problem which allows them to solve large problems
by decomposing them into a set of much smaller POMDPs. Doshi and
Roy \cite{doshi08:pref_elic} represent a preference elicitation
problem as a POMDP and take advantage of symmetry in the belief space
(essentially the idea that it doesn't matter what the value of the
variable you are trying to observe is) to exponentially shrink the
state space. Unfortunately, although we are still exploring the Doshi
and Roy approach, we have yet to find structure we can exploit
similarly in our domain due to the task planning component.
