
%% {\em runtime variables}
%% {\em decision variables}
%% {\em uninitialised variables}
%% {\em omitted variables}

We describe the partially observable propositional probabilistic
planning problem, with costs and rewards. We model a process
state $\state$ as the set of propositions that are true of the
state. Notationally, we have $\state \subseteq
\props$. The underlying process
dynamics are modelled in terms of a finite set of {\em probabilistic
STRIPS operators}~\cite{boutilier:1996} $\stochActions$ over
state-characterising propositions $\propositions$.
%%
We say an action $\stochAction \in \stochActions$ is
applicable if its precondition $\poss(\stochAction)$, a set of
propositions, are satisfied in the current state -- i.e., $\poss(\stochAction) \subseteq
\state$. We denote $\mu_{\stochAction}(\detAction_i)$ the probability that
nature chooses a deterministic STRIPS effect $\detAct_i$, and for
all \stochAction\ we require
$\sum_{\detAction_i \in \detActions(\stochAction)}
\mu_{\stochAction}(\detAction_i) = 1$.

%% We model a process state $\state$ as the set of propositions that are
%% true of the state. Notationally, we have $\state \subseteq
%% \props$. State change is induced by application of actions. A
%% stochastic action $\stochAction \in \stochActions$ is applicable if
%% its precondition $\poss(\stochAction)$, a set of propositions, are
%% satisfied in the current state. We write $\stochActions(\state)$ for set of actions
%% applicable at state $\state$.  If action $\stochAction \in
%% \stochActions(\state)$ is applied at \state, nature chooses one
%% element amongst a small set of deterministic outcomes
%% $\detActions(\stochAction) \equiv \{\detAction_1, \ldots, \detAction_k
%% \}$. We denote $\mu_{\stochAction}(\detAction_i)$ the probability that
%% nature takes outcome $\detAct_i$, and for all \stochAction\ we require
%% $\sum_{\detAction_i \in \detActions(\stochAction)}
%% \mu_{\stochAction}(\detAction_i) = 1$. The chosen outcome has an
%% effect on the state given in terms of two lists of propositions. The
%% add-list $\add(\detAction)$ and delete-list
%% $\delete(\detAction)$.\footnote{If a proposition is in the add-list of
%% $\detAction$, then it cannot be in the delete-list and vice versa.}
%% If outcome $\detAction$ with $\add(\detAction) := [\prop_1,
%% ..,\prop_n]$ and $\delete(\detAction) := [\prop_{n+1},
%% ..,\prop_m]$ is chosen by nature when \stochAction\ is
%% applied at state $\state$, then the resultant state is $ ( \state \cup
%% \add(\detAction) ) \backslash \delete(\detAction)$ -- i.e.,
%% propositions from $\add(\detAction)$ are added to $\state$, and those
%% from $\delete(\detAction)$ are removed from $\state$.

%% \citeauthor{rintanen:01}~(\citeyear{rintanen:01})

We are concerned with problems that feature partial
observability. Although we could invoke {\em extended probabilistic
STRIPS operators}~\cite{rintanen:01} to model actions and observations
propositionally, we find it convenient for our presentation to clearly
separate sensing and action. Therefore, we suppose a POMDP has a
perceptual model given in terms of a finite set of stochastic {\em
senses} $\stochSenses$, deterministic sensing outcomes $\detSenses$,
and perceptual propositions $\percepts$, called {\em percepts}. In
detail, we take an observation $\observ$ to be a set of percepts
$\observ \subseteq \percepts$, and denote \observations\ the set of
reachable observations. The underlying state of the process cannot be
observed directly, rather, senses $\stochSense \in \stochSenses$
effect an observation $\observ \in
\observations$ that informs what should be believed about the state a
process is in. If $\action$ is applied effecting a
transition to a successor state $\state'$, then an observation occurs
according to the active senses $\stochSenses(\stochAction, \state')
\subseteq \stochSenses$. A sense $\stochSense$ is active, written
$\stochSense \in \stochSenses(\stochAction, \state')$, if the senses'
action-precondition, $\poss_\stochActions(\stochSense)$, is equal to
$\stochAction$, and the state-precondition $\poss_\states(\stochSense)
\subseteq \propositions$ is satisfied by the state $\state'$; In the
usual sense that $\poss_\states(\stochSense) \subseteq \state'$.
%%
When a sense is active, nature must choose exactly one outcome amongst
a small set of deterministic choices $\detSenses(\stochSense)
\equiv \{\detSense_1, \ldots, \detSense_k \}$, so that for each
$i$ we have $\detSense_i \subseteq \percepts$. The probability of
the $i^{th}$ element being chosen is given by
$\psi_{\stochSense}(\detSense_i)$, where $\sum_{\detSense_i \in
\detSenses(\stochSense)} \psi_{\stochSense}(\detSense_i) =
1$. The observation received by the agent corresponds to the union of
perceptual propositions from the chosen elements of active
senses.

A POMDP has a starting configuration that corresponds to a Bayesian
belief-state. Intuitively, this is the robot's subjective belief about
its environment. Formally, a belief-state $\bstate$ is a probability
distribution over process states. We write $\bstate(\state)$ to denote
the probability that the process is in $\state$ according to
$\bstate$, and $\bstate_0$ when discussing the starting
configuration. 

Finally, we make a number of fairly standard assumptions. First, that
action execution and sensing occurs instantaneously, and that only one
action can be applied at a plan-step. Second, it can arise that in
some propositional states an action is applicable, and in others it is
not. Moreover, a belief-state can assign non-zero probability to
states where that applicability holds, and states where it does
not. We differ
from~\citeauthor{younes:littman:04}~(\citeyear{younes:littman:04}),
because we forbid the execution of actions that are not applicable in
any state of the current belief.  Also,
unlike~\citeauthor{hoffmann:brafman:2006}~(\citeyear{hoffmann:brafman:2006}),
we do incorporate a PPDDL-like default semantics, treating an action
\stochAction\ executed at $\state_i$ as if it has no effect at
states $\state$ if $\poss(\stochAction) \not\subseteq
\state$.


\subsection{Costs, Rewards, and Belief Revision}

Until now we have discussed the POMDP in terms of propositions and
percepts. In order to address belief revision and utility it is
convenient to consider the underlying decision process in a flat
format. This is given by the tuple
$\langle \states, \bstate_0, \actions, \transProb, \reward,
\observations, \obsDist \rangle$. Here, $\bstate_0$ is the initial
belief-state, \states\ is the finite set of reachable propositional
states, \actions\ is the finite set of actions, and \observations\ is
the finite set of reachable observations (i.e., perceptual states).
Where $\state, \state' \in \states$, $a \in \actions$, from $\mu$ we
have a state transition function $\transProb(\state, \action,
\state')$ giving the probability of a transition from state $s$ to
$s'$ if $a$ is applied. For any $\state$ and $\action$ we have
$\sum_{\state' \in \states} \transProb(\state, \action, \state') = 1$.
%%
Function $\reward:\states \times \actions \to \Re$ is a bounded real
valued reward function. Therefore a finite positive constant $c$
exists so that for all $\state \in \states$ and $\action \in
\actions$, $|\reward(\state, \action)| < c$. We model costs as
negative rewards.
%%
From $\psi$ we have that for each $\state \in \states$ and action
$\action \in \actions$, an observation $\observ \in \observations$ is
generated independently according to a probability distribution
$\obsDist(\state, \action)$. We denote $\obsDist_\observ(\state,
\action)$ the probability of getting observation $\observ$ in state
$\state$. For $\state$ and $\action$ we have $\sum_{\observ \in
\observations} \obsDist_\observ(\state, \action) = 1$.

Successive state estimation is available by application of Bayes'
rule.  Taking the current belief $\bstate$ as the {\em prior}, and
supposing action $\action$ is executed with perceptive outcome
$\observ$, then the probability that we are in $\state$ in the
successive belief-state $\bstate'$ is given by:

\begin{equation}\label{eq:revision}
\bstate'(\state) = \frac{\obsDist_\observ(\state, \action)
  \sum_{\state'\in \states} \transProb(\state', \action, \state) \bstate(\state') }{\pp{Pr}(\observ | \action, \bstate)}
\end{equation}

\noindent where $\pp{Pr}(\observ | \action, \bstate)$ is the
normalising factor, that is, the probability of getting observation
$\observ$ given we execute $\action$ in $\bstate$.

\subsection{Plan Evaluation}

An optimal solution to a finite-horizon POMDP is a contingent plan,
and can be expressed as a mapping from observation histories to
actions. Although suboptimal in general, useful plans can also take a
classical sequential format. This is the case in {\em conformant}
planning, where the objective is to find a sequence of actions that
achieves a goal ---i.e., reaches a state that satisfies a given
Boolean condition--- with probability $1$.  Generally, whatever the
structure of a plan, its value corresponds to the expected cumulative
reward:

%% More generally, a useful
%% formalism for representing solutions to POMDPs is the finite-state
%% controller (FSC). This is a three-tuple $\langle \nodes, \selection,
%% \transition \rangle$ where: $\node \in \nodes$ is a set of controller
%% states, $\selection_\node(\action) = P(\action | \node)$ gives the
%% probability that the controller prescribes \action\ when in state
%% \node, and $\transition_\node(\action, \observation, \node') =
%% P(\node'|\node, \action, \observation)$ gives the probability the
%% controller transitions to state $\node'$, supposing we execute
%% \action\ in state \node\ receiving observation \observation. When
%% acting for finite $N$ steps from controller state $n$, the value of a
%% controller corresponds to the expected cumulative reward:

\begin{equation}\label{eq:expectedvalue}
V_{\pp{PLAN}}(\bstate) = \Expect \bigg{[} 
\sum_{t=0}^{N-1}  \reward(\bstate_t, \pp{PLAN}_t) \mid \pp{PLAN}, \bstate_0 = \bstate \bigg{]}
\end{equation}

\noindent Where $b_t$ is the belief-state at step $t$, $\pp{PLAN}_t$ is
the action prescribed at step $t$, and

\[\reward(\bstate, \action) = \sum_{\state \in \states}
\bstate(\state)\reward(\state, \action).\] 

%% \noindent Finally, it is
%% useful to note that there is a corresponding deterministic FSCs for
%% both sequential (e.g., conformant) and contingent plan formats.

