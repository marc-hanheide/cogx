
There have been a number of developments recently towards planning
under uncertainty using systems that were intended for sequential
planning in deterministic problems.  Notably,
\system{FFR$_a$}~\cite{yoon:etal:2007}, the winning entry from the
probabilistic track of the 2004 International Planning Competition.
In the continual paradigm, \system{FFR$_a$} uses the fast satisficing
procedure \system{FF}~\cite{hoffmann:nebel:2001} to compute sequential
plans and corresponding execution traces.
%%
More computationally expensive approaches in this vein combine
sampling strategies on valuations over {\em runtime variables} with
deterministic planning procedures. The outcome is typically a more
robust sequential plan~\cite{yoon:etal:2008}, or contingent
plan~\cite{majercik:2006}.
%%
Also leveraging deterministic planners in problems that feature
uncertainty, \system{Conformant-FF}~\cite{hoffmann:brafman:2006} and
$T_0$~\cite{palacios:geffner:2009} demonstrate how conformant planning
---i.e., sequential planning in unobservable worlds--- can be modelled
as a deterministic problem, and therefore solved using sequential
systems. In this conformant setting, advances have been towards
compact representations of beliefs amenable to existing best-first
search planning procedures, and lazy evaluations of beliefs. Most
recently this research thread has been extended to contingent planning
in fully observable non-deterministic
environments~\cite{albore:etal:2009}.
