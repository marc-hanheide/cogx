

Planning for agents acting in partially observable environments with
stochastic actions is extremely computationally
expensive \cite{mdp-complexity}. Even relatively small partially
observable Markov decision problems (POMDPs), represented by tens of
propositional variables, produce state spaces far larger than existing
optimal or close to optimal planners can handle. For the easier
problem of planning in completely observable stochastic domains, an
alternative approach has been to use classical sequential planners and
replan when actions have unplanned-for
outcomes \cite{yoon:etal:2007}. However, these approaches rely
crucially on being able to determine properties of the current state
in order to decide to replan. For partially observable domains, only
probabilistic information about these properties will in general be
available.

In this paper we describe, \pcogx, the planning component in a robotic
system that continuously deliberates in a stochastic dynamic
environment in order to achieve objectives set by the user, and
acquire knowledge about its surroundings. This domain has the
characteristics of a POMDP in that the state is not perfectly known,
and state information is gained by the robot through the use of
unreliable sensing actions. However, the state representation is
propositional, and even for relatively simple problems the underlying
model is too large for a POMDP solver to be applied.

Inspired by the replanning approaches used successfully for MDP
planning \cite{yoon:etal:2007,yoon:etal:2008} we propose an approach
that uses a replanning sequential planner to build the majority of the
plan, but uses a decision-theoretic planner on small subproblems where
reasoning about observations is required to find good plans. We refer
to the complete system as a {\em switching continual planner}. The
approach is not optimal, particularly as it is based on a satisficing
sequential planner, but it performs better than the sequential planner
alone and is fast enough to be used in real-time on a mobile robot for
decision-making. 

As an illustration, consider a robot searching for a box that it knows
is often found in offices. The sequential planner might produce a plan
that assumes that {\sc Room a} is an office, that the box is in {\sc
Room a}, and that the box is in location 1 in {\sc Room a}. Given
these assumptions, the plan might be to go to {\sc Room a}, go to
location 1 and search for the box. When the plan gets executed and the
search begins the decision-theoretic planner is called since searching
involves making observations. The decision-theoretic planner might
retract the last assumption but leave the others, and build a plan to
search the whole of {\sc Room a} for the box. If this plan fails to
find the box, it will disprove the assumption that the box is in {\sc
Room a}, which will then lead to replanning by the sequential planner
based on the newly discovered knowledge.

To represent planning problems of this kind, we have developed a
first-order declarative language, called decision-theoretic planning
domain definition language (DTPDDL), for describing domains of
planning problems that correspond to POMDPs. We have created a system
that automatically constructs from this a deterministic planning domain
for the sequential planner that includes actions that allow
assumptions to be made about the values of imperfectly known state
variables. These are then used to restrict the state space and set the
reward function for the decision-theoretic planner, which can operate
with the original representation of the domain.

The sequential planner we use is based
on \fastdownward \cite{fast-downward}. We add the capability to
replan, and also allow agent knowledge to be represented explicitly,
so we can write actions that gain the agent knowledge. For the
decision-theoretic planner we use forward search in the information
state.

The switching planner approach gives us some important advantages in
our robotic domain. First, the significant source of uncertainty in
the domain is the unreliability of observations, and this approach is
particularly suited to problems that combine a deterministic task
planning problem with observations as the decision-theoretic planner
is only invoked to determine a characteristic of the underlying
state. Secondly, the replanning makes the system very robust to
changing objectives and the discovery of new facts about the world, both of
which are features of our domain.

The remainder of the paper proceeds as follows. We begin by describing
POMDPs with a propositionally factored representation of states and
observations. We also outline the relationship between this and a flat
POMDP representation. Next we describe our DTPDDL language with
example declarations from a mobile robot exploration task. Then we
describe the switching planner, including how the sequential domain is
produced and a plan is generated, and how this then provides the input
for the decision-theoretic planner. Finally we provide an evaluation
of the system on the example robot domain, and finish with a short
survey of related work and some concluding remarks.





%To accompany
%that language we have implemented an information-state
%\laostar\ procedure for solving problems expressed in DTPDDL. 

%%

%Have a good understanding of the contents and function of rooms, as
%well as the linguistic referrents to rooms, widgets, and their visual
%qualities.

%Having confidence in its beliefs about the linguistic terms for
%spaces, and the function of those spaces.



%sometimes difficult to predict, with exogenous events, such as the
%changing of the objective, 





%the speed and scalability of software for sequential planning in
%deterministic




%% We developed a first-order declarative language, called DTPDDL, for
%% describing domains of planning problems that correspond to POMDPs.  To
%% accompany that language we have implemented an information-state
%% \laostar\ procedure for solving problems expressed in DTPDDL. 

%% We have extended MAPSIM to parse and simulate DTPDDL problems, and
%% modified \fastdownward\ so that it can find useful sequential executions
%% given DTPDDL models of the problem at hand.




%%  called DTPDDL, along
%% the lines of PDDL for the partially observable case, an \laostar\
%% solution procedure, a determinisation of the DTPDDL problem in MAPL,
%% and modify the \fastdownward\ system to find high quality sequential plans
%% .

%% We model the environment as a partially observable Markov decision
%% process. Although planning in that model is undecidable in
%% general~\cite{}, an optimal finite-horizon plan corresponds to a
%% contingent plan, that is a function mapping action-observation
%% histories to actions.

%% our continual planner is reactive, replanning whenever the underlying
%% domain and problem models change. For example, replanning occurs if
%% the motivational component alters the objectives, and if an assumed
%% outcome of a sensing action is not realised.

%% brittle sensing model, 

%% the evaluation of a fluent at a state is either known. Moreover, there
%% is a sensing process that run-time variables  after-which 

%% For $\prop \in \state$ we say proposition $\prop$ characterises state
%% $\state$. There is always a unique starting state $\state_0$. The goal
%% $\goal$ is a set of propositions, and we say that state $\state$ is a
%% goal state iff $\goal \subseteq \state$.

%%  To keep this exposition
%% simple, for any two distinct actions $\stochAction_i \neq
%% \stochAction_j$, if outcome $\detAct$ is a possibility for
%% $\stochAction_i$ then it cannot also be a possibility for
%% $\stochAction_j$ -- i.e., if $\detAct \in
%% \detActions(\stochAction_i)$ then $\detAct \not\in
%% \detActions(\stochAction_j)$.


%% The solution to a probabilistic planning problem is a contingency
%% plan. This consists of an assignment of actions to states at each
%% discrete timestep up to the planning horizon $n$. The optimal
%% contingency plan is one which prescribes actions to states that
%% maximise the probability that the goal is achieved within $n$ steps
%% from the starting state $\state_0$. For the purposes of this paper
%% we say a plan fails, i.e. achieves the goal with probability $0$, in
%% situations where it does not prescribe an action. Computing the
%% optimal plan for a problem is computationally intractable, and an
%% important direction for research in the field is to develop
%% heuristic mechanisms for generating small linear plans
%% quickly~\cite{littman:etal:98}
