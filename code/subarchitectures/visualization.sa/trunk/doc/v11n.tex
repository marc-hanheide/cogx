% vim:ft=tex:sw=3:ts=8:et:fileencoding=utf-8

\section{Installation}

To get the visualization from the svn, edit the SVN externals in the root
of your CogX system directory:

\begin{Verbatim}[fontsize=\scriptsize,gobble=3]
   $ svn pe svn:externals .
\end{Verbatim}

Add the following line, save and do an svn update:

\begin{Verbatim}[fontsize=\scriptsize,gobble=3]
   ^/code/subarchitectures/visualization.sa/trunk subarchitectures/visualization.sa
\end{Verbatim}

The subarchitecture depends on some external libraries. If they are not
installed, the system may still compile, but some of the features won't be
available. The dependencies can be installed with the script in
visualization.sa/bin: 

\begin{Verbatim}[fontsize=\scriptsize,gobble=3]
   sudo visualization.sa/bin/ubuntu-install-deps.sh
\end{Verbatim}

The new subarchitecture must be added to the main CMakeLists.txt file of the 
project. It should be added before any subarchitectures that use visualization
so that the dependent libraries will build correctly. Add the following
block before the first subarchitecture in CMakeLists.txt:

\begin{Verbatim}[fontsize=\scriptsize,gobble=3]
   include(subarchitectures/visualization.sa/cmake/v11n.cmake)
\end{Verbatim}

Finally run \code{ccmake} and set \code{BUILD\_SA\_V11N} to \code{ON} to enable
the visualization subarchitecture.


\section{General operation}

The library provides a display server that displays objects created by its
clients.  The objects that are on the display server are grouped into views.
Initially every new object is displayed in its own view that has the same name
as the object.
\todo{(TODO: It will be posible to define additional views that display multiple objects)}
The display server creates rendering contexts in which the views are displayed.
There are three types of contexts: 2D, OpenGL and HTML.
% (TODO: additional contexts: eg. tabular)

The client that wants to display an object connects to the server and sends the
serialized object to the server. Each object has a name. If an object with the
same name already exists on the server it is updated with the new data. Some
objects are composed of multiple parts where each part can be manipulated
independently.

Currently the following types of objects are supported: images, HTML documents,
SVG graphics and OpenGL objects defined with LuaGl script or as TomGine
render-models. 

The display server can also display simple user interface elements. If a view
is active (visible) all the UI element associated with the view are also
displayed. When the state of a UI element changes, eg.  because of a muse
click, the client that owns the UI element is notified and it receives the new
state of the object.

(2010-06-09: at this time only two types of UI elements are provided: a button
and a checkbox).

\section{Sending data to the server}

The C++ client library provides a client interface that enables the interaction
with the server. It lets the programmer send objects to be dispayed to the
server and receive notifications about the events emited by the user interface
that was built on the server by the client component.
\todo{TODO: add interfaces for Java and Python}

For simple objects the server can be used without linking to the display client
because the file system as a communication channel. To do this a CAST
component FileMonitor has to be added to the cast file and configured to
monitor certain files that are created by other componetns.

\subsection{File system monitor}

The File system monitor is a CAST component that monitors the file system for
changes. Usually it observes a directory on the disk and waits for files in that
directory that have been closed after they were modified. If the modified file
is one of the files the monitor is waiting for, it reads the file processes it
and sends it to the server. A single component is able to monitor multiple files
in multiple directories.

The files that the monitor can process are images, SVG files, dot files created
for GraphWiz, LuaGl scripts, HTML chunks.

Here is an example CAST setup of a DisplayServer and a FileMonitorDC. Note that
all text after fileviewer.cli is on the same line.

\begin{Verbatim}[fontsize=\scriptsize,gobble=3]
   HOST localhost
   SUBARCHITECTURE visualization.sa 
   CPP WM SubarchitectureWorkingMemory
   CPP TM AlwaysPositiveTaskManager

   CPP MG display.srv DisplayServer

   CPP MG fileviewer.cli FileMonitorDC --displayserver "display.srv"
      --setvars "dir=subarchitectures/visualization.sa/src/c++/clients/fileviewer/test/xdata"
      --monitor "
         CurrentPlan=dot:%(dir)/{*.dot}
         AltPlan=twopi:%(dir)/{*.dot}
         ImageFile=%(dir)/{*.png;*.jpg;*.pnm;*.xpm}
         SvgFile=%(dir)/{*.svg}
         %(dir)/{*}"
         --log true --debug true
\end{Verbatim}

The file monitor first connects to the server using the \code{--displayserver}
parameter. If the server is not present, a display client will operate normally
except it won't be able to display its results.

The parameter \code{--setvars} sets some variables that are used later when
the component configures the monitored files. The format of the parameter
string is \code{"var1=value1 var2=value2 ..."}. Later in the parameter
\code{--monitor} a variable can be inserted with \code{\%(var1)}.

The most important and complex part is the definition of the files to be
monitored, the \code{--monitor} parameter. It is composed of multiple entries
that are separated by spaces. Each entry defines the directory where the files
are, the names of the files to monitor, the converters for the files and the
name of the object that will be created from the files. The general format of
each entry is:
\begin{Verbatim}[fontsize=\scriptsize,gobble=3]
   ObjectName=converter:directory/{filemasks}
\end{Verbatim}

If the converter is omitted, one of the known converters will be selected.
If the object name is omitted, the name of the object will be the filename.

We can now decode the meanings of the above definitions:

\begin{Verbatim}[fontsize=\scriptsize,gobble=3]
   CurrentPlan=dot:%(dir)/{*.dot}
   AltPlan=twopi:%(dir)/{*.dot}
\end{Verbatim}

Every new dot file in the directory will be converted with two converters: dot
and twopi. The files converted with dot will be displayed on the server as
CurrentPlan while the files converted with twopi well be displayed as AltPlan.
The file monitor also knows the following converters for dot: neato, circo and
fdp.

\begin{Verbatim}[fontsize=\scriptsize,gobble=3]
   ImageFile=%(dir)/{*.png;*.jpg;*.pnm;*.xpm}
   SvgFile=%(dir)/{*.svg}
\end{Verbatim}

Images and SVG files can be sent without conversion if the file extension is
one of the known ones. If a SVG file is stored in a file with an extension
other than \code{.svg}, the converter must be specified although no conversion
will be performed:

\begin{Verbatim}[fontsize=\scriptsize,gobble=3]
   SvgFile=svg:%(dir)/{*.xml}
\end{Verbatim}

The final entry will pick up any files that change in the directory. If the
type of the file (deduced from the extension) is one of the known types, the
file will be sent to the server. The name of the object on the server will be
the name of the file.

\begin{Verbatim}[fontsize=\scriptsize,gobble=3]
   %(dir)/{*}"
\end{Verbatim}

The converters {\em luagl}, {\em html} and {\em htmlhead} send the unprocessed
contents of the monitored file to the server.

\subsection{A Passive Display Client}

The library provides two types of clients: a passive client that only sends
data to the server and an active client that interacts with the server using UI
elements. We will first show how to create a passive client and then convert it
to an active client. We will use the VideoViewer component as an example. Only
the relevant parts of the code will be shown. Full code is in the VideoViewer
component in vision.sa.

Before we can start we need to prepare the project file to use the display server.
For every component that is defined in CMakeLists.txt we add the following:

\begin{Verbatim}[fontsize=\scriptsize,gobble=3]
   include_directories(${VISUALIZATION_INCLUDE_DIRS})
   link_cast_component(${CAST_COMPONENT_NAME} Video ${VISUALIZATION_LIBRARIES})
\end{Verbatim}

If the component uses OpenCV's IplImage we can let the client convert the image
before it is sent to the server. To do this we need to enable the client's OpenCV
interface with

\begin{Verbatim}[fontsize=\scriptsize,gobble=3]
   add_definitions(-DFEAT_VISUALIZATION_OPENCV)
\end{Verbatim}

Now the component is ready to interact with the display server. The component
will communicate with the server through a display client:

\begin{Verbatim}[fontsize=\scriptsize,gobble=3]
   class VideoViewer : public ManagedComponent, public VideoClient
   {
   private:
   #ifdef FEAT_VISUALIZATION
     cogx::display::CDisplayClient m_display;
   #endif
   };
\end{Verbatim}

Note that all the code related to visualization.sa is enclosed in conditional
blocks. The variable \code{FEAT\_VISUALIZATION} is defined when
visualization.sa is being built. If the visualization.sa is not used in the
project, the components that depend on it will continue to work, but they won't
display their data unless an alternative is provided (see VideoViewer.cpp which
uses OpenCV when visualization.sa is not used).

The client must connect to the server and for that it needs the parameters of the
connection (the parameter \code{--displaysever}):

\begin{Verbatim}[fontsize=\scriptsize,gobble=3]
   void VideoViewer::configure(const map<string,string> & _config)
   {
   #ifdef FEAT_VISUALIZATION
     m_display.configureDisplayClient(_config);
   #endif
   }
\end{Verbatim}

The connection between the client and the server is established in the component's
\code{start()} method:
\begin{Verbatim}[fontsize=\scriptsize,gobble=3]
   void VideoViewer::start()
   {
   #ifdef FEAT_VISUALIZATION
      m_display.connectIceClient(*this);
   #endif
   }
\end{Verbatim}

When a new image is available, the VideoServer sends the image to the VideoViewer which
passes the image to the DisplayServer:

\begin{Verbatim}[fontsize=\scriptsize,gobble=3]
   void VideoViewer::receiveImages(const std::vector<Video::Image>& images)
   {
   #ifdef FEAT_VISUALIZATION
     m_display.setImage(getComponentID(), images[0]);
   #endif
   }
\end{Verbatim}

Here \code{getComponentID()} is used for the name of the object since the component
is displaying only one object. Any other string can be used instead.


\section{Interaction with the display server}

An active display client is needed when the component wants to present some UI
elements with which it can be configured. The client receives notifications and
requests related to these elements from the display server. The component must
define two callback functions that will handle the events sent by the server.

\begin{Verbatim}[fontsize=\scriptsize,gobble=3]
   class VideoViewer : public ManagedComponent, public VideoClient
   {
   private:
   #ifdef FEAT_VISUALIZATION
     cogx::display::CActiveDisplayClient<VideoViewer> m_display;
     void handleGuiEvent(const Visualization::TEvent &event);
     std::string getControlState(const std::string& ctrlId);
   #endif
   };
\end{Verbatim}

The event path from the server to the client must be initiated. This is done right
after the component connects to the client. After the connections are established,
the callback functions are registered:

\begin{Verbatim}[fontsize=\scriptsize,gobble=3]
   void VideoViewer::start()
   {
   #ifdef FEAT_VISUALIZATION
     m_display.connectIceClient(*this);
     m_display.installEventReceiver();
     m_display.setEventCallback(this, &VideoViewer::handleGuiEvent);
     m_display.setStateQueryCallback(this, &VideoViewer::getControlState);
   #endif
   }
\end{Verbatim}

The flow of images from the VideoServer to the VideoViewer can be stopped at
any time. To let the user control the flow the component has to add a UI
element to the display server.  An appropriate type of UI element for this
purpose is a chekcbox:

\begin{Verbatim}[fontsize=\scriptsize,gobble=3]
   void VideoViewer::start()
   {
   #ifdef FEAT_VISUALIZATION
     m_display.addCheckBox(getComponentID(), "toggle.viewer.running", "&Streaming");
   #endif
   }
\end{Verbatim}

The first parameter is the name of the {\em view} that will display the UI
element when it is active, not the name of object. This means that the UI
element will be displayed only when an object is displayed in its default view.
When an object is displayed in another view, the UI element won't be available.
\todo{TODO: user-defined views will be able to specify which UI elements to show}

The \code{handleGuiEvent()} callback function will be called every time the
UI element emits a known event. For the chekcbox element we need to respond
to the \code{evCheckBoxChange} event:

\begin{Verbatim}[fontsize=\scriptsize,gobble=3]
   #ifdef FEAT_VISUALIZATION
   void VideoViewer::handleGuiEvent(const Visualization::TEvent &event)
   {
     if (event.type == Visualization::evCheckBoxChange) {
       if (event.sourceId == "toggle.viewer.running") {
         bool newrcv = (event.data != "0");
         if (newrcv != receiving) {
           if(receiving) {
             videoServer->stopReceiveImages(getComponentID());
           }
           else {
             videoServer->startReceiveImages(getComponentID(), camIds, 0, 0);
           }
           receiving = !receiving;
         }
       }
     }
   }
   #endif
\end{Verbatim}

Note that the data for UI elements passed between the client and the server is
serialized into strings. The checkbox element knows only 2 values: \code{"0"}
and everything else.
\todo{TODO: Add more element types, define values}

A UI element may be displayed in multiple views. When a UI element is first
displayed it must be put in a state that reflects the current state of the
component.  Since a UI element may be created at any time it wouldn't be enough
to set the initial state of the element at creation time. Instead, we define
the \code{getControlState()} callback function:

\begin{Verbatim}[fontsize=\scriptsize,gobble=3]
   #ifdef FEAT_VISUALIZATION
   std::string VideoViewer::getControlState(const std::string& ctrlId)
   {
     if (ctrlId == "toggle.viewer.running") {
       return receiving ? "2" : "0";
     }
     return "";
   }
   #endif
\end{Verbatim}

When the function returns \code{""} the display server won't change the state
of the UI element.

\section{Displaying objects}

The server can display various types of objects. Initially it was created to
display images, but with time other objects (supported by Qt) were added:
SVG images, HTML documents and OpenGL scenes.

The objects can be geometrically transformed on the server when they are
displayed. A client side interface is provided to set display transformations
of objects and their parts.

\subsection{Images}

The display client can send various types of images to the server: raw images, 
\code{Video::Image}, OpenCV's \code{IplImage} and encoded images that are
supported by Qt. An \code{IplImage} can only be sent if OpenCV was enabled
during compilation.
The client-side interface for sending images is:

\begin{Verbatim}[fontsize=\scriptsize,gobble=3]
   void setImage(const std::string& id, int width, int height, int channels,
         const std::vector<unsigned char>& data);
   void setImage(const std::string& id, const Video::Image& image); 
   void setImage(const std::string& id, const std::vector<unsigned char>& data,
         const std::string &format=""); 
   #ifdef FEAT_VISUALIZATION_OPENCV
      void setImage(const std::string& id, const IplImage* pImage); 
   #endif
\end{Verbatim}

Raw images can have 1 or 3 channels and the specified image size must match the
size of \code{data}. The supported formats of the compressed image are: bmp,
gif, jpeg, jpg, png, pbm, pgm, ppm, tiff, xbm, xpm. If the format is not
specified, Qt will try to deduce the type from \code{data}. An \code{IplImage}
is first converted to a contiguous array of bytes and sent to the server as a
raw image.
 
Every image has an id which uniquely identifies the image on the server. This
way the image (as any other object) can be replaced on the server.

Images can only be displayed in 2D contexts. Image objects can be transformed
by defining a $3 \times 3$ transformation matrix that is applied before
the image is drawn.

\subsection{SVG objects}

SVG objects are sent to the server as XML documents which are then interpreted
by the Qt SVG subsystem. Every SVG object can have multiple parts where each
part is defined in a separate XML document.
\todo{TODO: the interface to define the order of drawing for objects and parts
is missing}
The client-side interface for defining SVG objects is:

\begin{Verbatim}[fontsize=\scriptsize,gobble=3]
   void setObject(const std::string& id, const std::string& partId,
         const std::string& xmlData); 
\end{Verbatim}

The function is named \code{setObject} because initially it was planned that
the server will support multiple types of objects. Currently the server
supports only SVG and doesn't check if the SVG file is valid.

SVG objects can only be displayed in 2D contexts. Every part of an object can
be transformed by defining a $3 \times 3$ transformation matrix that is applied
before the part is drawn.

\subsection{HTML documents}

The display server renders HTML documents with Qt-WebKit. It provides the
sturcture of the HTML document while display clients provide chunks of
HTML code that goes into the head or the body of the document.
The client-side interface for defining HTML document chunks is:

\begin{Verbatim}[fontsize=\scriptsize,gobble=3]
   void setHtml(const std::string& id, const std::string& partId,
         const std::string& htmlData); 
   void setHtmlHead(const std::string& id, const std::string& partId,
         const std::string& htmlData); 
\end{Verbatim}

Chunks for the head and the body part of a document are independent. This means
that a chunk in the head may have the same partId as a chunk in the body of
a document. When a part is removed from a document, the chunks for the head
and the body with the requested partId are removed.
\todo{TODO: ICE interface for removing (parts of) objects is still missing}

\subsection{OpenGL objects}

OpenGL objects can be rendered in two ways. The more flexible option is the use
of the GlLuaScript subsystem where normal OpenGL commands are written to a Lua
script and sent to the server. The other option is to build TomGine models on
the client, and send them to the server using boost serialization. 


\subsubsection{LuaGlScript objects}

A LuaGlScript objects are multipart objects where each part of the object is a
Lua script engine into which Lua scripts are loaded. The scripts are sent to
the server with a display client function:

\begin{Verbatim}[fontsize=\scriptsize,gobble=3]
   void setLuaGlObject(const std::string& id, const std::string& partId,
         const std::string& script);
\end{Verbatim}

Each script has a method called \code{render()} that is called every time the
scene needs to be rendered.  This method then calls various OpenGL functions to
render the scene. Most of the OpenGL functions should be available (textures
are not supported at this time). An example of a script that renders a colored
spiral could look like this (the tables \code{tabcos} and \code{tabsin} were
defined in another part of the same script):

\begin{Verbatim}[fontsize=\scriptsize,gobble=3]
   N=36
   function render()
      glBegin(GL_LINES)
      for i=1,10000 do
         glColor(i%23/23.0,i%17/17.0,i%11/11.0)
         glVertex(tabcos[i%N+1], tabsin[i%N+1], -i/100.0)
      end
      glEnd()
   end
\end{Verbatim}

A script is loaded into the Lua script engine every time the server receives it
from a client. This way the appearance of the object can be changed during its
lifetime. Complex objects can use display lists to make rendering simpler and 
faster. Lists are created for each OpenGL context separately so if an object
is visible in multiple contexts, each context will have a copy of every
display list defined by the object.
\todo{VERIFY: diasplay lists are local to every LuaGlScript part}
The following commands are available for display list management:

\begin{Verbatim}[fontsize=\scriptsize,gobble=3]
   DispList:newList(name)
   DispList:endList()
   DispList:delete(name)
   DispList:exists(name)
   DispList:draw(name)
   DispList:setDirty(name)
   DispList:getDirty(name_table)
\end{Verbatim}

In the following example we create a function that creates a display list named
{\em myobject}. The function \code{initlists()} calls it when {\em myobject}
doesn't exist in the current context. Finally \code{render()} intializes the
lists before doing the rest of the rendering.

\begin{Verbatim}[fontsize=\scriptsize,gobble=3]
   function make_the_thing()
      DispList:newList('myobject')
        glBegin(GL_TRIANGLES)
        glVertex(0.1, 0.1, 0.1)
        glVertex(0.3, 0.1, 0.2)
        glVertex(0.3, 0.4, 0.3)
        glEnd()
      DispList:endList()
   end

   function initlists()
      local dirty = DispList:getDirty( {'myobject'} )
      if dirty['myobject'] then
         make_the_thing()
      end
   end

   function render()
      initlists()
      DispList:draw('myobject')
      glPushMatrix()
         glTranslate(0.5,0.5,0.0)
         DispList:draw('myobject')
      glPopMatrix()
   end
\end{Verbatim}

With this approach the lists are created only once in an OpenGL context and
only when a context is active. When the code is loaded into the Lua script
engine, the script is locked and the rendering of the object is not possible.
The OpenGL context might not even be available at the time or a wrong one might
be active. That is why all OpenGL calls should be enclosed in functions that
are accessible only from \code{render()}.

It is also possible for a programmer to change only a part of the object by
sending a new definition for \code{make\_the\_thing()}:

\begin{Verbatim}[fontsize=\scriptsize,gobble=3]
   function make_the_thing()
      DispList:create('myobject')
        glBegin(GL_QUADS)
        glVertex(0.1, 0.1, 0.1)
        glVertex(0.3, 0.1, 0.2)
        glVertex(0.3, 0.4, 0.3)
        glVertex(0.0, 0.5, 0.4)
        glEnd()
      DispList:endList()
   end
   DispList:setDirty('myobject')
\end{Verbatim}

The function \code{setDirty()} tells the display list manager that the object
was modified. The renderer will detect the change the next time
\code{getDirty()} is called, in this case from \code{initlists()}.
\code{getDirty(namelist)} returns a table with keys that represent the names of
the objects from {\em namelist} that need to be recreated. It will contain the
names of the objects that were not yet created and the objects that were marked
as dirty. The list of dirty objects is reset after every call to
\code{getDirty()}.

The module \code{StdModel} defines the following functions that create simple shapes:

\begin{Verbatim}[fontsize=\scriptsize,gobble=3]
   StdModel:box(sx, sy, sz)
   StdModel:cylinder(rx, ry, np)
\end{Verbatim}

The box and the cylinder are created with the center in $(0, 0, 0)$. The
parameters for the box define the size of the box. The cylinder has an elliptic
shape with the given radii and is approximated with {\em np} sides.

