

From an automated planning perspective, the problem of practical
mobile robot control poses important and contrary challenges.
%%
On the one hand, planning and execution monitoring must be
lightweight, robust, timely, and should span the lifetime of the
robot. Those processes must seamlessly accommodate exogenous events,
changing objectives, and the underlying unpredictability of the
environment.
%%
On the other hand, robot planning should perform computationally
expensive reasoning about contingencies, and possible revisions of
subjective belief according to quantitatively modelled uncertainty in
acting and sensing. 

In this paper we address these challenges, developing a continual
planner that switches between sequential and decision-theoretic
planning. Given a POMDP model of the environment, sequential planning
is used to compute an initial deterministic sequential plan and
complementary runtime evolution of the decision process. That plan is
executed until a validation of an assumption about a runtime
proposition is requested.


The sequential mode of our system always schedules decision-theoretic
planning before executing actions that are not applicable with
certainty. This is vastly inefficient if the utility of that plan is
not dependent on the actions successful execution. This is the case
where a plan is conformant. Therefore, in the future we must develop a
cheap (heuristic) procedure to evaluate the utility of sensing at
steps in sequential plan.


One theoretical criticism of switching continual planning concerns
interleaved sequential and decision-theoretic sessions failing to make
any progress towards the POMDP objectives. For example, in the worst
case we could have each sequential session producing an identical
plan, and each decision-theoretic session rejecting it without further
sensing. Although this has not been an issue in our work so far, it
must be dealt with rigorously moving forward. We propose a good way of
mitigation this problem is by using a {\em motivational} component
that maintains a {\em dynamic} reward model for sensing.






The switching continual planning system we have described serves as
the underlying planner for CogX.




the effects of a {\em sense} schema are perceptual, whereas the
effects of an {\em operator} schema are over state propositions.




posted by a motivational component of the underlying robotic
architecture. 

contingent sensory plans that are tailored to current
objectives.

In this paper we present an approach to continual planing that uses
two planning systems. The first 

 to a distinct class of
challenges. We suppose 
%%
The underlying environment is modelled as a POMDP. We use the fast
classical satisficing system FastDownward to find a deterministic
sequential plan and complementary runtime evolution of that
process. This corresponds to a generalisation of replanning in
probabilistic planning to problems with partial observability.

Interaction between the sequential planner and execution proceeds
more-or-less analogously to popular replanning approaches

Addressing these challenges in a monolithic framework, we present a
{\em switching} continual planner, that uses the fast sequential
satisficing procedure FastDownward to perform net-benefit
%%
makes reasonable assumptions about the evolution of the runtime state
given a POMDP model of the environment. 

contingent sensory plans that are tailored to current
objectives.


The decision-theoretic planner is able to tailor sensory processing on
a robot platform to the current objective, while FastDownward  quickly 


In this paper we develop a continual planing system that uses two
planning systems. The first, is a state-of-the-art domain independent
planner for deterministic problems. The second is a information-state
contingency planning the information-state space of 


Continual planning is a powerful technique that goes some way to
addressing those challenges. That approach interleaves planning and
execution, deliberately postponing planning for contingencies unless
they eventuate during execution. 


computing a single sequential plan and
eventuality



To these challenges it seems a continual planning approach is best, 

where the runtime state evolves during plan execution 


interleaved planning and execution. 

the latter is able to tailor sensory processing on a robot platform,
in order that it.

{\em ad-hoc} 

, reason about degrees of belief and uncertainty about the world, 

probabilistic sequential decision making in practical sized problems
is intractable.

quantitative probabilistic models of the perception and action .
