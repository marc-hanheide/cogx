



We describe, \pcogx, the planning component in a robotic system that
continuously deliberates in a stochastic dynamic environment in order
to achieve objectives set by the user, and acquire knowledge about its
surroundings.
%%

Have a good understanding of the contents and function of rooms, as
well as the linguistic referrents to rooms, widgets, and their visual
qualities.

Having confidence in its beliefs about the linguistic terms for
spaces, and the function of those spaces.



sometimes difficult to predict, with exogenous events, such as the
changing of the objective, 





the speed and scalability of software for sequential planning in
deterministic




We developed a first-order declarative language, called DTPDDL, for
describing domains of planning problems that correspond to POMDPs.  To
accompany that language we have implemented an information-state
\laostar\ procedure for solving problems expressed in DTPDDL. 

We have extended MAPSIM to parse and simulate DTPDDL problems, and
modified \fastdownward\ so that it can find useful sequential executions
given DTPDDL models of the problem at hand.




 called DTPDDL, along
the lines of PDDL for the partially observable case, an \laostar\
solution procedure, a determinisation of the DTPDDL problem in MAPL,
and modify the \fastdownward\ system to find high quality sequential plans
.

We model the environment as a partially observable Markov decision
process. Although planning in that model is undecidable in
general~\cite{}, an optimal finite-horizon plan corresponds to a
contingent plan, that is a function mapping action-observation
histories to actions.

our continual planner is reactive, replanning whenever the underlying
domain and problem models change. For example, replanning occurs if
the motivational component alters the objectives, and if an assumed
outcome of a sensing action is not realised.

brittle sensing model, 

the evaluation of a fluent at a state is either known. Moreover, there
is a sensing process that run-time variables  after-which 

%% For $\prop \in \state$ we say proposition $\prop$ characterises state
%% $\state$. There is always a unique starting state $\state_0$. The goal
%% $\goal$ is a set of propositions, and we say that state $\state$ is a
%% goal state iff $\goal \subseteq \state$.

%%  To keep this exposition
%% simple, for any two distinct actions $\stochAction_i \neq
%% \stochAction_j$, if outcome $\detAct$ is a possibility for
%% $\stochAction_i$ then it cannot also be a possibility for
%% $\stochAction_j$ -- i.e., if $\detAct \in
%% \detActions(\stochAction_i)$ then $\detAct \not\in
%% \detActions(\stochAction_j)$.


%% The solution to a probabilistic planning problem is a contingency
%% plan. This consists of an assignment of actions to states at each
%% discrete timestep up to the planning horizon $n$. The optimal
%% contingency plan is one which prescribes actions to states that
%% maximise the probability that the goal is achieved within $n$ steps
%% from the starting state $\state_0$. For the purposes of this paper
%% we say a plan fails, i.e. achieves the goal with probability $0$, in
%% situations where it does not prescribe an action. Computing the
%% optimal plan for a problem is computationally intractable, and an
%% important direction for research in the field is to develop
%% heuristic mechanisms for generating small linear plans
%% quickly~\cite{littman:etal:98}

An outline of the paper is as follows. We describe POMDPs with a
propositionally factored representation of states and observations,
and describe how to evaluate plans when they correspond to
finite-state controllers.
