

We describe our approach that switches between sequential and
contingent sessions. As a continual planning approach, it proceeds by
interleaving planning and execution in a deterministic-action POMDP
described in DTPDDL. During a sequential session, planning is
performed by a ``classical'' system,\footnote{That is, a planner
designed to solve fully observable deterministic tasks.}  and
execution proceeds according to the {\em trace} computed by that
system. Taking the form of a classical plan, the trace specifies a
sequence of POMDP actions that achieve the agent's objectives in a
deterministic approximation, i.e., {\em determinisation}, of the
problem at hand. More precisely, the trace is an interleaved sequence
of POMDP actions and {\em assumptive} actions. The latter correspond
to assumptions the planner makes about the truth value of propositions
-- e.g. that a box of cornflakes is located in the kitchen at the
third plan step. They are called {\em applicability} assumptions if
the trace includes an action $\action$ that is not applicable with
probability $1$ at the belief-state
\bstate\ that the system is projected to be in when \action\ is
scheduled for execution, i.e., $\exists\state\in\states\;
\bstate(\state) > 0$ and $\poss(\action)\not\subseteq\state$. By
scheduling
\action,  the serial planner makes an assumption about the
observability of the precondition $\poss(\action)$.

Our approach always begins with a sequential session. Non-assumptive
actions from the trace are executed in sequence until the
applicability of the next scheduled action is uncertain. We denote
that action \switchAction.  A contingent session then begins, that
tailors sensory processing by the agent to determine whether the
assumptions made in the trace hold. 
%%
We add {\em disconfirm} and {\em confirm} actions to the model, so
that session is encouraged to judge the assumptions made in the
trace. On execution of one of those actions, control is returned to
the sequential session that continues at the {\em underlying}
belief-state.
%%
%% For each assumption we add a {\em
%% disconfirm} action to the POMDP whose execution is rewarding
%% (resp. costly) if the assumption is false (resp. true). We also add
%% one {\em confirm} action that is rewarding (resp. costly) to execute
%% if $\poss(\switchAction)$ is true (resp. false). If execution of the
%% contingent plan applies a disconfirm action, then a new sequential
%% session begins at the {\em underlying} belief-state. If the confirm
%% action is executed, the sequential session is resumed, and
%% \switchAction\ is executed. Otherwise, control rests in the continual
%% session.
%%
Because contingent planning is only practical in relatively small
POMDPs, contingent sessions plan in an abstract decision process
determined by the current trace. This abstraction is constructed by
first omitting all propositions that do not feature in the trace, and
by then iteratively refining the model while the result is of a
practicable size.

Finally, whether proceeding in a sequential or contingent session, our
continual planner maintains a factored representation of successive
belief-states, by performing belief revision according to
Eq~\ref{eq:revision}. Moreover, our internal representation of the
underlying belief-state corresponds more-or-less directly to an
$(\pp{:init})$ declaration. Our approach uses that distribution: (1)
as the source of candidate determinisations for sequential planning,
(2) to determine when to switch, and (3) as a mechanism to guide
construction of an abstract process for contingent sessions.

%% PLEASE STICK A PARAGRAPH HERE, (SENSIBLY) REQUESTED BY MORITZ, ABOUT
%% HOW WE DO BELIEF REVISION ACROSS SWITCHING.

%% \Omit{
%%   Therefore, although our current implementation does not
%%   explicitly support it, our approach generalises to POMDPs with
%%   stochastic action models.
%% }

\subsection{Sequential Planning in POMDPs}

A sequential session uses a classical planner to compute a trace. The
latter encapsulates assumptions about: (1) the true underlying state,
(2) how execution will progress, and (3) the possibility of the agent
eventually holding strong beliefs about the truth values of specific
state propositions. Here we describe the deterministic planning
problem, derived from the DTPDDL model, that admits plans which
correspond to traces.

In a deterministic-action POMDP all the uncertainty in state and
action is expressed in the $(\pp{:init})$ declaration. Our approach
uses the structure of that, as it occurs in the problem description,
to define a set of state-assumptions available to sequential planning.
%%
Writing \#\ if the value of a proposition is unspecified, for
$\pp{DORA}$ we have the following assumptions:

\small
\begin{tabular}{cccc}
\hline
Probability & (is-in DORA)  & (is-in box)  & (is-in cup) \\
\hline
.24 & kitchen & office & office \\
.06 & kitchen & kitchen & office \\
.56 & kitchen & office & kitchen \\
.14 & kitchen & kitchen & kitchen \\
.7 & kitchen & \# &  kitchen\\
.3 & kitchen & \# & office \\
.8 & kitchen & office & \# \\
.2 & kitchen & kitchen & \# \\
1.0 & kitchen & \# & \# \\
\hline
\end{tabular}
\normalsize

\noindent An assumption corresponds to a {\em
relaxed} visitation of the root term of $(\pp{:init})$. In this
relaxed case, a conjunctive term is visited iff its atomic subterms
are visited, and zero or more of its immediate probabilistic subterms
are visited. The starting state, $\state_0$, for sequential planning
is an {\em abstract} state that captures the unique assumption which
has probability $1$. In $\pp{DORA}$, that is:

\[
\begin{array}{l}
\state_0 \equiv \{(=(\pp{is-in}~\pp{DORA})~\pp{kitchen}),\\
\;\;(=(\pp{is-in}~\pp{box})~\#), (=(\pp{is-in}~\pp{cup})~\#)\}.
\end{array}
\]

In order to make assumptions available to the sequential planner, we
add one {\em assumptive} action $\assumptiveS{i}$ to the problem for
each element, $\prob_i (T_i)$, of each probabilistic term from
$(\pp{:init})$. The physics of these actions is as
follows. $\assumptiveS{i}$ can be executed if no $\assumptiveS{j}$,
$j \neq i$, has been executed from the same probabilistic term, and,
either $(\pp{probabilistic}~..\prob_i~(T_i)..)$ is in the root
conjunct, or it occurs in $T_k$ for some executed
$\assumptiveS{k}$. Also, an assumption about a proposition
$\prop$ can only be made once, and must be sequenced before any POMDP
actions that have $\prop$ in their precondition or effect.
%%
In other words, an assumptive action
must be applied to make a truth assignment to \prop, and then that
truth assignment can be modified by the DTPDDL domain operators.
%%
%%
Executing $\assumptiveS{i}$ in $\state$ effects a transition to a
successor state $\state^{T_i}$ with probability $\prob_i$, and
$\state^\bot$ with probability $1 - \prob_i$. Here, $\state^{T_i}$ is
the union of $\state$ with atomic terms from $T_i$ with the
proposition denoting the undefined fact is deleted. For example, if
the action adds $(=(\pp{is-in}~\pp{box})~\pp{office})$, then it must
delete $(=(\pp{is-in}~\pp{box})~\#)$. State $\state^\bot$ is an added
sink, where $\forall\action$ $\reward(\state^\bot, \action) = 0$ and
$\transProb(\state^\bot, \action, \state^\bot) = 1$.
%%
Lastly, we have that if $\action$ has a precondition that is not true
with probability $\theta$, a given threshold, in the underlying belief
$\bstate$, then immediately before it is executed blindly the system
switches to a contingent session (described below).


%% For sequential planning the operators from the POMDP model are
%% available with semantics that accommodate the above abstraction as
%% follows: In the deterministic model a proposition $\prop$ can be
%% thought to have a ternary interpretation at a state \state, as either
%% {\em true}, written $\prop\in\state$, {\em false}, written
%% $\prop\not\in\state$, or {\em unspecified}, with a slight abuse of
%% notation written $\prop\#\in\state$. For example, in $\pp{DORA}$
%% $(=(\pp{is-in}~\pp{box})\pp{office})\#\in\state_0$. For POMDP actions
%% $\action\in\actions$ in the deterministic model, if
%% $\prop \in \poss(\action)$, or if $\prop$ is the subject of a positive
%% or negative effect of $\action$, then $\action$ is not applicable
%% in \state\ if $\prop\#\in\state$. 





%% Addressing now the switching semantics of action execution. I
%% %% In detail, we halt the sequential
%% session at \action\ if $\exists
%% \state$ s.t. $\bstate(\state)>0$, 
%% $\prop\in\poss(\action)$ , and $\prop\not\in\state$. According to the
%% semantics of action execution there must be a $\state'$ s.t.
%% $\bstate(\state')>0$ and $\prop\in\state'$. Otherwise, \action\ must
%% not have been scheduled at \bstate.


%% then $\reward(\state,\action)=10$, and
%% otherwise if $\prop\not\in\state'$ then
%% $\reward(\state',\action)=-10$.


We now describe the optimisation criteria given to the classical
planner. Where $\prob_i$ is the probability that the $i^{th}$
sequenced action $\action_i$ does not transition to $\state^\bot$, we
define the value of a trace
$\state_0, \action_0, \state_1, \action_1,.., \state_N$ from the
deterministic model to equal:

\begin{equation}\label{eq:tracevalue}
V(\state_0, \action_0, \state_1, \action_1,.., \state_N) =  \prod_{i=1..N-1} \prob_i \sum_{i=1..N-1} \reward(\state_i, \action_i)
\end{equation}

\noindent The optimal trace given a plan, i.e., the sequence of non-assumptive
actions from the trace, therefore has value:

\[
V^* = \max_{\prob_1, .., \prob_n} \prod_{i=1..N-1} \prob_i \sum_{i=1..N-1}
\reward(\state_i, \action_i),
\]

\noindent which is equal to the maximal 
additive contribution a trace of the plan can make to
Eq~\ref{eq:expectedvalue}.  In goal directed problems, where the only
non-zero reward is received at the first transition to a goal state,
the Eq~\ref{eq:tracevalue} criteria gives us the behaviour sought
by \system{FFR$_a$}~\cite{yoon:etal:2007}. From the perspective of
gradient-based reinforcement learning systems, such as William's
REINFORCE and GPOMDP variants for planning~\cite{olivier:doug:2009},
under reasonable assumptions the optimal trace identifies a maximal
gradient step from a uniformly random soft-max policy. Finally, it is
worth clarifying that although the above discussion has been in terms
of a trace of length $N$, there is nothing in our approach that
artificially limits the length of candidate sequential plans.


\subsection{Decision-Theoretic Planning in Abstractions}

In a contingent session our switching planner solves a sensing problem
in an abstract processes defined in terms of the assumptive actions in
the current trace, and the $(\pp{:init})$ declaration that models the
underlying belief-state.

The focus for contingent planning is on sensing, we therefore give the
process in this session an augmented reward model.  This reflects the
value of performing sensing actions in the context of the trace
proposed by the preceding sequential deliberation. First, all rewards
from the original POMDP are retained. Then, for each $\assumptiveS{i}$
action scheduled by the current trace, we have a {\em dual}
$\assumptiveDT{i}$ so that $\forall\state\in\states$:

\[
\reward(\state, \assumptiveDT{i}) = \bigg\{ \begin{array}{ll}
\$(T_i) & \pp{if}~\;\;T_i \not\subseteq \state \\
\hat\$(T_i) & \pp{otherwise} \\
\end{array}
\]

\noindent where $\$(T_i)$ (resp. $\hat\$(T_i)$) is a positive
(negative) numeric quantity which captures the utility the agent
receives for correctly (incorrectly) rejecting a false assumption. The
added dual actions are propositionally trivial -- i.e.,
$\forall\state\in\states$ $\poss(\assumptiveDT{i})
\subseteq \state$ and $\assumptiveDT{i}(\state) \equiv \state$.  
%%
In order to reduce the number of possible rejections we omit
assumptions that are not {\em active} with respect to the action,
\switchAction, whose scheduled execution switched the system into
the contingent session. An assumption $\assumptiveS{i}$ is active if
\switchAction\ is not applicable in the sequential model unless
$\assumptiveS{i}$ is scheduled in the trace prefix
to \switchAction. For example, if $\pp{DORA}$'s trace is:

\[
\begin{array}{l}
\actions^{\circ}(.8;(=(\pp{is-in}~\pp{box})\pp{office}));\\
\actions^{\circ}(.3;(=(\pp{is-in}~\pp{cup})\pp{kitchen}));\\
(\pp{look-for-object}~\pp{DORA}~\pp{box}~\pp{office});\\
(\pp{look-for-object}~\pp{DORA}~\pp{cup}~\pp{kitchen});\\
(\pp{report-is-in}~\pp{box}~\pp{office}); \\
(\pp{report-is-in}~\pp{cup}~\pp{kitchen})
\end{array}
\]

\noindent Taking the switching action \switchAction\ to be
$(\pp{look-for-object}~\pp{DORA}~\pp{box}~\pp{office})$, we have that
$\actions^{\circ}(.3;(=(\pp{is-in}~\pp{cup})\pp{kitchen}))$ is not
active, and therefore exclude
$\actions^{\bullet}(.3;(=(\pp{is-in}~\pp{cup})\pp{kitchen}))$ from the
POMDP posed to the contingent session. 

We also include another propositionally trivial action
$\actions.\poss(\switchAction)$ with the reward property:

\[
\reward(\state, \actions.\poss(\switchAction)) = \bigg\{ \begin{array}{ll}
\$(\poss(\switchAction)) & \pp{if}~\;\; \poss(\switchAction) \subseteq \state \\
\hat\$(\poss(\switchAction)) & \pp{otherwise} \\
\end{array}
\]

For the purposes of successive switching, execution of either a dual
action of the form $\assumptiveDT{i}$, or the
$\actions.\poss(\switchAction)$, returns control to a sequential
session. If a dual is executed, then the sequential planner must
replan with $\bstate_0$ equal to the underlying
belief-state. Otherwise, if $\actions.\poss(\switchAction)$ is
executed, then the current sequential plan is executed until further
sensing is scheduled, or to completion. Lastly, we note that in our
project the rewards, i.e., functions
$\$,\hat\$:2^\propositions\to\Re$, for the POMDP posed to the
contingent session are sourced from a motivational component of our
robotic architecture. In this paper, for $\assumptiveDT{i}$ actions we
set $\$(x)$ to be a positive constant, and have $\hat\$(x)= - \$(x)(1
- \prob) / \prob$ where $\prob$ is the probability that $x$ is true. For
$\actions.\poss(\switchAction)$ actions we have $\hat\$(x)= -
\$(x)\prob/(1-\prob)$.\footnote{These values were chosen to encourage
  sensing.}

%%  We make no assumptions here about how they are
%% derived.

The starting belief-state for the abstract process is given in terms
of a {\em relaxed} $(\pp{:init})$ declaration. We first construct a
tree which only includes terms from the {\em relaxed} visitation
characterised by the active {\em assumptive} actions in the current
sequential plan. Figure \ref{fig:abstraction} shows the belief-state
of a more detailed version of our previous example. The diamonds
represent probabilistic terms of the state description while circles
represent conjunctive or atomic terms. 

\begin{figure}[h!]
  \centering
  \tikzstyle{tree} = [sibling distance=4.5mm]
  \tikzstyle{toplevel} = [grow'=right, sibling distance=22mm]
  \tikzstyle{seclevel} = [sibling distance=9mm]
  \tikzstyle{pnode} = [diamond, draw=black, minimum size=2.5mm]
  \tikzstyle{cnode} = [circle, draw=black, minimum size=3mm]
  \tikzstyle{assumption} = [solid, thick, draw=black]
  \tikzstyle{selected} = [solid, draw=black]
  \tikzstyle{unused} = [densely dashed, draw=black!40]
  \subfloat[The initial belief-state with the assumptions made by the continual
    planner in bold.]{
      \label{fig:abstraction-a}
      \begin{tikzpicture}[
    level 3/.style={tree}]
    \node[pnode, assumption] (cat) at (1,1) {} [toplevel]
      child[seclevel] {node[cnode, assumption] (office) {}
        child {node [pnode, assumption] (box) {}
          child {node [cnode] (boxp1) {}}
          child {node [cnode, assumption] (boxp2) {}}
        }
        child {node [pnode] (cup) {} 
          child {node [cnode] (cupp1) {}}
          child {node [cnode] (cupp2) {}}
        }
      }
      child {node[cnode] (kitchen) {}
        child {node [pnode] (box2) {} 
          child {node [cnode] (box2p1) {}}
          child {node [cnode] (box2p2) {}}
        }
      };
   \tiny
   \draw[assumption] (cat) -- (office) -- (box) -- (boxp2);
   \node[above=0 of cat] {$\pp{(category room1)}$};
   \node[below=0 of kitchen] {$\pp{kitchen}$};
   \node[below=0 of office] {$\pp{office}$};
   \node[below=0 of box] {$\pp{(is-in box)}$};
   \node[below=0 of box2] {$\pp{(is-in box)}$};
   \node[below=0 of cup] {$\pp{(is-in cup)}$};
   \node[right=0 of boxp1] {$\pp{place1}$};
   \node[right=0 of boxp2] {$\pp{place2}$};
   \node[right=0 of box2p1] {$\pp{place1}$};
   \node[right=0 of box2p2] {$\pp{place2}$};
   \node[right=0 of cupp1] {$\pp{place1}$};
   \node[right=0 of cupp2] {$\pp{place2}$};
    % \node[node] (kitchen) right of (cat) {};
  \end{tikzpicture}}
\qquad
  \subfloat[Removing facts that are not part of an assumption.]{
    \label{fig:abstraction-b}
    \begin{tikzpicture}[
    level 3/.style={tree}]
    \node[pnode, assumption] (cat) at (1,1) {} [toplevel, unused]
      child[seclevel] {node[cnode, assumption] (office) {}
        child {node [pnode, assumption] (box) {}
          child {node [cnode, unused] (boxp1) {}}
          child {node [cnode, assumption] (boxp2) {}}
        }
        child {node [pnode, unused] (cup) {} 
          child {node [cnode, unused] (cupp1) {}}
          child {node [cnode, unused] (cupp2) {}}
        }
      }
      child {node[cnode, selected] (kitchen) {}
        child {node [pnode, selected] (box2) {} 
          child {node [cnode, unused] (box2p1) {}}
          child {node [cnode, selected] (box2p2) {}}
        }
      };
   \draw[assumption] (cat) -- (office) -- (box) -- (boxp2);
   \draw[selected] (cat) -- (kitchen) -- (box2) -- (box2p2);
   \tiny
   \node[above=0 of cat] {$\pp{(category room1)}$};
   \node[below=0 of office] {$\pp{office}$};
   \node[below=0 of box] {$\pp{(is-in box)}$};
   \node[below=0 of box2] {$\pp{(is-in box)}$};
   \node[right=0 of boxp2] {$\pp{place2}$};
   \node[right=0 of box2p2] {$\pp{place2}$};
    % \node[node] (kitchen) right of (cat) {};
  \end{tikzpicture}}
\qquad
  \subfloat[Refinement, by adding relevant facts until the $\bstate_0$ size limit is reached.]{
    \label{fig:abstraction-c}
    \begin{tikzpicture}[
    level 3/.style={tree}]
    \node[pnode, assumption] (cat) at (1,1) {} [toplevel, unused]
      child[seclevel] {node[cnode, assumption] (office) {}
        child {node [pnode, assumption] (box) {}
          child {node [cnode, selected] (boxp1) {}}
          child {node [cnode, assumption] (boxp2) {}}
        }
        child {node [pnode, unused] (cup) {} 
          child {node [cnode, unused] (cupp1) {}}
          child {node [cnode, unused] (cupp2) {}}
        }
      }
      child {node[cnode, selected] (kitchen) {}
        child {node [pnode, selected] (box2) {} 
          child {node [cnode, selected] (box2p1) {}}
          child {node [cnode, selected] (box2p2) {}}
        }
      };
   \draw[assumption] (cat) -- (office) -- (box) -- (boxp2);
   \draw[selected] (box) -- (boxp1);
   \draw[selected] (cat) -- (kitchen) -- (box2) -- (box2p2);
   \draw[selected] (box2) -- (box2p1);
   \tiny
   \node[above=0 of cat] {$\pp{(category room1)}$};
   \node[below=0 of office] {$\pp{office}$};
   \node[below=0 of box] {$\pp{(is-in box)}$};
   \node[below=0 of box2] {$\pp{(is-in box)}$};
   \node[right=0 of boxp1] {$\pp{place1}$};
   \node[right=0 of boxp2] {$\pp{place2}$};
   \node[right=0 of box2p1] {$\pp{place1}$};
   \node[right=0 of box2p2] {$\pp{place2}$};
    % \node[node] (kitchen) right of (cat) {};
  \end{tikzpicture}}
  
  \caption{Generating the abstract belief space.}
\label{fig:abstraction}
\end{figure}

% \small
% \begin{tabtt}
% (\=:init (= (is-in DORA) kitchen) \+ \\
%        (probabilistic \=.8 (= (is-in box) office)  \\
% 		      \>.2 (= (is-in box) kitchen))) \\
% \end{tabtt}
% \normalsize

% \noindent and therefore, the belief-state $\bstate_0$ is:

% \small
% \begin{tabular}{cccc}
% \hline
% Probability & (is-in DORA)  & (is-in box)  & (is-in cup) \\
% \hline
% .8 & kitchen & office & \# \\
% .2 & kitchen & kitchen & \# \\
% \hline
% \end{tabular}
% \normalsize

 % With
% regards to abstraction, the contingent and serial sessions apply the
% same applicability conditions ---i.e., that each effect and condition
% $\prop$ is specified, $\prop\in\state$--- to actions from the DTPDDL
% problem description.

In the first step, we remove all facts that are not part of an
assumption (Fig. \ref{fig:abstraction-b}). At this point, the session
would proceed in an abstraction of the environment that does not
contain $\pp{place1}$, the $\pp{cup}$ or a $\pp{kitchen}$. 
%%
In a second step, we iteratively refine the relaxed declaration by
adding terms from the original statement of $(\pp{:init})$ while the
number of abstract states in $\bstate_0$ that occur with non-zero
probability according to that refined declaration remains of a
practicable size. In detail, for each excluded atomic term, we compute
the {\em entropy} of the active assumptions, {\em conditional} on the
corresponding proposition.  Loosely speaking, the lower the {\em
  conditional entropy} of an atom, the more information about the
assumptions the contingent planner expect to derive from it. Thus,
propositions that in the first place were excluded are iteratively
added to the problem posed to the conditional session in increasing
order according to that measure (Fig. \ref{fig:abstraction-c}).


%% In detail, for each excluded atomic term, we compute the {\em
%% entropy} of the corresponding proposition, {\em conditional} on the
%% active assumptions. The {\em conditional entropy} measure
%% is higher the more uncertain that proposition is given truth
%% assignments to propositions that are the subject of active
%% assumptions. Propositions that in the first place were excluded are
%% iteratively added to the problem posed to the conditional session in
%% increasing order according to that measure.


%% In particular, we compute the {\em conditional entropy} of
%% each excluded proposition given the active assumptions. Propositions
%% are candidates for addition in increasing order according to that
%% measure.

%% Of course, it is expensive to measure size exactly,
%% and in practise we estimate this to be $2^{|\propositions^\#|}$, where
%% $\propositions^\#$ is the set of propositions in the abstraction that
%% were not assigned the {\em unknown} value \#. Terms are proposed for
%% addition because they are probabilistic, and because they were omitted
%% in the {\em relaxed} visitation characterised by {\em assumptive}
%% actions in the sequential plan. In each iteration, we add one
%% candidate term with the largest number of parent terms already in the
%% candidate tree.









%%% Local Variables: 
%%% mode: latex
%%% TeX-master: "moritz_2011"
%%% End: 
