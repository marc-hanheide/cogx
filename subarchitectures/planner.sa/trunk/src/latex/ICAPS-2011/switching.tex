

We describe our continual planning approach that switches between
sequential and contingent sessions. Continual planning and execution
proceeds in a deterministic-action POMDP described in DTPDDL. During a
sequential session, planning is performed by a ``classical''
system,\footnote{That is, a system designed to solve fully observable
deterministic tasks.}  and execution proceeds according to the {\em
trace} computed by that system. Taking the form of a classical plan,
the {\em trace} specifies a sequence of POMDP actions that achieve the
agent's objectives in a deterministic approximation, i.e., {\em
determinisation}, of the problem at hand. More precisely, the trace is
an interleaved sequence of POMDP actions and {\em assumptive}
actions. The latter correspond to assumptions the planner makes about
the truth value of {\em runtime state variables} -- e.g. that a box of
cornflakes is located in the kitchen at the third plan step. Another
kind of assumption, called an {\em applicability} assumption, is made
when the trace includes an action $\action$ that is not applicable
with probability $1$ at the belief-state
\bstate\ that the system is projected to be in when \action\ is executed, 
i.e., $\exists\state\in\states\;
\bstate(\state) > 0$ and $\poss(\action)\not\subseteq\state$. By
scheduling
\action,  the serial planner makes an assumption about the
observability of the precondition $\poss(\action)$.

Our approach always begins with a sequential session. Non-assumptive
actions from the trace are executed until the applicability of the
next scheduled action, here \action, is questionable. A contingent
session then begins, which tailors sensory processing by the agent to
perceive whether the assumptions made in the trace hold. For each {\em
active} assumption in the trace, we add a {\em disconfirm} action to
the POMDP whose execution is rewarding (resp. costly) if the
assumption is false (resp. true). We also add one {\em confirm} action
that is rewarding (resp. costly) to execute if $\poss(\action)$ is
true (resp. false). If execution of the contingent plan applies a
disconfirm action, then a new sequential session begins at the current
underlying belief-state. If the confirm action is executed, the
sequential session is resumed, and \action\ is executed. Otherwise,
control rests in the continual session.



Because contingent planning is only practical in relatively small
POMDPs, contingent sessions proceed in an abstract decision process
determined by the current trace. That abstract process is constructed
by first omitting all variables from the model that are not featured
in the trace, and by then iteratively refining that abstraction while
the resulting model is of a practicable size.


%% PLEASE STICK A PARAGRAPH HERE, (SENSIBLY) REQUESTED BY MORITZ, ABOUT
%% HOW WE DO BELIEF REVISION ACROSS SWITCHING.

\Omit{
  Therefore, although our current implementation does not
  explicitly support it, our approach generalises to POMDPs with
  stochastic action models.
}

\subsection{Sequential Planning in POMDPs}

A sequential session uses a classical planner to compute a trace. The
latter encapsulates assumptions about: (1) the true underlying state,
(2) how the plan's execution will progress, and (3) the possibility of
the agent eventually holding strong beliefs about the truth values of
specific state propositions. Here we describe the deterministic
planning problem, derived from the DTPDDL model, that admits plans
which correspond to traces.

In a deterministic-action POMDP all the uncertainty in state and
action is expressed in the $(\pp{:init})$ DTPDDL declaration. Our
approach uses the structure of that, as it occurs in the problem
description, to define a set of state-assumptions available to
sequential planning.
%%
Writing \#\ if the value of a proposition is unspecified, for
$\pp{DORA}$ we have the following assumptions:

\small
\begin{tabular}{cccc}
\hline
Probability & (is-in DORA)  & (is-in box)  & (is-in cup) \\
\hline
.24 & kitchen & office & office \\
.06 & kitchen & kitchen & office \\
.56 & kitchen & office & kitchen \\
.14 & kitchen & kitchen & kitchen \\
.7 & kitchen & \# &  kitchen\\
.3 & kitchen & \# & office \\
.8 & kitchen & office & \# \\
.2 & kitchen & kitchen & \# \\
1.0 & kitchen & \# & \# \\
\hline
\end{tabular}
\normalsize

\noindent In detail, a valid assumption in our setting corresponds to a {\em
relaxed} visitation of the root term of $(\pp{:init})$. In this
relaxed case, a conjunctive term is visited iff its atomic subterms
are visited, and zero or more of its immediate probabilistic subterms
are visited. The starting state, $\state_0$, for sequential planning
is an {\em abstract} state that captures the unique assumption which
has probability $1.0$. In $\pp{DORA}$, that is:

\[
\begin{array}{l}
\state_0 \equiv \{(=(\pp{is-in}~\pp{DORA})~\pp{kitchen}),\\
\;\;(=(\pp{is-in}~\pp{box})~\#), (=(\pp{is-in}~\pp{cup})~\#)\}.
\end{array}
\]

In order to make assumptions available to the sequential planner, we
add one {\em assumptive} action $\assumptiveS{i}$ to the problem for
each element, $p_i (T_i)$, of each probabilistic term from
$(\pp{:init})$. The physics of these actions is as
follows. $\assumptiveS{i}$ can be executed if no $\assumptiveS{j}$,
$j \neq i$, has been executed from the same probabilistic term, and,
either $(\pp{probabilistic}~..p_i~(T_i)..)$ is in the root conjunct,
or it occurs in $T_k$ for some executed $\assumptiveS{k}$.
%%
Executing $\assumptiveS{i}$ in $s$ effects a transition to a successor
state $s^{T_i}$ with probability $p_i$, and $s^\bot$ with probability
$1 - p_i$. Here, $s^{T_i}$ is the union of $s$ with atomic terms from
$T_i$ and conflicting propositions from $s$ deleted. For example, if
the action adds $(=(\pp{is-in}~\pp{box})~\pp{office})$, then it must
delete $(=(\pp{is-in}~\pp{box})~\#)$. State $s^\bot$ is an added sink,
where $\forall\action\in\actions$ $\reward(\state^\bot, \action) = 0$
and $\transProb(\state^\bot, \action, \state^\bot) = 1$.

For sequential planning the operators from the POMDP model are
available with semantics that accommodate the above abstraction as
follows. In the deterministic model a proposition $\prop$ can be
thought to have a ternary interpretation at a state \state, as either
{\em true}, written $\prop\in\state$, {\em false}, written
$\prop\not\in\state$, or {\em unspecified}, with a slight abuse of
notation written $\prop\#\in\state$. For example, in $\pp{DORA}$
$(=(\pp{is-in}~\pp{box})\pp{office})\#\in\state_0$. For POMDP actions
$\action\in\actions$ in the deterministic model, if
$\prop \in \poss(\action)$, or if $\prop$ is the subject of a positive
or negative effect of $\action$, then $\action$ is not applicable
in \state\ if $\prop\#\in\state$. Indeed, first an assumptive action
must be applied to make a truth assignment to \prop, and then that
truth assignment could be modified by POMDP operators. Lastly, we
address the switching semantics of action execution. If
$\action\in\actions$ has a precondition that is not true with
probability $1$ in the underlying belief $\bstate$, then before it is
executed blindly, the system switches to a contingent session,
discussed in detail in the following section.


We now describe the optimisation criteria given to the classical
planner. Where $\prob_i$ is the probability that the $i^{th}$
sequenced action $\action_i$ does not transition to $\state^\bot$, we
define the value of a trace
$\state_0, \action_0, \state_1, \action_1,.., \state_N$ from the
deterministic model to equal:

\begin{equation}\label{eq:tracevalue}
V(\state_0, \action_0, \state_1, \action_1,.., \state_N) =  \prod_{i=1..N-1} \prob_i \sum_{i=1..N-1} \reward(\state_i, \action_i)
\end{equation}

\noindent The optimal trace given a plan, i.e., the sequence of non-assumptive
actions from the trace, therefore has value:

\[
V^* = \max_{\prob_1, .., \prob_n} \prod_{i=1..N-1} \prob_i \sum_{i=1..N-1}
\reward(\state_i, \action_i),
\]

\noindent which is equal to the maximal additive contribution a trace of the plan can
make to Eq~\ref{eq:expectedvalue}, taking an FSC that corresponds to
the sequential plan.  In goal directed problems, where the only
non-zero reward is received at the first transition to a goal state,
the Eq~\ref{eq:tracevalue} criteria gives us the behaviour
of \system{FFR$_a$}~\cite{yoon:etal:2007}. From the perspective of
gradient-based reinforcement learning systems, such as William's
REINFORCE and GPOMDP variants for planning~\cite{olivier:doug:2009},
under reasonable assumptions the optimal trace identifies a maximal
gradient step from a uniformly random soft-max policy. Finally, it is
worth clarifying that although the above discussion has been in terms
of a trace of length $N$, there is nothing in our approach that
artificially limits the length of candidate sequential plans.


\subsection{Decision-Theoretic Planning in Abstractions}

In a contingent session our switching planner solves a sensing problem
in an abstract processes defined in terms of the assumptive actions in
the current trace, and the $(\pp{:init})$ declaration from the
original problem description.

The focus for contingent planning is on sensing, we therefore give the
process in this session an augmented reward model.  This reflects the
value of performing sensing actions in the context of the trace
proposed by the preceding sequential deliberation. First, all rewards
from the original POMDP are retained. Then, for each $\assumptiveS{i}$
action scheduled by the current trace, we have a {\em dual}
$\assumptiveDT{i}$ so that $\forall\state\in\states$:

\[
\reward(\state, \assumptiveDT{i}) = \bigg\{ \begin{array}{ll}
\$(T_i) & \pp{if}~\;\;T_i \not\subseteq \state \\
-\$(T_i) & \pp{otherwise} \\
\end{array}
\]

\noindent Where $\$(T_i)$ is a positive numeric quantity which
captures the reward the agent receives for correctly rejecting a
false assumption. The added dual actions are propositionally trivial
-- i.e., $\forall\state\in\states$ $\poss(\assumptiveDT{i})
\subseteq \state$ and $\assumptiveDT{i}(\state) \equiv \state$.  
%%
In order to reduce the number of possible rejections we do not
consider assumptions that are not {\em active} with respect to the
action, $\action_i$, whose scheduled execution switch the system into
the contingent session. A assumption $\assumptiveS{i}$ is active if
$\action_i$ is not applicable in the sequential model unless
$\assumptiveS{i}$ is scheduled in the trace prefix to $\action_i$. For
example, if $\pp{DORA}$'s trace is:

\[
\begin{array}{l}
\actions^{\circ}(.8;(=(\pp{is-in}~\pp{box})\pp{office}));\\
\actions^{\circ}(.3;(=(\pp{is-in}~\pp{cup})\pp{kitchen}));\\
(\pp{look-for-object}~\pp{DORA}~\pp{box}~\pp{office});\\
(\pp{look-for-object}~\pp{DORA}~\pp{cup}~\pp{kitchen});\\
(\pp{report-is-in}~\pp{box}~\pp{office}); \\
(\pp{report-is-in}~\pp{cup}~\pp{kitchen})
\end{array}
\]

\noindent Taking the switching action $\action_i$ to be
$(\pp{look-for-object}~\pp{DORA}~\pp{box}~\pp{office})$, we have that
$\actions^{\circ}(.3;(=(\pp{is-in}~\pp{cup})\pp{kitchen}))$ is not
active, and therefore exclude
$\actions^{\bullet}(.3;(=(\pp{is-in}~\pp{cup})\pp{kitchen}))$ from the
POMDP posed to the contingent session.

Again taking $\action_i$ to be the switching action, we include
another propositionally trivial action $\actions.\poss(\action_i)$
with the reward property:

\[
\reward(\state, \actions.\poss(\action_i)) = \bigg\{ \begin{array}{ll}
\$(\poss(\action_i)) & \pp{if}~\;\; \poss(\action_i) \subseteq \state \\
-\$(\poss(\action_i)) & \pp{otherwise} \\
\end{array}
\]

For the purposes of successive switching, execution of either a dual
action of the form $\assumptiveDT{i}$, or the
$\actions.\poss(\action_i)$, returns control to a sequential
session. If a dual is executed, then the sequential planner must
replan with $\bstate_0$ equal to the underlying
belief-state. Otherwise, if $\actions.\poss(\action_i)$ is executed,
then the current sequential plan is executed until further sensing is
scheduled, or to completion. In our project the rewards, i.e.,
function $\$:2^\propositions\to\Re$, for the POMDP posed to the
contingent session are sourced from a motivational component of our
robotic architecture.

The starting belief-state for the abstract process is given in terms
of a {\em relaxed} $(\pp{:init})$ declaration.  We first construct a
tree which only includes terms from the {\em relaxed} visitation
characterised by the active {\em assumptive} actions in the current
sequential plan. Taking our example trace from earlier, we have a {\em
relaxed} declaration:

\small
\begin{tabtt}
(\=:init (= (is-in DORA) kitchen) \+ \\
       (probabilistic \=.8 (= (is-in box) office)  \\
		      \>.2 (= (is-in box) kitchen))) \\
\end{tabtt}
\normalsize

\noindent and therefore, the belief-state $\bstate_0$ is:

\small
\begin{tabular}{cccc}
\hline
Probability & (is-in DORA)  & (is-in box)  & (is-in cup) \\
\hline
.8 & kitchen & office & \# \\
.2 & kitchen & kitchen & \# \\
\hline
\end{tabular}
\normalsize

\noindent At this point, the session would proceed in an
abstraction of the environment that does not contain a
$\pp{cup}$. With regards to abstraction, the contingent and serial
sessions apply the same applicability conditions ---i.e., that each
effect and condition $\prop$ is specified $\prop\in\state$
(resp. $\prop\#\in\state$)--- to actions from the DTPDDL problem
description.

In a second step, we iteratively refine the relaxed declaration by
adding terms from the original statement of $(\pp{:init})$ while the
number of abstract states that occur with non-zero probability
according to that refined declaration remains of a practicable
size. In detail, for each excluded atomic term, we compute the {\em
entropy} of the corresponding proposition, {\em conditional} on the
active assumptions. The {\em conditional entropy} measure
is higher the more uncertain that proposition is given truth
assignments to propositions that are the subject of active
assumptions. Propositions that in the first place were excluded are
iteratively added to the problem posed to the conditional session in
increasing order according to that measure. 

%% In particular, we compute the {\em conditional entropy} of
%% each excluded proposition given the active assumptions. Propositions
%% are candidates for addition in increasing order according to that
%% measure.

%% Of course, it is expensive to measure size exactly,
%% and in practise we estimate this to be $2^{|\propositions^\#|}$, where
%% $\propositions^\#$ is the set of propositions in the abstraction that
%% were not assigned the {\em unknown} value \#. Terms are proposed for
%% addition because they are probabilistic, and because they were omitted
%% in the {\em relaxed} visitation characterised by {\em assumptive}
%% actions in the sequential plan. In each iteration, we add one
%% candidate term with the largest number of parent terms already in the
%% candidate tree.








