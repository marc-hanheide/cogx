

We now describe a continual planning approach that switches between
sequential and decision-theoretic planning. 


In this work, the problem at hand is a POMDP described in DTPDDL. From
hereon, we restrict our attention to POMDPs with deterministic
actions. Therefore it is worthwhile noting that any POMDP with
stochastic actions can be compiled into an equivalent
deterministic-action POMDP, where all the original action uncertainty
is expressed in the starting-state
distribution~\cite{ng:Jordan:2000}. In our setting, that of
finite-horizon planning, such a compilation yields a finite flat
representation of the original POMDP, only with deterministic
actions. 


PLEASE STICK A PARAGRAPH HERE, (SENSIBLY) REQUESTED BY MORITZ, ABOUT
HOW WE DO BELIEF REVISION ACROSS SWITCHING.

\Omit{
  Therefore, although our current implementation does not
  explicitly support it, our approach generalises to POMDPs with
  stochastic action models.
}

\subsection{Sequential Planning in POMDPs}


Our approach uses sequential planning to compute a linear plan, and
one trace of that plan's execution. The latter encapsulates
assumptions about: (1) the true underlying state, (2) how the plan's
execution will progress, and (3) the possibility of the agent
eventually holding strong beliefs about the truth values of specific
state propositions. Sequential planning proceeds in a continual manner
until the verification of an assumption about a runtime state
proposition is scheduled. At that point the system switches to a
decision-theoretic planner that is used to tailor sensory processing
by the robot to test the validity of the assumptions made so far.

In a deterministic-action POMDP, all the uncertainty in state and
action is expressed in the $(\pp{:init})$ declaration. Our approach
uses the structure of that, as it occurs in the problem description,
do define a set of abstract state-assumptions available to 
sequential planning.  
%%
Writing \#\ if the value of a proposition is unspecified in the
assumption, for the $\pp{R2D2}$ example we have that the following
assumptions are available:

\small
\begin{tabular}{cccc}
\hline
Probability & (is-in R2D2)  & (is-in box)  & (is-in cup) \\
\hline
.24 & kitchen & office & office \\
.06 & kitchen & kitchen & office \\
.56 & kitchen & office & kitchen \\
.14 & kitchen & kitchen & kitchen \\
.7 & kitchen & \# &  kitchen\\
.3 & kitchen & \# & office \\
.8 & kitchen & office & \# \\
.2 & kitchen & kitchen & \# \\
1.0 & kitchen & \# & \# \\
\hline
\end{tabular}
\normalsize

\noindent A valid assumption in our setting corresponds to a {\em
relaxed} visitation of the root term of $(:init)$. In this relaxed
case, a conjunctive term is visited iff its atomic subterms are
visited, and zero or more of its immediate probabilistic subterms are
visited. The starting state, $\state_0$, for sequential planning is
the abstract state with probability $1.0$ -- E.G., for $\pp{R2D2}$
$s_0 \equiv \{(= (is-in R2D2) kitchen)\}$.

In order to make assumptions available to the sequential planner, we
add one {\em assumptive} action $\actions(p_i;(T_i))$ to the problem
for each element, $p_i (T_i)$, of each probabilistic term from
$(:init)$. The physics of these actions is as
follows. $\actions(p_i;(T_i))$ can be executed if no
$\actions(p_j;(T_j))$, $j \neq i$, has been executed from the same
probabilistic term, and, either $(\pp{probabilistic}~..p_i~(T_i)..)$
is in the root conjunct, or it occurs in $T_k$ for some executed
$\actions(p_k;(T_k))$.
%%
Executing $\actions(p_i;(T_i))$ in $s$ effects a transition to a
successor state $s^{T_i}$ with probability $p_i$, and $s^\bot$ with
probability $1 - p_i$. Here, $s^{T_i}$ is the union of $s$ and atomic
terms from $T_i$, and $s^\bot$ is a sink state -- I.e.,
$\forall\action\in\actions$ $\reward(\state^\bot, \action) = 0$ and
$\transProb(\state^\bot, \action, \state^\bot) = 1$.

Because $\state_0$ is abstract, and because actions can have
preconditions that are not uniformly true in the underlying
(resp. abstract) belief-state, we must make two further adjustments to
the problem posed to the sequential planner. First, if
$\action\in\actions$ has a precondition that is not true with
probability $1$, then before it is executed blindly, the sequential
continual planning switches to a decision-theoretic mode, discussed in
detail below. Overall, by scheduling $\action$ the sequential planner
makes a naive assumption about the possibility of the agent eventually
``committing'' to the truth values of specific state
propositions. Thus, the role of decision-theoretic plans is to verify
the validity of assumptions about the runtime state in sequential
planning. Second, if the truth value of $\prop$ is not known, and no
assumption is made about it in the current abstract state, then if
$\prop \in
\poss(\action)$ we have that action $\action$ is not applicable.

%% Because $\state_0$ is abstract, and because actions can have
%% preconditions that are not uniformly true in the underlying
%% (resp. abstract) belief-state, we must make two further adjustments to
%% the problem posed to the sequential planner. First, if
%% $\action\in\actions$ has a precondition that is not true with
%% probability $1$, then before it is executed blindly, the sequential
%% continual planning switches to a decision-theoretic mode, discussed in
%% detail below. Overall, by scheduling $\action$ the sequential planner
%% makes a naive assumption about the possibility of the agent eventually
%% committing ---for the purpose of the current task--- to the truth
%% values of specific state propositions. Second, if the truth value of
%% $\prop$ is not known, and no assumption is made about it in the
%% current abstract state, then if $\prop \in \poss(\action)$ we have
%% that action $\action$ is not applicable.

%% In other words, the value of a proposition in an abstract
%% state is either {\em true}, {\em false}, or {\em unspecified} (i.e.,
%% \#). In the last case actions whose execution is preconditioned on the
%% term being truth functional are not available.


Finally, we come to valuing sequential plans, and therefore giving a
criteria for the sequential planner to optimise. Where $p_i$ is the
probability that the $i^{th}$ sequenced action does not transition to
$s^\bot$, we value a trace as follows:

\begin{equation}\label{eq:tracevalue}
V(s_0, a_0, s_1, a_1,.., s_N) =  \prod_{i=1..N} p_i \sum_{i=1..N-1} \reward(\state_i, \action_i)
\end{equation}

\noindent The optimal trace given a plan, i.e., the non-assumptive
actions in sequence, therefore has value:

\[
V^* = \max_{p_1, .., p_n} \prod_{i=1..N} p_i \sum_{i=1..N-1}
\reward(\state_i, \action_i),
\]

\noindent which is equal to the maximal contribution a trace of the plan can
make to Eq~\ref{eq:expectedvalue}. In other words, writing $V_i$ for
the value of the $i^{th}$ trace of the plan expressed in $s_0, a_0,
s_1, a_1,.., s_N$, we have

\[
V^* = \arg\min_{V_i} \Expect \bigg{[} 
\sum_{t=0}^{N}  \reward(\bstate_t) \mid \pp{PLAN}, \bstate_0
\bigg{]} - V_i.
\]

In goal directed problems, where the only non-zero reward is received
at the first transition to a goal state, the Eq~\ref{eq:tracevalue}
criteria corresponds exactly to that of the infamous degenerate
hindsight system
\system{FFR$_a$}~\cite{yoon:etal:2007}. Form the perspective of
gradient-based reinforcement learning systems, such as William's
REINFORCE and GPOMDP variants for planning~\cite{olivier:doug:2009},
under reasonable assumptions the optimal trace identifies a maximal
gradient step from a uniformly random soft-max policy. Finally, it is
worth clarifying that although the above discussion has been in terms
of a trace of length $N$, there is nothing in our approach that
artificially limits the length of candidate sequential plans.


%% Finally, we come to the issue of valuing sequential plans, and
%% therefore giving the criterion that the sequential planner seeks to
%% optimise. Where $p_i$ is the probability that the $i^{th}$ action
%% execution is not a transition to $s^\bot$, we value a trace according
%% to:

%% \begin{equation}\label{eq:tracevalue}
%% V(s_0, a_0, s_1, a_1,.., s_N) =  \prod_{i=1..N} p_i \sum_{i=1..N-1} \reward(\state_i, \action_i)
%% \end{equation}

%% \noindent In goal directed problems, where the only non-zero reward is
%% received at the first transition to a goal state,
%% Eq~\ref{eq:tracevalue} corresponds exactly to the criteria of the
%% infamous degenerate hindsight system
%% \system{FFR$_a$}~\cite{yoon:etal:2007}. Form the perspective of
%% gradient-based reinforcement learning systems, such as William's
%% REINFORCE and GPOMDP variants for planning~\cite{olivier:doug:2009},
%% under reasonable assumptions the optimal trace yields a maximal
%% gradient step from a uniform soft-max policy. 

\subsection{Decision-Theoretic Planning in Abstractions}

The problem posed to the decision-theoretic system is an abstraction
of the problem at hand with a distinct utility model that focuses the
agents activities on assumption verification. The abstraction is
induced according to the executing sequential plan. Specifically, we
first omit all problem variables that are not mentioned in the linear
trace, and then refine that preliminary model, by progressively adding
more problem variables while the resulting state space remains of a
practicable size. Refinement in our approach is a simple and
inexpensive step, guided by the level at which variable symbols occur
in the description of the initial conditions. The abstract process is
also given a utility model that captures the reward (resp. penalty) of
correctly (resp. falsely) in/validating the assumptions made in the
linear trace. In detail, for each assumption on a state variable, we
can have a reward of $1$ for correctly rejecting it, and a penalty of
$-1$ for falsely rejecting it. Decision theoretic planning proceeds in
a more-or-less continual manner until an assumption is rejected, or
until it returns control to the sequential system because it can see
no benefit from performing further actions in the abstraction.



Addressing briefly the specifics of our mobile robot's
architecture. In our scenario the utility model for the POMDP posed to
decision-theoretic planning is sourced from a motivational component
of our robotic architecture. This proposes a non-zero reward
(resp. penalty) for sequential planning with a true (resp. false)
assumption about a state proposition.

