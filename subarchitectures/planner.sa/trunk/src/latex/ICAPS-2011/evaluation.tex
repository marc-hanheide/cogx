% coming soon

\subsection{Implementation}

The implementation of the continual planning and switching parts is
based on MAPSIM \cite{brenner:nebel:jaamas09} and is able to use several planners as
the underlying planning system. We use a modified version of Fast
Downward \cite{fast-downward}, which we extended with the support for
actions with probabilities.

Our implementation uses the
~\citeauthor{king:2009}~(\citeyear{king:2009}) \system{dlib-ml} for
successive estimation of the underlying belief-state.

The baseline approach is using the same continual planning system as
the switching planner, but instead of creating an observation problem
for the decision theoretic planner it will just execute one sensing
action -- assuming that this action will confirm its assumption.

\subsection{Experiments}
For evaluation, we used a robot exploration domain similar to the one
used on a physical robotic system. The goal of the robot is to find
objects in a human environment and reporting back their position to a
human.

The basic building blocks of the domain consist of {\tt rooms}, {\tt
places} and {\tt objects}. Places are grouped in rooms and objects (as
well as the robot) are always in a place. The robot can move around
the rooms via connections between places given by the {\tt connected}
predicate. Each room has a (possibly unknown) {\em category/}
(e.g. kitchen, office, living room) and depending on this category,
objects are placed inside the rooms. An instance of this domain is
shown in figure\ref{fig}.

The robot can find out if an object is at a certain place by executing
the {\tt look-for-object} action, which may result in a perception if
the object is in fact there (though some objects are harder to detect
than others -- so absence of a percept is no proof of the object's
absence). Additionally, if the robot is in the presence of a human, it
may simply ask what type of room they are currently in -- but
conducting a dialogue is a bit more costly than simply running the
vision algorithm (cost of 8 vs costs of 3).

We conducted experiments on scenarios of several sizes with several
kinds of goals (though the goal is always reporting the position of a
certain kind of object). In order to determine the impact of sensor
reliability on both approaches, we also ran several tests in identical
problems with different probabilities for sensing the goal object.
These problems are denoted by the -easy, -medium and -hard suffixes;
in the easy case the probability of percieving the object is 0.9, in
medium it is 0.65 and 0.4 in hard.

As the initial state of the problems is stochastic, we ran 50
simulations for each configuration. In each run the simulator
generated the true world state by sampling from the same initial state
distribution the planner was using. So on average the world did
conform to the planner's expectations. Not all problems generated in
this manner have a solution. We did not reject these problems, as we
think that detecting the non-existance of a solution is as valuable in
exploration problems than finding one.

Fast Downward was run with the cyclic causal graph heuristic using A*
search or weighted A* with a weight of 5, depending on the difficulty
of the problem. We then performed multiple tests with different limits
for the belief space size of the decision theoretic subtask. Higher
limits should cause longer planning times but be beneficial to plan
quality as more contingencies can be taken in to account by the POMDP
planner.

The graphs show the average costs of the agent's actions, the number
of solved tasks as a percentage of the solvable tasks, as well as the
planing time for the continual and decision theoretic planners.

For objects that can be easily detected there is little gain in using
a decision theoretic planner, as the greedy sensing appoach by the
baseline continual planner is obviously sufficient here. With
decreasing sensor reliability the more sophisticated observation
planning pays off: while the resulting plans are still longer on
average, the impact on the number of solved tasks was much smaller
than for the baseline system. 

We think that a part of the improvement is due to the segmentation of
the plan into several subtask, essentially performing hierarchical
planning. Especially when the continual planner performs badly this is
a huge gain.

%%% Local Variables: 
%%% mode: latex
%%% TeX-master: "moritz_2011"
%%% End: 
