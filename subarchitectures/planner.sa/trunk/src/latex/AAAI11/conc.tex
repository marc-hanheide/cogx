%\vspace{-1ex}

We have addressed a key challenge, specifically that of high-level
continual planning and execution monitoring for efficient
deliberations according to rich probabilistic models afforded by
recent integrated robotic systems. We developed a system that can plan
quickly given large realistic probabilistic models, by switching
between: (a) fast sequential planning, and (b) expensive DT planning
in small abstractions of the problem at hand. Sequential and DT
planning is interleaved, the former identifying a rewarding sequential
plan in the underlying process, and the latter solving small sensing
problems posed during sequential plan execution.
%%
We have evaluated our system in large real-world task and
observation planning problems, finding that it performs quickly
and relatively efficiently.

%% We developed a continual planning and execution monitoring
%% system for use on a mobile robot platform. .
%% %%
%% Our planner switches between: (a) fast sequential planning, and (b)
%% expensive DT planning in small abstractions of the problem at
%% hand. Our approach addresses a key challenge put to the community, by
%% Planning for a DTPDDL model of the environment, sequential and DT
%% sessions are interleaved, the former identifying a rewarding plan, and
%% the latter solving practically sized sensing problems posed during
%% serial plan executions execution.
%% %%
%% Our system answers a key challenge, with {\em probabilistic} --i.e., {\em quantitative}--
%% models of noisy sensing and state.




%% is used to compute an
%% sequential plans and complementary runtime evolutions of the
%% decision process. That plans are executed until the outcome of the next
%% action scheduled for execution is too uncertain. At that point a DT
%% session performs sensing that can help determine how the sequential
%% plan might evolve, or otherwise acts to achieve the goals. We find our
%% approach performs quickly and relatively efficiently in a number of
%% large real-world task and observation planning problems.

\Omit{
The most pressing item for future research, is to develop a scheme
whereby the serial planner can relax {\em executability} assumptions,
so that conformant (or semi-conformant) plans can be executed without
interruption by a DT session. A more general criticism of
switching continual planning concerns interleaved sequential and
DT sessions failing to make any progress towards the
objectives. For example, in the worst case we can have each sequential
session producing an identical plan, and each decision-theoretic
session rejecting it without further sensing. Although this has not
been an issue in our work so far, it must be dealt with rigorously
in the future. We suggest that a good way to mitigate this problem is
by developing a {\em motivational} component that maintains a {\em
dynamic} reward model whose limiting behaviour prevents that switching
deadlock.
}




%% The sequential mode of our system always schedules decision-theoretic
%% planning before executing actions that are not applicable with
%% certainty. This is inefficient if the utility of the plan is not
%% dependent on that actions successful execution. This situation always
%% arises for conformant plans. 



%% The switching continual planning system we have described serves as
%% the underlying planner for CogX.




%% the effects of a {\em sense} schema are perceptual, whereas the
%% effects of an {\em operator} schema are over state propositions.




%% posted by a motivational component of the underlying robotic
%% architecture. 

%% contingent sensory plans that are tailored to current
%% objectives.

%% In this paper we present an approach to continual planning that uses
%% two planning systems. The first 
				   
%%  to a distinct class of
%% challenges. We suppose 
%% %%
%% The underlying environment is modelled as a POMDP. We use the fast
%% classical satisficing system FastDownward to find a deterministic
%% sequential plan and complementary runtime evolution of that
%% process. This corresponds to a generalisation of replanning in
%% probabilistic planning to problems with partial observability.

%% Interaction between the sequential planner and execution proceeds
%% more-or-less analogously to popular replanning approaches

%% Addressing these challenges in a monolithic framework, we present a
%% {\em switching} continual planner, that uses the fast sequential
%% satisficing procedure FastDownward to perform net-benefit
%% %%
%% makes reasonable assumptions about the evolution of the runtime state
%% given a POMDP model of the environment. 

%% contingent sensory plans that are tailored to current
%% objectives.


%% The decision-theoretic planner is able to tailor sensory processing on
%% a robot platform to the current objective, while FastDownward  quickly 


%% In this paper we develop a continual planing system that uses two
%% planning systems. The first, is a state-of-the-art domain independent
%% planner for deterministic problems. The second is a information-state
%% contingency planning the information-state space of 


%% Continual planning is a powerful technique that goes some way to
%% addressing those challenges. That approach interleaves planning and
%% execution, deliberately postponing planning for contingencies unless
%% they eventuate during execution. 


%% computing a single sequential plan and
%% eventuality



%% To these challenges it seems a continual planning approach is best, 

%% where the runtime state evolves during plan execution 


%% interleaved planning and execution. 

%% the latter is able to tailor sensory processing on a robot platform,
%% in order that it.

%% {\em ad-hoc} 

%% , reason about degrees of belief and uncertainty about the world, 

%% probabilistic sequential decision making in practical sized problems
%% is intractable.

%% quantitative probabilistic models of the perception and action .
