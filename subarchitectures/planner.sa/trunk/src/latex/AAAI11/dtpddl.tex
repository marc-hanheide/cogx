


We give an overview of the declarative first-order language DTPDDL, an
extension of PPDDL that can express probabilistic models of the
sensing consequences of acting, to quantitatively capture
unreliability in perception. There are straightforward compilations
from problems expressed in DTPDDL to flat state-based (and
propositionally factored) representations of the underlying decision
process. Although similar to the POND input
language~\cite{Bryce:2008}, DTPDDL distinguishes itself by explicitly
treating state and perceptual symbols separately, and by providing
distinct declarations for operators (i.e, state model) and senses
(i.e., observation model). In this last respect, DTPDDL admits more
compact domain descriptions where sensing effects are common across
multiple operators. In detail, DTPDDL has perceptual analogues of
fluent and predicate symbols. For example, a simple {\em object
search} domain would have:

\vspace{-1ex}
\small
\begin{tabtt}
(\=:functions\\
  \> (is-in ?v - visual-object) - location )\\
(:perceptual-functions\\
  \> (o-is-in ?v - visual-object) - location )
\end{tabtt}
\normalsize
\vspace{-1ex}

\noindent Where the first fluent symbol models the actual location of
objects, and the second the instantaneous sensing of objects
following application of an action with sensing consequences.
%%
To model sensing capabilities, we have operator-like ``sense''
declarations, with preconditions expressed using state and action
symbols, and uniformly positive effects over perceptual symbols. For
example, where {\em look-for-object} is the operator that applies an
object detection algorithm at a specific place, an {\em
object search} task will have:

\vspace{-1ex}
\small
\begin{tabtt}
(\= :sense vision  :parameters \+ \\
  (?r -robot ?v -visual-object ?l -location) \\
 :execution  (look-for-object ?r ?v ?l) \\
 :precondition (and (= (is-in ?r) ?l) ) \\
 :\=effect (and \+\\
    (\= when (= (is-in ?v) ?l)\\
    \>(probabilistic .8 (= (o-is-in ?v) ?l))) \\
   (when (not (= (is-in ?v) ?l)) \\
    \>(probabilistic .1 (= (o-is-in ?v) ?l))))) \\
\end{tabtt}
\normalsize
\vspace{-3ex}

\noindent I.e., there is a 10\% {\em false positive} rate, and 20\%
  probability of a {\em false negative}. This representation allows us to represent actions that have multiple independent observational effects.

We now review the DTPDDL syntax for describing an initial state
distribution, taken verbatim from PPDDL, in order to aid us in
discussing our planner. That distribution is expressed in a tree-like
structure of terms. Each term is either: (1) atomic, e.g., a state
proposition such as $(=(\pp{is-in}~\pp{box})\pp{office})$, (2)
probabilistic, e.g., $(\pp{probabilistic}~\prob_1 (T_1) .. \prob_n
(T_n))$ where $T_i$ are conjunctive, or (3) a conjunct over
probabilistic and atomic terms. The root term is always conjunctive,
and the leaves are atomic. For example, a simplified object search
could have:\footnote{In PDDL, $(\pp{:init}~T_1..T_n)$ expresses the
conjunctive root of the tree -- i.e., the root node
$(\pp{and}~T_1..T_n)$. Also, we shall write $\prop$, rather than
$(\pp{and}~\prop)$, for conjunctive terms that contain a single atomic
subterm.}

\vspace{-1ex}
\small
\begin{tabtt}
(\=:init (= (is-in Robot) kitchen) \+ \\
       (probabilistic \=.8 (= (is-in box) office)  \\
		      \>.2 (= (is-in box) kitchen)) \\
       (probabilistic .3 (= (is-in cup) office)  \\
		      \>.7 (= (is-in cup) kitchen))) \\
\end{tabtt}
\normalsize

\vspace{-3ex}

\noindent The interpretation is given
by a {\em visitation} of terms: An atom is {\em visited} iff
its conjunctive parent is visited, and a conjunctive term is visited
iff all its immediate subterms are visited. A probabilistic term is
visited iff its conjunctive parent is visited, and exactly one of its
subterms, $T_i$, is visited. Each visitation of the root term
according to this recursive definition defines a starting state,
along with the probability that it occurs. The former corresponds to the
union of all visited atoms, and the latter corresponds to the product
of $\prob_i$ entries on the visited subterms of probabilistic
elements. Making this concrete, the above example yields the
following flat distribution:


\small
\begin{tabular}{cccc}
\hline
Probability & (is-in Robot)  & (is-in box)  & (is-in cup) \\
\hline
.24 & kitchen & office & office \\
.06 & kitchen & kitchen & office \\
.56 & kitchen & office & kitchen \\
.14 & kitchen & kitchen & kitchen \\
\hline
\end{tabular}
\normalsize




%% From hereon, to simplify the discussion (and implementation)
%% we shall restrict our attention to POMDPs with deterministic
%% actions. We note that POMDPs with stochastic
%% actions can be compiled into equivalent deterministic-action POMDPs,
%% where all the original action uncertainty is expressed in the
%% starting-state distribution~\cite{ng:Jordan:2000}. In our setting,
%% that of finite-horizon planning, such a compilation yields a finite
%% flat representation of the original POMDP, only with deterministic
%% actions.



%% The syntax and semantics for describing initial state distributions in
%% DTPDDL is taken verbatim from PPDDL. It is useful to present that
%% factored tree-like structure here,

%% \noindent Here, the {\em execution} declaration is a lifted description of
%% the sense action-precondition -- i.e,
%% $(\pp{look-for-object}~\pp{DORA}~\pp{box}~\pp{office})$ is equal to
%% $\poss_\stochActions(\pp{vision}~\pp{DORA}~\pp{box}~\pp{office})$. Above,
%% we include a redundant state-precondition, $(=(\pp{is-in}~?r)?l)$ in
%% order to fully demonstrate the syntax. Interpreting the above schema,
%% if action $(\pp{look-for-object}~\pp{DORA}~\pp{box}~\pp{office})$ is
%% executed, there is a $0.8$ chance of perceiving the box if it is in
%% the $\pp{office}$, and otherwise a $0.1$ chance of perceiving it.


%% \small
%% \begin{tabtt}
%% (\= :sense vision \+\\
%%  :parameters \= (\= ?r - robot ?v - visual-object\\
%%  \>\>  ?l - location) \\
%%  :execution \> ( \> look-for-object ?r ?v ?l) \\
%%  :precondition (and (= (is-in ?r) ?l) ) \\
%%  :effect \>  (  \> and (when (= (is-in ?v) ?l) \\
%%    \> \> (probabilistic 0.8 \\
%%    \>  \>(= (o-is-in ?v) ?l))) \\
%%   \> (when (not (= (is-in ?v) ?l)) \\
%%    \>  \> (probabilistic 0.1 \\
%%    \>  \> (= (o-is-in ?v) ?l))))) \\
%% \end{tabtt}
%% \normalsize

%% A
%% sense declaration corresponds to a lifted description of proposition
%% {\em senses}. Whereas the effects of an {\em operator} schema describe
%% how states change under application of actions, the effects of a {\em
%% sense} schema are perceptual, specifying the composition of an
%% observation following the execution of an action.

%% Also, and without a loss of generality, we find it
%% convenient to separate the actional precondition from the state
%% precondition.

%% Making the above ideas concrete with an example, suppose a robot
%% called $\pp{DORA}$ is able to look for a visual-object, such as a
%% $\pp{box}$, at a given place. We can model that deterministic action
%% using the following operator schema:


%% \small
%% \begin{tabtt}
%% (\=:action look-for-object \+ \\
%%    :parameters (\=?r - robot ?v - visual-object\\
%%    \> ?l - location) \\
%%    :precondition (and (= (is-in ?r) ?l) ) \\
%%    :effect (and (assign (reward) -3) ) ) \\
%% \end{tabtt}
%% \normalsize


%% \noindent In a slight departure from PPDDL, we suppose that
%% $\pp{look-for-object}$ has an effect on the state. That is, its
%% execution incurs an instantaneous penalty, $3$, that corresponds to
%% the {\em cost} of performing the visual search.\footnote{In PPDDL {\em
%% reward} is a {\em reserved word}, and occurs in {\em increased} and
%% {\em decreased} terms in operator schemata. In that setting, reward is
%% accumulated, whereas in DTPDDL it is instantaneous.}

%% The modelling language of choice for planning in probabilistic
%% problems is the Probabilistic Planning Domain Definition
%% Language~\cite{younes:etal:2005}. PPDDL has been
%% used in all three of the International Planning Competitions since
%% 2004. A variation on PDDL for describing domains with stochastic
%% actions and uncertain starting configurations, PPDDL is a declarative
%% first-order language that facilitates factored descriptions of domains
%% and problems. There are straightforward compilations from problems
%% expressed in PPDDL to propositional representations amenable
%% to state-of-the-art planners.

%% Because PPDDL cannot model domains that feature partial observability,
%% we develop an extension we call {\em Decision-Theoretic
%% (DT)PDDL}. This can express probabilistic models of the sensing
%% consequences of acting, to quantitatively capture unreliability in
%% perception. That expressive power is achieved by incorporating
%% perceptual analogues of fluent, predicate, and action definitions. In
%% detail, we have declarations of state characterising predicate and
%% fluent symbols according to the PPDDL syntax. In addition, we allow
%% two other declarations, for perceptual predicates and fluents
%% respectively. For example, suppose our robot is tasked with exploring
%% {\em locations} in order to identify the whereabouts of a {\em
%% visual-object}. We must describe state and perceptual facts that model
%% the {\em true}, resp. perceived, locations of objects. In DTPDDL,
%% these declarations appear as:




%% If a dual is executed, then the sequential planner must
%% replan with $\bstate_0$ equal to the underlying
%% belief-state. Otherwise, if $\actions.\poss(\switchAction)$ is
%% executed, then the current sequential plan is executed until further
%% sensing is scheduled, or to completion.



% \small
% \begin{tabtt}
% (\=:init (= (is-in DORA) kitchen) \+ \\
%        (probabilistic \=.8 (= (is-in box) office)  \\
% 		      \>.2 (= (is-in box) kitchen))) \\
% \end{tabtt}
% \normalsize

% \noindent and therefore, the belief-state $\bstate_0$ is:

% \small
% \begin{tabular}{cccc}
% \hline
% Probability & (is-in DORA)  & (is-in box)  & (is-in cup) \\
% \hline
% .8 & kitchen & office & \# \\
% .2 & kitchen & kitchen & \# \\
% \hline
% \end{tabular}
% \normalsize

 % With
% regards to abstraction, the contingent and serial sessions apply the
% same applicability conditions ---i.e., that each effect and condition
% $\prop$ is specified, $\prop\in\state$--- to actions from the DTPDDL
% problem description.

%%% Local Variables: 
%%% mode: latex
%%% TeX-master: "aaai11"
%%% End: 
