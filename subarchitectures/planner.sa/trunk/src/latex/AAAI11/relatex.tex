


%%  and
%% using various techniques to manage the large state
%% space

%% \citeauthor{hippo-jnl}~(\citeyear{hippo-jnl}) take this
%% approach in a 

%%  ---essentially the idea that
%% it does not matter what the value of the variable you are trying to
%% observe is---

Addressing task and observation planning specifically, there have been
a number of recent developments where the underlying problem is
modelled as a POMDP.
%%
For vision algorithm
selection,~\citeauthor{hippo-jnl}~(\citeyear{hippo-jnl}) exploit an
explicitly modelled hierarchical decomposition of the underlying
POMDP. \citeauthor{doshi08:pref_elic}~(\citeyear{doshi08:pref_elic})
represent a preference elicitation problem as a POMDP and take
advantage of symmetry in the belief-space to exponentially shrink the
state-space. Although we have been actively exploring
the \citeauthor{doshi08:pref_elic} approach, those exploitable
symmetries are not present in problems we consider due to the task
planning requirement.
%%
Finally, our approach is in a similar vein to {\em dual-mode}
control~\cite{cassandra96actingunder}, where planning switches between
entropy and utility focuses.

%% serial planning in
%% problem determinisations underlines strategies to goal achievement,
%% and 

%% in our case we use serial
%% planning in a determinisation of the underlying process to guide DT
%% sessions, which typically act to reducing entropy.


%% entropy reduction is targeted by planning in an abstract process which
%% is informed by one execution trace ---computed by a {\em classical}
%% planner--- and the underlying belief-state.


There has also been much recent work on scaling offline approximate
POMDP solution procedures to medium-sized instances. Recent
contributions propose more efficient belief-point sampling
schemes~\cite{kurniawati:etal:2010,shani:etal:08}, and factored
representations with procedures that can efficiently exploit
structures in those
representations~\cite{brunskill:russell:2010,shani:etal:2008}. Offline
domain independent systems scale to {\em logistics} problems with
$2^{22}$ states~\cite{shani:etal:2008}, taking over an hour to
converge, and around 10 seconds on average to perform each Bellman
backup. \citeauthor{brunskill:russell:2010} are able to solve problems
with approximately $10^{30}$ states, by further exploiting certain
problem features -- E.g., problems where no actions have negative
effects.
%%
Moving someway towards supporting real-time decision making, recent
online POMDP solution procedures have been developed which leverage
highly approximate value functions -- computed using an offline
procedure -- and heuristics in forward
search~\cite{ross:etal:2008}. These approaches are applicable in
relatively small problems, and can require
expensive \emph{problem-specific} offline processing in order to yield
good behaviours.
%%
A {\em very} recent and promising online approach for
larger POMDPs employs Monte-Carlo sampling to break the curse of
dimensionality in situations where goal reachability is {\em easily}
determined~\cite{silver:veness:2010}. 


In the direction of leveraging {\em classical} systems/approaches for
planning under uncertainty, the most highlighted system to date has
been \system{FFR$_a$}~\cite{yoon:etal:2007}; The winning entry from
the probabilistic track of the 2004 International Planning
Competition.  In the continual paradigm, \system{FFR$_a$}
uses \system{FF} to compute sequential plans and execution traces.
%%
More computationally expensive approaches in this vein combine
sampling strategies on valuations over {\em runtime variables} with
deterministic planning procedures~\cite{yoon:etal:2008}. %% The outcome is typically a more
%% robust sequential plan, or contingent
%% plan~\cite{majercik:2006}. 

Also leveraging deterministic planners in problems that feature
uncertainty, \system{Conformant-FF}~\cite{hoffmann:brafman:2006} and
$T_0$~\cite{palacios:geffner:2009} demonstrate how conformant planning
---i.e., sequential planning in unobservable worlds--- can be modelled
as a deterministic problem, and therefore solved using sequential
systems. In this conformant setting, advances have been towards
compact representations of beliefs amenable to existing best-first
search planning procedures, and lazy evaluations of beliefs. Most
recently this research thread has been extended to contingent planning
in fully observable non-deterministic
environments~\cite{albore:etal:2009}.


%% Their applicability in our setting is
%% limited, firstly because they only scale to smaller problems, with
%% thousands of states, and also due to the large amount of
%% \emph{problem-specific} offline processing that might be required to get useful
%% search guidance. 


%%  In that
%% setting \citeauthor{kurniawati:etal:2010}~(\citeyear{})
%% recently addressed an inefficiency of offline point-based techniques
%% in problems with medium length planning horizons, however their
%% approach take tens of minutes to good plans, and only scales to tens
%% of thousands of states. In the case of general domain-independent
%% factored systems, the state-of-the-art scales to relatively small
%% problems with $2^{22}$ states~\cite{shani:etal:2008}. 

%% At their limit,
%% these procedures take over an hour to converge, and $\sim10$ seconds
%% on average to perform a single Bellman backup.  For classes of POMDP
%% that feature exploitable structures (e.g., no actions with negative
%% effects), problems with as many as $10^{30}$ states can be targeted by
%% such offline procedures~\cite{brunskill:russell:2010}. 


%% Although we suppose it an interesting
%% item for future work to pursue that direction, it should be noted that
%% ease of goal reachability is not guaranteed in the problems we face,
%% and is certainly not a property to be assumed in domain independent
%% planning.



%%
%% The continual planning system that motivated our
%% project~\cite{brenner:nebel:jaamas09} also has this characteristic,
%% and has been applied in completely observable domains, particularly
%% those featuring multiple communicating agents. 

%% The use of knowledge
%% operators in domains allows plans that act to gain knowledge, but the
%% approach assumes that such actions are deterministic and reliable, an
%% assumption that we relax.



%%No! They simply haven't been evaluated in PO settings. They may, or
%%may not struggle. They have sampling of traces, and that would
%%include observations, and therefore evolutions of beliefs. SSAT was
%%proposed by Littman for POMDPs. So the majercik stuff is perfectly
%%suited to POMDPs.

% Normally if we are going to compare with related work, we do
% actually *compare*. Why didn't you try those approaches? I think
% it's safe to say that FFR will struggle. Why would it even include
% in its plan an observational action that doesn't change the world?

%%
%% However, as we said in the introduction,
%% all these approaches struggle in partially observable domains as they
%% rely on being able to determine the state at all times.
